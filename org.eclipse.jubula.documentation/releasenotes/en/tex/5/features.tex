\subsection{New features and developments}
\textbf{Code coverage}
\begin{itemize}
\item \app{} now lets you run code coverage for Java applications written in Java 1.5 or higher.
\item The code coverage tool \bxname{JaCoCo} is supported out-of-the-box and can be added to your \gdaut{} configuration in the \gdaut{} configuration dialog. 
\item You can view the results of the code coverage monitoring from the \gdtestsummaryview{}. 
\end{itemize}

\textbf{Teststyle}
\begin{itemize}
\item \app{} has a new feature to improve the quality and structure of the tests you write.
\item The \bxname{Teststyle} plugin lets you define guidelines for your \gdproject{} and informs you when a guideline is not upheld. 
\item You can set the message level (information, warning, error) for the guidelines as well as the context they should be valid for.
\item \bxname{Teststyle} is added by default to all \gdprojects{}. You can alter the Teststyle settings via the \gdproject{} properties.
\end{itemize}

\textbf{New license mechanism}
\begin{itemize}
\item The license mechanism in \app{} has been changed. 
\item \app{} no longer uses a license server to manage floating licenses. The new individual user licenses can be requested from the \app{} team as part of your maintenance contract. 
\item The licenses can be entered in the Preferences, under: \bxname{\app{} - Licenses}.
\end{itemize}

\textbf{ID Attribute Name for HTML \gdauts{}}
\begin{itemize}
\item Each supported component in HTML \gdauts{} can have its own attribute used to identify it during test execution.
\item In your \gdaut{} configuration, you can define an attribute name which should be used as an identifier for components. 
\item For example if your attribute name is: 
 \bxshell{testid} \\
(e.g. \bxshell{<div testid=''Username''></div>}) \\
then you would enter \bxname{testid} in the \gdaut{} configuration. 
\end{itemize}


\textbf{Test debugging - continue without \gdehandler{}}
\begin{itemize}
\item There is a new option in \app{} to make running interactive tests more comfortable. 
\item On the toolbar, there is a new button to \bxcaption{Continue without \gdehandler{}}. You can use this button in combination with the \bxcaption{Pause on Error} option. 
\item If a test you are running encounters an error, the test is paused die to the \bxname{Pause on Error} option. You can then click \bxcaption{Continue without \gdehandler{}} to ignore the error and continue as if no error had occurred. 
\end{itemize}

\textbf{Copy ID to clipboard and open \gdcase{} via its ID}
\begin{itemize}
\item There is a new option in the \gdtestcasebrowser{} to copy the unique ID for a \gdcase{} to the clipboard.
\item You can use this ID to refer to the \gdcase{} from external systems such as bug-tracking systems or requirements management systems. 
\item You can open a \gdcase{} based on its ID by pressing \bxkey{Shift+F9}.
\end{itemize}

\textbf{Support information package}
\begin{itemize}
\item To reflect the changes to the support system, the option to send a support email has been updated to create a .zip file as a support information package.
\item Should you find an error, you can upload the information package to Bugzilla as a part of the error description.
\end{itemize}

\textbf{New \gddb{} configuration}
\begin{itemize}
\item The configuration tool and \gddb{} configuration tool are no longer a part of the installation package. 
\item The configuration options for the \gddb{} connections are now in the preferences under: \bxname{Test - Database Connections}.
\item The embedded \gddb{} is configured as the default. You can add other \gddb{} configurations using the buttons and the dialog to add \gddb{} connection details.
\item You can export and import your \gddb{} preferences using the \bxmenu{File}{Export}{} menu.
\end{itemize}

\textbf{New -data parameter in the test executor and dbtool}
\begin{itemize}
\item As the \gddb{} configurations are now in the preferences, you must enter a workspace (where the correct preferences are saved) for test execution via the test executor and for working with the dbtool.
\item The workspace parameter is \bxshell{-data <path to workspace>}. 
\item The workspace parameter cannot currently be entered in the configuration file for the test executor, it must be entered directly into the command line interface.
\end{itemize}


\textbf{\gdproject{} migration to the new version}
\begin{itemize}
\item We have changed our \gddb{} abstraction layer from Hibernate to JPA/Eclipse Link.
\item For this reason, the migration of your \gdprojects{} to the new version cannot be carried out automatically by the \gddb{} migration tool. 
\item Instructions on migrating your \gdprojects{} and clearing your \gddb{} scheme are in the Installation Manual.
\end{itemize}

\textbf{Amount of results shown in the \gdtestsummaryview{} is now configurable}
\begin{itemize}
\item  You can now configure how many results should be shown in the \gdtestsummaryview{} based on how old they are.
\item In the \bxname{Test result} preferences, you can specify how many days' worth of test result summaries should be shown. The default is set to 30 days.
\item You can see the current setting in the \gdtestsummaryview{}.
\item Results not displayed are not removed from the \gddb{} and are still considered for report generation. They are simply not visible to make filtering and sorting over more recent results more comfortable.
\end{itemize}

\textbf{Two new BIRT reports}
\begin{itemize}
\item There are two new reports that can be generated from the \gdtestsummaryview{}. 
\item The first is a report that shows code coverage for the chosen \gdsuites{}.
\item The second is a report that shows any comments written for unsuccessful \gdsuites{} which can be used as a daily summary of the test status.
\end{itemize}

\textbf{BIRT reports show comment title for executed \gdsteps{}}
\begin{itemize}
\item If you have a BIRT report which shows the executed \gdsteps{} for a \gdsuite{}, you can hover over the point on the graph marking the number of executed \gdsteps{} to see the title of any comment you wrote for the test run.
\item If the expected \gdsteps{} and executed \gdsteps{} are the same (i.e. the whole test ran), you will not see the comment.
\end{itemize}

\textbf{Test execution speed increased}
\begin{itemize}
\item We have significantly increased the speed at which tests are executed.
\item Should you want to slow a test down to watch it, then you can set a \bxname{step delay} in the \gdpropview{} for the \gdsuite{}.
\item You may notice that some tests require explicit synchronization (e.g. waiting for windows to appear or to close) to ''keep up'' with the increased speed. If you prefer, then adding a \bxname{step delay} of 1000(ms) should produce the test execution speed from previous versions. 
\end{itemize}

\textbf{Extensions must be adapted}
\begin{itemize}
\item As there have been changes to our API, any extensions you are using in your \gdprojects{} must be adapted to the new changes. 
\item The largest change is that package prefixes have been changed. Instead of \bxshell{com.bredexsw.guidancer}, the new name is \bxshell{org.eclipse.jubula}.
\item Other changes include modified class names (the GD prefix has been removed or replaced by JB). 
\item Your client and RC (server) extensions must be adjusted and recompiled with the new Jar files.
\end{itemize}

\textbf{Support for div containers in HTML}
\begin{itemize}
\item We now support components created using <div></div> containers on the level of \bxname{Graphics Component}.
\item This means you will be able to perform certain checks on such components and also send clicks to them.
\end{itemize}

\textbf{Support for lists in HTML}
\begin{itemize}
\item You can now check the existence of entries in lists created in HTML using the <ol> and <ul> tags.
\end{itemize}

\textbf{Support for sliders, date and time, spinner and scale components in SWT}
\begin{itemize}
\item The SWT components \bxname{slider, date and time, spinner} and \bxname{scale} components in SWT are now supported as \bxname{Graphics Components}.
\item This means you will be able to perform certain checks on the components as well as click them or in them.
\end{itemize}

\textbf{New names for various \app{} components}
\begin{itemize}
\item We have renamed various parts of \app{}:
\begin{itemize}
\item What we previously referred to as the client (the application where tests are written) is now known as the Integrated Test Environment (\ite{}).
\item The \bxname{gdrun} command for starting \gdauts{} is now called \bxname{autrun}. The command is also now called \bxshell{autrun} and must be changed in any scripts that use it.
\item The command line client for test execution is now called the \bxname{test executor}. The command is called \bxshell{testexec} and must be changed in any scripts that use it. 
\item The RCP Accessor is now called the RCP Remote Control Plugin. 
\item The environment variables that can be used in \app{} have also been renamed:
\begin{itemize}
\item \bxname{GD\_COMP\_NAME} (used for setting a unique name for RCP components) is now called \bxname{TEST\_COMP\_NAME}.
\item \bxname{GDAutStarterPort} (used to set a default port number for the system) is now called \bxname{TEST\_AUT\_AGENT\_PORT}.
\item \bxname{GDUDV\_} (used to set your own environment variables is now called \bxname{TEST\_UDV}.
\item The pre-defined test execution variables are now preceded with \bxname{TEST\_} instead of \bxname{GD\_}. 
\end{itemize}
\end{itemize}
\end{itemize}

\textbf{New perspective names}
\begin{itemize}
\item The perspectives have been renamed. Each perspective type (specification, reporting, execution) is preceded by \bxname{Functional Testing}. 
\end{itemize}

\textbf{New and moved menu options}
\begin{itemize}
\item The \ite{} now contains a menu point:\bxmenu{File}{}{}.
\item This menu allows you to import and export \gddb{} preferences to \app{}, to save, rename, refresh and exit the \ite{}.
\item The previous \bxmenu{Project}{}{} menu is now called \bxmenu{Test}{}{}. Here you will find the menu options that used to be under \bxmenu{Project}{}{}, including creating, deleting, importing and exporting \gdprojects{}. 
\end{itemize}

\textbf{Moved preferences}
\begin{itemize}
\item The preferences for the test can now be found in the preference dialog under the option \bxname{Test}. 
\end{itemize}


\textbf{New BIRT version}
\begin{itemize}
\item If you create your own reports for \app{} from this version, then you must use version 2.6.1 of the BIRT viewer and engine. Our reports are created with this version of BIRT and Eclipse version 3.6.1.
\end{itemize}





