\textbf{New information in \gdomm{} : property information}
\begin{itemize}
\item When you collect a component in the \gdomm{} in a Swing or SWT/RCP \gdaut{}, you can now see information on the properties of that component in the \gdpropview{}.
\item The information is shown as a list of properties with their values. If a value cannot be accessed, this is communicated.
\item The information is not saved - to see the properties again, you must re-collect the component in the \gdomm{}.
\item This can be used to help you write tests that use the actions \bxname{check property} or \bxname{store property}.
\end{itemize}

\textbf{Show where used also available for \gdsuites{}}
\begin{itemize}
\item You can now use ''show where used'' on \gdsuites{} to find out which \gdjobs{} they have been reused in.
\end{itemize}

\textbf{Change tracking in \gdprojects{}}
\begin{itemize}
\item You can now configure in the \gdproject{} properties that changes to \gdcases{}, \gdsuites{} and \gdjobs{} in the \gdproject{} are tracked. You can configure a system / environment property that is shown alongside the timestamp of the change to track e.g. who made the changes. 
\item When change tracking is activated, when you save an editor for a \gdcase{}, \gdsuite{} or \gdjob{}, the timestamp of the save as well as the property will be saved and displayed in the properties view.
\item You can specify the amount of changes to track per node, or the length of time that change tracking information should be kept. Once the time has passed or the amount of changes has been reached, the next saving of the node will result in invalid change information being discarded.
\item You can remove all change tracking information from the \gdproject{} via the \gdproject{} properties. 
\end{itemize}

\textbf{\bxkey{F3} always opens selected item}
\begin{itemize}
\item You can now press \bxkey{F3} on any \gdcase{}, \gdsuite{}, or \gdjob{} to open the specification of that item.
\item This now includes the actual original specification of the item itself, to avoid confusion in larger \gdprojects{}. 
\end{itemize}
