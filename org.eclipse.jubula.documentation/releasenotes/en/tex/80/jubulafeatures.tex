\textbf{Support for JavaFX \gdauts{}}
\begin{itemize}
\item We have added support for testing JavaFX \gdauts{}. 
\item There is an example Simple Adder \gdaut{} in the \bxname{<installation-directory>/examples/AUTs} folder. You can use this to perform example tests as described in the cheat sheets.
\item You can see the overview of the support via \url{http://eclip.se/421595}.
\item The toolkit is extensible, and we welcome feedback and contributions.
\end{itemize}

\textbf{ALM integration: Support for commenting on HP ALM tasks after a test run}
\begin{itemize}
\item You can now connect your \ite{} to a HP ALM repository (version 11+) to view tasks from the repository in the \ite{}.
\item You can also add comments to tasks in a HP ALM repository when a test has run.
\item To use the HP ALM integration, you must use a separate connector for HP ALM which may incur license costs. Visit the Tasktop website for more details \url{http://www.tasktop.com}.
\end{itemize} 

\textbf{ALM integration: Reporting to ALM systems after headless execution now supported}
\begin{itemize}
\item When a test has run via testexec, you can now trigger the reporting to ALM tools manually from the \gdtestsummaryview{}. 
\item This allows you to add the test results from your continuous integration tests to your issues, bugs and features in your ALM system. 
\end{itemize}

\textbf{ALM integration: Amount of reported items shown}
\begin{itemize}
\item When reporting to ALM systems takes place, you can now see how many tasks were reported to, how many results were written to each task, and a total amount of tasks written to in the console.
\end{itemize}

\textbf{Support for embedded \gdagent{} in testexec}
\begin{itemize}
\item When executing tests via the command line using the \bxname{testexec} tool, you can now choose to use an embedded \gdagent{} for the test.
\item This means that you do not have to start a separate \gdagent{} on the machine you are testing on.
\item To use an embedded \gdagent{}, don't enter the \bxname{-server} parameter. You can either enter a port number for the embedded \gdagent{} to be started on, or you can also leave out the \bxname{-port} parameter to start the \gdagent{} on a dynamically chosen port.
\end{itemize}

\textbf{Create New Version is now available in the dbtool}
\begin{itemize}
\item You can now create new versions of a \gdproject{} using the dbtool.
\item Use \bxshell{-createVersion <project-name> <old-version> <new-version>} to create a new version of an existing \gdproject{}.
\end{itemize}

\textbf{Change tracking in \gdprojects{}}
\begin{itemize}
\item You can now configure in the \gdproject{} properties that changes to \gdcases{}, \gdsuites{} and \gdjobs{} in the \gdproject{} are tracked. You can configure a system / environment property that is shown alongside the timestamp of the change to track e.g. who made the changes. 
\item When change tracking is activated, when you save an editor for a \gdcase{}, \gdsuite{} or \gdjob{}, the timestamp of the save as well as the property will be saved and displayed in the properties view.
\item You can specify the amount of changes to track per node, or the length of time that change tracking information should be kept. Once the time has passed or the amount of changes has been reached, the next saving of the node will result in invalid change information being discarded.
\item You can remove all change tracking information from the \gdproject{} via the \gdproject{} properties. 
\end{itemize}

\textbf{Support for iOS 7}
\begin{itemize}
\item You can now also test \gdauts{} written with iOS 7.
\end{itemize}

\textbf{New information in \gdomm{} : property information}
\begin{itemize}
\item When you collect a component in the \gdomm{} in a Swing or SWT/RCP \gdaut{}, you can now see information on the properties of that component in the \gdpropview{}.
\item The information is shown as a list of properties with their values. If a value cannot be accessed, this is communicated.
\item The information is not saved - to see the properties again, you must re-collect the component in the \gdomm{}.
\item This can be used to help you write tests that use the actions \bxname{check property} or \bxname{store property}.
\end{itemize}

\textbf{Component match heuristic shown for each component in test results}
\begin{itemize}
\item In the HTML and XML test result reports, and in the result reports in the \ite{}, you can now see the value with which the component was found when the test was executed.
\item This lets you see whether a component is ''easily'' found, or whether a small change may lead to it no longer being found, so that a preventative remapping may be worthwhile.
\end{itemize}

\textbf{Graphics component support for Canvas}
\begin{itemize}
\item You can now map org.eclipse.ui.forms.widgets that inherit directly or indirectly from canvas. 
\item This includes, e.f. org.eclipse.ui.forms.widgets.Hyperlink.
\item You can use actions such as click and check existence (and other actions on the graphics component level) on these components.
\end{itemize}

\textbf{Show where used also available for \gdsuites{}}
\begin{itemize}
\item You can now use ''show where used'' on \gdsuites{} to find out which \gdjobs{} they have been reused in.
\end{itemize}

\textbf{\bxkey{F3} always opens selected item}
\begin{itemize}
\item You can now press \bxkey{F3} on any \gdcase{}, \gdsuite{}, or \gdjob{} to open the specification of that item.
\item This now includes the actual original specification of the item itself, to avoid confusion in larger \gdprojects{}. 
\end{itemize}

\textbf{Luna \gdauts{} supported}
\begin{itemize}
\item \gdauts{} that use the Eclipse Luna platform can also be tested with the \ite{}. This was also possible in version 7.2.
\end{itemize}

\textbf{Export improved}
\begin{itemize}
\item The export mechanism has been improved to use less memory, allowing larger \gdprojects{} to be exported.
\end{itemize}

\textbf{Toggling relevance of test runs}
\begin{itemize}
\item You can now toggle the relevance of multiple test runs at once.
\end{itemize}

\textbf{New option for content assist in the \gdcompnamesview{}}
\begin{itemize}
\item You can now configure a delay for the content assist in the \gdcompnamesview{}.
\item In the Preferences (in the \bxname{Test} section, you can enter a delay in milliseconds. This amount of time will be waited after each character entry before opening the content assist.
\item If you have many component names, and / or don't require the content assist as often, then you can increase this value to be able to type more of your name before seeing content assist.
\item We have also adapted the behaviour of the content assist to not appear if you remove all of the content of the \bxname{new name} field in the \gdcompnamesview{}. 
\item You can press \bxkey{ctrl+space} at any time to open the content assist.
\end{itemize}

\textbf{Log files now split after 10MB}
\begin{itemize}
\item Log files that get larger than 10MB are now automatically split and zipped.
\end{itemize}

\textbf{Closing HTML \gdauts{} via the close button}
\begin{itemize}
\item If you close a HTML \gdaut{} by closing the browser, the \ite{} will correctly notice the closure after 5 seconds (configurable) and will remove the \gdaut{} from the running \gdauts{} view.
\item The mechanism works by polling the \gdaut{}, and if it is no longer there after the configured time, the \gdaut{} is considered to be stopped.
\item If your \gdaut{} may sometimes be unreachable for longer than the default 5 seconds, you can change this time by using a process or system property:\\
\bxshell{TEST\_MAX\_AUT\_RESPONSE\_TIME=<timeInMs>}.
\item Further information on this is available in this issue:\\
\url{http://bugzilla.bredex.de/1391}.
\end{itemize}


