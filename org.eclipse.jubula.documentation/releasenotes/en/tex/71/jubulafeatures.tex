\textbf{Search scopes supported}
\begin{itemize}
\item The search dialog now lets you choose whether you want to search the whole \gdproject{}, or just the node(s) you have selected. 
\item If you choose to just search within the nodes you have selected, then you can specify whether you want to use the selected node in the \gdtestsuitebrowser{}, the \gdtestcasebrowser{}, or both. 
\end{itemize}

\textbf{New mass replace operation for \gdcases{}}
\begin{itemize}
\item From the \gdsearchresultview{}, you can now perform a mass replace of \gdcases{}. 
\item If you have searched, e.g. for all places where a \gdcase{} is reused, and want to replace some or all of those places with a new \gdcase{}, then you can select the option to replace the current \gdcase{} references with a new \gdcase{} reference. 
\item Any comments, \gdcase{} reference names and the commented-out state of each replaced \gdcase{} are transferred onto the new \gdcase{} reference. 
\item You can also match any propagated component names and referenced parameters from the old \gdcase{} to propagated component names and parameters from the new \gdcase{}. 
\end{itemize}

\textbf{New function for accessing node attribute}
\begin{itemize}
\item A new function has been added to the set of pre-defined functions in \app{}.
\item The function is: \bxname{?getNodeAttribute()} and it can have either \bxname{name} or \bxname{comment} as arguments.
\item When \bxname{name} is chosen, the function reads the name of the node (e.g. \gdcase{}, \gdstep{}) on which it is resolved and uses this for the test.
\item  When \bxname{comment} is chosen, the function reads the comment of the node (e.g. \gdcase{}, \gdstep{}) on which it is resolved and uses this for the test.
\end{itemize}

\textbf{New function for accessing values in central data sets}
\begin{itemize}
\item There is a new function available in the pre-defined functions in \app{}.
\item The name of the new function is: \bxname{?getCentralTestDataSetValue()}, and it requires four arguments:
\begin{description}
\item [dataSetName:]{The name of the central data set to search in.}
\item [keyColumnName:]{The column that is used to find the correct line of the data set.}
\item [entryKey:]{The value to locate in the key column that will provide the correct line.}
\item [columnName:]{The name of the column in which the value to be chosen can be found.}
\end{description}
\item Within the specified data set, the required value is located based on the line found using the key column and entry key, as well as the column name for the actual value. 
\end{itemize}

\textbf{New option to change the column in a central test data set used by \gdcases{}}
\begin{itemize} 
\item If you have used a central test data set in multiple \gdcases{} and later realize that you have two columns in the central test data set that contain the same information, then you can change all \gdcases{} that use this central test data set to just use one column. Once you have done this, you can remove the unnecessary column from the central test data set.   
\item If you have searched, e.g. for all places where a central test data set is reused, and want to alter \gdcases{} that use this central test data set to use a different column in the central test data set, then you can select the option to change the central test data set column usage.
\item You can change an existing parameter present in the selected \gdcases{} to target a different column in the central data set using the dialog that appears.  
\end{itemize}


\textbf{Multi-window support for HTML \gdauts{}}
\begin{itemize}
\item If your HTML \gdaut{} uses multi windows (e.g. pop-ups), then you can now specify this in the \gdaut{} configuration. 
\item \gdauts{} that are running in multi-window mode show the Selenium console as well as the \gdaut{} when the \gdaut{} is started. 
\item The \gdomeditor{} has a new button to allow switching between multiple open windows for mapping components, and there are new actions in the HTML unbound modules to allow you to switch between windows during the test. 
\end{itemize}

\textbf{On-click expansion for pictures in HTML reports}
\begin{itemize}
\item In HTML reports, you can now click on any screenshots taken on errors and they will be expanded to make the details more visible.
\item Clicking the image again will reduce it to its original size in the test result report.
\end{itemize}

\textbf{Kepler \gdauts{} can now be tested}
\begin{itemize}
\item It is now possible to test \gdauts{} that are based on Kepler.
\item Due to improvements in the object recognition, some components for RCP \gdauts{} may need to be remapped.
\end{itemize}

\textbf{New actions to correctly shutdown and restart an \gdaut{}}
\begin{itemize}
\item Users testing Swing and RCP \gdauts{} now have two new actions to \bxname{Prepare an \gdaut{} for Termination} and to \bxname{Synchronize the termination and re-start}. 
\item These actions can be used instead of the \bxname{restart} action for cases where the \gdaut{} cannot be restarted using this action, or when an \gdaut{} needs to be correctly shutdown and restarted as part of a test (e.g. to save a workspace, settings etc.).
\item The actions must be placed on either side of any action that will cause the \gdaut{} to terminate (not to restart). The \bxname{prepare for termination} action must come first, followed (not necessarily directly) by an action to close the \gdaut{} (e.g. selecting \bxcaption{Exit}). Directly after this action, the \bxname{synchronize termination and re-start} action must be used.
\item The action to close the \gdaut{} must be an action that causes the \gdaut{} to run through the JVM shutdown hooks to properly terminate. Do not use the \bxname{restart} action to shut your \gdaut{} down.
\item The combination of these three actions allows the \gdaut{} to be shutdown correctly and restarted. 
\end{itemize}
