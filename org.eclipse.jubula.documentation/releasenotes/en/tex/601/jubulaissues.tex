\textbf{Mac OSX Mountain Lion gatekeeper prevents install4j generated installer from starting}
\begin{itemize}
\item This is caused by a too-restrictive setting on the Mountain Lion gatekeeper security feature.
\item Setting the security level to Mac OSX Lion compatible level resolves the issue.
\end{itemize}


\textbf{Documentation error in Chronon section}
\begin{itemize}
\item The documentation for using Chronon while running an \gdaut{} specifies that the package pattern must be a regular expression. This is incorrect. The correct text should be: 
\begin{quote}Enter a comma-separated list of packages that you want to be covered by the monitoring. The packages must adhere to the patterns as defined in the Chronon documentation, for example \bxshell{com.myorg.**} selects the whole com.myorg namespace. \bxshell{com.myorg.*} selects only classes in the com.myorg package. The documentation for the patterns is located at \\
\href{https://chronon.onconfluence.com/display/DOC/Include+and+Exclude+patterns}{https://chronon.onconfluence.com/display\\
/DOC/Include+and+Exclude+patterns}.\\
 If you enter no patterns, the recording file will be empty.
\end{quote}
\end{itemize}

\textbf{Chronon and continuous integration}\\

Although you could technically use Chronon with \app{} outside of the \ite{} in Continuous Integration environments, neither BREDEX nor Chronon Systems recommends or will provide support for such cases.

The current integration with Chronon is meant to be used inside of the \ite{}. If you do want to record your tests outside of the \ite{}, please look at Chronon Systems' Chronon recording server:\\
 \href{http://www.chrononsystems.com/products/chronon-recording-server/}{http://www.chrononsystems.com/products/chronon-recording-server/}\\
and Embedded Chronon:\\
\href{http://www.chrononsystems.com/products/embedded-chronon}{http://www.chrononsystems.com/products/embedded-chronon} \\
 offerings which are explicitly designed for and support such use cases.

