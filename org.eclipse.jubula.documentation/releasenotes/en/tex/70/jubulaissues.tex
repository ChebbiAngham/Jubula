\textbf{Migration of existing toolkit extensions}

Due to major refactoring in the remote control code of \app{} you must migrate any toolkit extensions you have written. Theses changes only affect the remote control extension. The \ite{} extensions must not be changed. As a first step, we recommend reading the Extension Manual. Most information about the new implementation and a simple example can be found there.

To migrate any existing extensions, the following steps must be performed:
\begin{itemize}
\item Define a target platform for the remote control enviroment as mentioned in the \textbf{Extension Manual 2.3.1}
\item Alter your remote control extension project to be an OSGi-fragment and set the appropiate host bundle. See \textbf{Extension Manual 2.3.1}
\item Adapt your current implementation as described below.
\item Deploy your fragment as described in \textbf{Extension Manual 2.3.2}
\end{itemize}

All 'old' implementation classes (*ImplClass) have been removed and rewritten. If you previously used the \app{} implementation classes  you will have to change the class you are extending:
\begin{itemize}
\item \bxshell{org.eclipse.jubula.rc.common.tester}\\ in this package the base classes for most of the
  TesterClasses are located. Some toolkit-specific implementations are in the toolkit bundles named below.
\item \bxshell{org.eclipse.jubula.rc.swt.tester}\\ SWT-specific implementation for \bxshell{List}, \bxshell{Table}, \bxshell{Tree}, \bxshell{Application} and \bxshell{Menus}
\item \bxshell{org.eclipse.jubula.rc.swing.tester}\\ Swing-specific implementation for \bxshell{JList}, \bxshell{JTable}, \bxshell{JTree}, \bxshell{Application} and \bxshell{Menus}
\end{itemize}

You must also write your own adapter factory (\textbf{Extension Manual 2.3.1 Create an Adapter}), which must be in the package \bxshell{org.eclipse.jubula.rc.common.adapter}.

This adapter factory manages adapters that are needed for the tester classes. The adapters are necessary for the tester classes because the tester classes are toolkit-independent.

In the adapter factory you write, you must define which component (\bxname{widget}) is adapted by which class. For example, a \bxname{JTable} is adapted by the \bxname{JTableAdapter}. For this you can reuse existing SWT or Swing adapters or you can write your own. If you want to write your own adapter it is important that you use the \app{} interfaces of the adapter, because the tester classes require this information.  

The following packages are important for adapters:
\begin{itemize}
\item \bxshell{org.eclipse.jubula.rc.swt.tester.adapter}\\ SWT adapter.
\item \bxshell{org.eclipse.jubula.rc.swing.tester.adapter}\\ Swing adapter.
\item \bxshell{org.eclipse.jubular.rc.common.tester.adapter.interfaces}\\Interfaces for all adapters.
\end{itemize}

\textbf{Migration of GEF accessibility identifer}

Due to changes in the structure of our RCP accessor we have changed the location of the GEF implementation. The bundle has changed to \bxshell{org.eclipse.jubula.rc.rcp.e3} and also the packages now begin with \bxshell{org.eclipse.jubula.rc.rcp.e3}. To migrate your existing plugin you have to do following steps
\begin{itemize}
\item Change the target definition as described in the \textbf{Extension Manual 3.1.1}
\item Change the bundle in the MANIFEST.MF\\ from \bxshell{org.eclipse.jubula.rc.rcp}\\ to \bxshell{org.eclipse.jubula.rc.rcp.e3}
\item Change the adapter type in the plugin.xml to \\ \bxshell{org.eclipse.jubula.rc.rcp.e3.gef.identifier.IEditPartIdentifier}
\item Change all packages in your classes to the new appropriate packages
\end{itemize}

\textbf{Changes to resolving of referenced parameters}\\
\begin{itemize}
\item In previous versions, there was an error in the way referenced parameters were resolved, which lead to the data for the parameter being read each time the parameter was referenced. This could lead to problems with e.g. time-sensitive functions. If a \gdcase{} was structured to enter the date using the ?now() function and then check the entered text (where the data for the text entry and the check were referenced and ?now() entered at the parent \gdcase{}), then the entered value and checked value would not be the same.
\item This has been changed for this version, so that references are resolved at the place where the parameter is defined. 
\item Any users who have \gdcases{} whose parameters define variables which are declared within the \gdcase{} itself will have to restructure such \gdcases{} so that the variable is not set before it is called, e.g.
\begin{quote}
-TC1 [VARIABLE\_SET=VAR1;VARIABLE\_CHECK=\$VAR1]\\
-- Set variable =VARIABLE\_SET\\
-- Check variable =VARIABLE\_CHECK
\end{quote}
\item The correct way to structure such \gdcases{} is as follows:
\begin{quote}
-TC1\\
-- Set variable: VAR1\\
-- Check variable:\$VAR1
\end{quote}
\end{itemize}

\textbf{Removal of option \bxcaption{Create support information package}}
\begin{itemize}
\item The menu option: \\
\bxmenu{Help}{Create support information package}{}\\
has been removed.
\end{itemize}

\textbf{Chronon support for \gdauts{} now with separate installation}
\begin{itemize}
\item You can now only use a separate installation of Chronon as a monitoring agent for your \gdaut{}. 
\item The support for the embedded Chronon recording in \gdauts{} has been removed.
\end{itemize}

\textbf{Some categories, component names and unbound modules renamed}
The following items have been renamed in the unbound modules:
\begin{description}
\item [Combo boxes]{ have been renamed to \bxname{Combo Components}. This is also evident in the abbreviation used for the unbound modules and the component names. It has changed from \bxname{cbx} to \bxname{cbc}.}
\item [Tabbed panes]{ have been renamed to \bxname{Tabbed Components}. This is also evident in the abbreviation used for the unbound modules and the component names. It has changed from \bxname{tpn} to \bxname{tbc}.}
\item [Text field / text area / text pane]{ has been renamed to \bxname{Text Component}. This is also evident in the abbreviation used for the unbound modules and the component names. It has changed from \bxname{txf} to \bxname{txc}.}
\end{description}

\textbf{Known issue with using observation mode on Gnome}
\begin{itemize}
\item When using the observation mode on Gnome, \app{} may close unexpectedly. 
\item This is documented in issue \url{http://bugzilla.bredex.de/729}.
\end{itemize}

\textbf{Deprecated modules and \gdsteps{} removed}
\begin{itemize}
\item All unbound modules that were preceded with \bxname{DEPRECATED} have been removed. These modules should be removed from use in your \gdprojects{} before switching to the new version. 
\item All \gdsteps{} that had the status \bxname{deprecated} have also been removed. You should ensure that you no longer use deprecated \gdsteps{} before switching to the new version. 
\end{itemize}

\textbf{Updated migration information}
\begin{itemize}
\item The information on migrating to new versions has been updated.
\item The Installation Manual contains the migration information. 
\end{itemize}
