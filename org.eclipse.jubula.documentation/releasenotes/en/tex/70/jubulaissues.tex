\textbf{Changes to resolving of referenced parameters}\\
\begin{itemize}
\item In previous versions, there was an error in the way referenced parameters were resolved, which lead to the data for the parameter being read each time the parameter was referenced. This could lead to problems with e.g. time-sensitive functions. If a \gdcase{} was structured to enter the date using the ?now() function and then check the entered text (where the data for the text entry and the check were referenced and ?now() entered at the parent \gdcase{}), then the entered value and checked value would not be the same.
\item This has been changed for this version, so that references are resolved at the place where the parameter is defined. 
\item Any users who have \gdcases{} whose parameters define variables which are declared within the \gdcase{} itself will have to restructure such \gdcases{} so that the variable is not set before it is called, e.g.
\begin{quote}
-TC1 [VARIABLE\_SET=VAR1;VARIABLE\_CHECK=\$VAR1]\\
-- Set variable =VARIABLE\_SET\\
-- Check variable =VARIABLE\_CHECK
\end{quote}
\item The correct way to structure such \gdcases{} is as follows:
\begin{quote}
-TC1\\
-- Set variable: VAR1\\
-- Check variable:\$VAR1
\end{quote}
\end{itemize}

\textbf{Removal of option \bxcaption{Create support information package}}
\begin{itemize}
\item The menu option: \\
\bxmenu{Help}{Create support information package}{}\\
has been removed.
\end{itemize}

\textbf{Chronon support for \gdauts{} now with separate installation}
\begin{itemize}
\item You can now only use a separate installation of Chronon as a monitoring agent for your \gdaut{}. 
\item The support for the embedded Chronon recording in \gdauts{} has been removed.
\end{itemize}

\textbf{Some categories, component names and unbound modules renamed}
The following items have been renamed in the unbound modules:
\begin{description}
\item [Combo boxes]{ have been renamed to \bxname{Combo Components}. This is also evident in the abbreviation used for the unbound modules and the component names. It has changed from \bxname{cbx} to \bxname{cbc}.}
\item [Tabbed panes]{ have been renamed to \bxname{Tabbed Components}. This is also evident in the abbreviation used for the unbound modules and the component names. It has changed from \bxname{tpn} to \bxname{tbc}.}
\item [Text field / text area / text pane]{ has been renamed to \bxname{Text Component}. This is also evident in the abbreviation used for the unbound modules and the component names. It has changed from \bxname{txf} to \bxname{txc}.}
\end{description}

\textbf{Known issue with using observation mode on Gnome}
\begin{itemize}
\item When using the observation mode on Gnome, \app{} may close unexpectedly. 
\item This is documented in issue \url{http://bugzilla.bredex.de/729}.
\end{itemize}

\textbf{Deprecated modules and \gdsteps{} removed}
\begin{itemize}
\item All unbound modules that were preceded with \bxname{DEPRECATED} have been removed. These modules should be removed from use in your \gdprojects{} before switching to the new version. 
\item All \gdsteps{} that had the status \bxname{deprecated} have also been removed. You should ensure that you no longer use deprecated \gdsteps{} before switching to the new version. 
\end{itemize}

\textbf{Migration of existing remote control Extension}

A lot of changes were done in the remote control part. Firstly we recommend you
to read the extension manual. There are the most information about how the new
implementation works with an simple example.

The following must be done.
\begin{itemize}
\item Change the target for the rc Enviroment as mentioned in \textbf{Extension Manual 2.3.1}
\item Create a fragment and set the appropiate Host Bundle. See \textbf{Extension Manual 2.3.1}
\item copy your current implementation into the fragment and adapt it as described below
\end{itemize}
All ImplClasses have been deleted and rewritten. If you previously used our ImplClasses you
have to change the class you are extending.
\begin{itemize}
\item \bxshell{org.eclipse.jubula.rc.common.tester}\\ in this package are the
  most TesterClasses. Some implementation are in the toolkit specific
  bundles named below.
\item \bxshell{org.eclipse.jubula.rc.swt.tester}\\ SWT specific implementation
 for List, Table, Tree, Application and Menus
\item \bxshell{org.eclipse.jubula.rc.swing.tester}\\ Swing specific
  implementation for JList, JTable, JTree, Application and Menus
\end{itemize}

If you are using graphics components, which are subclasses of supported classes
this should already work(e.g. a subclass of an button). If your
component is a new implementation and only a subclass of Component you must follow the Steps in the \textbf{Extension Manul 2.3.1 Create an Adapter}.

If you have written your own adapter or reusing one you must write an
adapterfactory as described in the \textbf{Extension Manual 2.3.1 Create an Adapter}. This is important to make
the adapter usable in our classes.

The following packages are important for the adapters.
\begin{itemize}
\item \bxshell{org.eclipse.jubula.rc.swt.tester.adapter}\\ SWT adapter.
\item \bxshell{org.eclipse.jubula.rc.swing.tester.adapter}\\ Swing adapter.
\item \bxshell{org.eclipse.jubular.rc.common.tester.adapter.interfaces}\\Interfaces for all adapter.
\end{itemize}

\textbf{Updated migration information}
\begin{itemize}
\item The information on migrating to new versions has been updated.
\item The Installation Manual contains the migration information. 
\end{itemize}
