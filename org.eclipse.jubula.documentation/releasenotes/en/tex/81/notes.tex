\makeatletter
\section{Release Notes for \@bxversion}
\makeatother

\subsection{New Features and Developments}
\textbf{-datadir and -resultdir parameters are now optional for testexec}
\begin{itemize}
\item The parameters to enter a result directory (where HTML and XML reports are created) and to enter the place where any external data files reside are now optional in the testexec.
\item This reduces the amount of parameters you have to enter for a simple configuration.
\end{itemize}

\textbf{Improved no-run option in testexec}
\begin{itemize}
\item The no-run option in the testexec now has a number of parameters so that you can define how far the test should be checked. 
\item You can use this option to perform all steps up to the actual test execution to ensure that e.g. \gdauts{} can be started, object mapping is complete etc.
\end{itemize}

\textbf{Composite components in JavaFX can now be mapped}
\begin{itemize}
\item It is now easier / possible to map composite components such as accordeons and choice boxes in JavaFX \gdauts{}. 
\end{itemize}

\textbf{Support for derived components in JavaFX}
\begin{itemize}
\item You can now test components in JavaFX \gdauts{} that are derived from currently supported JavaFX components. 
\end{itemize}

\subsection{Known issues and other information}
\textbf{Support for iOS 6.1 dropped}
\begin{itemize}
\item We no longer support iOS 6.1 for testing. The iOS support is now for 7.0 applications. 
\end{itemize}
