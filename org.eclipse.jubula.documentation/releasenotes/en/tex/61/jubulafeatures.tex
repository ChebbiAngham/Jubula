%need to add 601 release notes to this as well, make sure nothing double and all in

\textbf{Chronon support}
\begin{itemize}
\item \app{} now allows the use of an embedded Chronon recorder and also the production of Chronon recording files during automated tests (\jb{} users: this feature is currently only available in the standalone version). 
\item Activating the embedded Chronon recorder in \app{} will result in your actions being recorded in a monitoring file which can be used for analysis purposes should you encounter any errors.
\item You can also configure your \gdaut{} to use Chronon as a monitoring agent so that you can generate recording files from your automated tests. 
\item The currently used version of Chronon is 2.0. There are some performance issues still in this version that may lead to test runs and performance being affected, particularly in larger \gdauts{} and \gdprojects{}. Please follow the instructions in the documentation for increasing heap size and ensuring the capability of your test machines. 
\item We do not recommend having Chronon running at all times, due to possible performance degradations through the monitoring. Instead, we recommend that you activate the Chronon recorder when you are trying to reproduce an error.
\end{itemize}

\textbf{Frames and iFrames now supported for Web Testing}
\begin{itemize}
\item \gdauts{} that contain Frames and iFrames can now be tested by \app{}
\end{itemize}

\textbf{Tests updated to newer versions of IE and Firefox}
\begin{itemize}
\item The tested versions for web \gdauts{} now include IE 9 and Firefox 10
\end{itemize}

\textbf{Multiple \gdtestcasebrowser{} instances now supported}
\begin{itemize}
\item You can now open the \gdtestcasebrowser{} multiple times.
\item You can designate one \gdtestcasebrowser{} as the main browser. This browser will be used for actions such as show specification etc.
\end{itemize}

\textbf{Workspace dialog now allows remembering of workspace}
\begin{itemize}
\item You can now select the option to remember your default workspace in the workspace chooser.
\end{itemize}

\textbf{Remember password and auto-login for \gddb{}}
\begin{itemize}
\item In the \gddb{} login dialog, you can select the option to remember your password. 
\item If you have selected this option, you can also specify that the \gddb{} should be used as the default \gddb{} you will be automatically logged in when a \gddb{} connection is required. 
\item The above options are not available for the embedded \gddb{}.
\item To change the \gddb{} you are connected to, or to change your default database login, use:\\
\bxmenu{Test}{Select Database}{}
\end{itemize}

\textbf{Auto-load for default \gdproject{}}
\begin{itemize}
\item In the \bxname{Open Project} dialog, you can now identify the selected \gdproject{} and version number as your default \gdproject{}.
\item You can have one default \gdproject{} per workspace. 
\item When you have a default \gdproject{} set, then selecting:\\
\bxmenu{Test}{Open}{}\\
will automatically load this \gdproject{}.
\item You can remove the default loading in the \bxname{Test} preferences. 
\end{itemize}
