\textbf{Multiple \gdtestcasebrowser{} instances now supported}
\begin{itemize}
\item You can now open the \gdtestcasebrowser{} multiple times.
\item You can designate one \gdtestcasebrowser{} as the main browser. This browser will be used for actions such as show specification etc.
\end{itemize}

\textbf{Workspace dialog now allows remembering of workspace}
\begin{itemize}
\item You can now select the option to remember your default workspace in the workspace chooser.
\end{itemize}

\textbf{Remember password and auto-login for \gddb{}}
\begin{itemize}
\item In the \gddb{} login dialog, you can select the option to remember your password. 
\item If you have selected this option, you can also specify that the \gddb{} should be used as the default \gddb{} you will be automatically logged in when a \gddb{} connection is required. 
\item The above options are not available for the embedded \gddb{}.
\item To change the \gddb{} you are connected to, or to change your default database login, use:\\
\bxmenu{Test}{Select Database}{}
\end{itemize}

\textbf{Auto-load for default \gdproject{}}
\begin{itemize}
\item In the \bxname{Open Project} dialog, you can now identify the selected \gdproject{} and version number as your default \gdproject{}.
\item You can have one default \gdproject{} per workspace. 
\item When you have a default \gdproject{} set, then selecting:\\
\bxmenu{Test}{Open}{}\\
will automatically load this \gdproject{}.
\item You can remove the default loading in the \bxname{Test} preferences. 
\end{itemize}

\textbf{Testexec and DBTool can use -dburl instead of workspace (-data) parameter}
\begin{itemize}
\item To make working with the testexec and dbtool easier, you can now use the dburl parameter to specify which \gddb{} to connect to during the process.
\item This removes the need to specify which workspace to use (via \bxshell{-data}) when working with the testexec and dbtool. 
\item You can still enter the workspace and the -dbscheme information if you prefer not to use the -dburl parameter. 
\end{itemize}


\textbf{Frames and iFrames now supported for Web Testing}
\begin{itemize}
\item \gdauts{} that contain Frames and iFrames can now be tested by \app{}
\end{itemize}

\textbf{Chronon: New preference page and options for \ite{} Chronon}
\begin{itemize}
\item In the preferences dialog, there is now a Chronon category. In this category, you can select the \bxname{ITE Chronon} page. 
\item On this page, you can configure options for the embedded Chronon in the \ite{}. You can configure the output directory for the Chronon recording, as well as any include or exclude patterns that should be included and / or ignored for the recording.  
\end{itemize}

\textbf{Chronon: Support for Chronon server installation for use with \gdauts{}}
\begin{itemize}
\item When you use Chronon with your \gdaut{}, you now have two choices -- you can use the integrated Chronon to record your \gdaut{} or you can specify where your Chronon installation is, thus allowing you to use the Chronon Recording Server. This is the recommended choice for working in continuous integration processes.   
\end{itemize}

\textbf{Chronon: Support for exclude patterns in \gdaut{} configurations}
\begin{itemize}
\item If you are using the integrated Chronon support for testing your \gdaut{}, then you now have the option to specify exclude patterns for the recording as well as include patterns. 
\end{itemize}

\textbf{JRE version updated}

\textbf{VM now runs in server mode}
