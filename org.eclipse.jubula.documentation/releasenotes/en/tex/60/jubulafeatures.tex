\textbf{Test Execution Duration now shown for each node in \gdtestresultview{}}\\
\begin{itemize}
\item The amount of time taken to execute each \gdstep{}, \gdcase{} and \gdsuite{} is now shown in the \gdtestresultview{}.
\item You can use this information to see whether your tests or parts of your tests are taking longer than you expected them to, 
without having to set timers or manually check timestamps.
\item The HTML and XML reports generated or exported for the test results also show the execution times.
\item You can switch off the decoration in the \app{} client in the \bxname{label decoration} preferences.
\end{itemize} 

\textbf{Parameter values now shown for each node in \gdtestresultview{}}\\
\begin{itemize}
\item The parameter values entered or used at a specific level (\gdcase{}, \gdstep{}) are now shown in the \gdtestresultview{}.
\item You can use this information to see which data were used for your \gdcases{}, without having to click through each individual node in the \gdtestresultview{}. This can be especially useful if you have one \gdcase{} that runs multiple times with different datasets.
\item You can switch off the decoration in the \app{} client in the \bxname{label decoration} preferences.
\end{itemize} 

\textbf{Missing data decoration now shown in the \gdtestcaseeditor{} and \gdtestsuiteeditor{}}
\begin{itemize}
\item When working on a \gdcase{} or \gdsuite{} in their respective editors, you will now see small red crosses on any referenced \gdcases{} or \gdsteps{} in that editor which have missing data \textit{at this level} (i.e. data that should be entered in this editor). 
\item This will help avoid problems with forgetting to enter data in the \gdpropview{}. 

\end{itemize}

\textbf{Functions can be entered as parameter values}
\begin{itemize}
\item Alongside concrete values (\bxshell{abc}), references (\bxshell{=P1}) and variables (\bxshell{\$VAR}), you can now enter functions as data, or as parts of your data.
\item The sign to introduce a function is the question mark: \bxshell{?}
\bxwarn{Any question marks used as test data will have to be escaped using one or more backslashes as of this new version. A warning in the console view will appear after importing your \gdproject{} if you have any unescaped question marks in your data. You can then use the search function to find them and mask them.} 
\item There are functions available out of the box for various mathematical calculations and for working with dates.  
\item Functions can be embedded in other functions, e.g. \bxshell{?add(?sub(3,2),4)} to add the result of (3-2) to 4. 
\item Functions can also make use of other data entry options in \app{} such as variables (\bxshell{?add(\${VAR1},={P1})} adds the number in the variable VAR1 to the number in the reference P1). 
\item You can add your own functions via an extension point. 
\end{itemize}

\textbf{Categories now supported in the \gdtestsuitebrowser{} and \gddataeditor}\\
\begin{itemize}
\item It is now possible to create categories for \gdsuites{} and \gdjobs{} in the \gdtestsuitebrowser{}, as well as for central data sets in the \gddataeditor{}. 
\item Existing \gdprojects{} will be shown as having two categories (\bxname{Test Suites} and \bxname{Test Jobs}) in the \gdtestsuitebrowser{} as was previously the case. However, you can now rename, delete and rearrange categories in the \gdtestsuitebrowser{}. 
\end{itemize} 


\textbf{Save As New \gdcase{}}
\begin{itemize}
\item There is a new option in the \gdtestcaseeditor{} and \gdtestsuiteeditor{} to save selected items as a new \gdcase{}.
\item The selected nodes are added to a new \gdcase{} whose name you define.
\item The nodes are not copied, but their references are: the effect is the same as if you had manually created a new \gdcase{} and added the \gdcases{} to it. 
\end{itemize}

\textbf{New action: Store Property Value}\\
\begin{itemize}
\item The \bxname{abstract} toolkit contains a new action on the \bxname{Graphics Component} component to store the value of a property.
\item The action works in a similar way to the \bxname{Check Property} action, except that the expected value must not be entered. Instead, the actual value is saved into a variable you define.
\item You can use this value later on in the test, or you can compare various values using the actions to compare string values or compare numeric values. 
\item One example use case for this new action is to test table filters. The rowCount of the table can be saved into a variable before filtering, and the rowCount after filtering into a different variable. The variables can then be compared with each other to ensure that the second value is less than the first, for example.
\end{itemize} 

\textbf{New extensible adapter mechanism for Swing components}\\
\begin{itemize}
\item \app{} now allows you to add support for renderers for Swing components without the getText() method.
\item An example of the adapter mechanism can be found here:\\
\bxname{http://git.eclipse.org/c/jubula/org.eclipse.jubula.core.git/}\\
\bxname{tree/org.eclipse.jubula.examples.extension.swing.rc.adapter}
\item This does not replace the support for custom Swing renderers that can be changed directly by your developers. 
\item If you are able to change the renderers yourself, you can still implement one of the following in your renderer:
\begin{quote}
public String getTestableText();
public String getText();
\end{quote}
\end{itemize}

\textbf{Completeness check shown as progress window}\\
\begin{itemize}
\item After saving in the \ite{}, the completeness check which validates the state of the \gdproject{} (test data, object mapping, missing \gdcases{} etc) is now shown explicitly as a progress window.
\item This results from performance improvements made to the completeness check.
\end{itemize} 

\textbf{Changes to layout of properties view for test results}\\
\begin{itemize}
\item The properties view for test results now shows all test data as two columns: the parameter name and the value.
\item This reduces the amount of space required to see test data for test results. 
\end{itemize} 


\textbf{GD\_AUT\_STARTUP\_DELAY renamed to TEST\_AUT\_STARTUP\_DELAY}\\
\begin{itemize}
\item There is an undocumented variable that can be set as an environment variable to increase the delay between starting an \gdaut{} and checking that it is there.
\item This delay was named \bxshell{GD\_AUT\_STARTUP\_DELAY} and has now been updated to \bxshell{TEST\_AUT\_STARTUP\_DELAY}. Any customers using this variable should update it in their test environment. 
\end{itemize}

\textbf{Selenium update}
\begin{itemize}
\item The version of Selenium used by \app{} has been updated to 2.12.0.
\end{itemize}
