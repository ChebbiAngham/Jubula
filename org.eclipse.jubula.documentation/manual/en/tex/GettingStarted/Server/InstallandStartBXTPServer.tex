% $Id: InstallandStartBXTPServer.tex 1720 2006-03-02 12:25:33Z dennis $
% Local IspellDict: english
% --------------------------------------------------------
% User documentation
% copyright by BREDEX GmbH 2004
% --------------------------------------------------------
The \gdserver links the \gdaut to the \gdclient so that specified 
tests can be executed. 
It can be run on a server or a local machine, and can even be installed on 
a different operating system from the client
but it must be installed on the 
same machine as the \gdaut{}. The \gdserver must be
 running and a connection to it
must have been established before any object mapping, observation or testing 
can take place.






%Before tests can be specified with \GD  the
%\gdserver has to be started  with the parameter \bxshell{-p} and a
%valid port-number (refer to \bxref{portnumber1}). The user may
%start the \gdserver on a server or the local machine. 

%The \gdserver will notify the \gdclient when the start-up is completed.




%% The \gdserver must be started on the same system as the \gdaut{}.
%%  Depending on the operating system, the \gdserver can either be started via
%%  the Start Menu or a  command line.

%% No matter how the server is started, it must be running and a connection
%% to it must be established before any
%% testing, object mapping or recording can take place. 
%% On machines running a server operating system
%%  (i.e \bxname{UNIX, Windows 2003 Server}),
%% the \gdserver can be started more than once. Each instance must use a 
%% unique port number; 
%% free and unused port numbers are obtainable
%%  from the local system administrator.
%% The default port number is specified during installation.

%% Other port numbers can be specified either by setting an
%% environment variable called \bxname{GDServerPort} or by specifying 
%% a command line argument. The \gdserver is started with the
%% following command:\label{portnumber1}
%% \begin{quote} 
%% \bxshell{gdserver [-p <port-number>]} 
%% \end{quote}

%% UNIX based systems  use the script named \bxshell{gdserver}.
%% On \bxname{Windows} the script is called \bxshell{gdserver.bat}.
%% The script is located in the \bxshell{Server} subdirectory of the installation. 

