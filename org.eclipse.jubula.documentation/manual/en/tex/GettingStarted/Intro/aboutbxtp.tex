% $Id: aboutbxtp.tex 10922 2010-04-21 11:26:09Z alexandra $
% Local Variables:
% ispell-check-comments: nil
% Local IspellDict: american
% End:
% --------------------------------------------------------
% User documentation
% copyright by BREDEX GmbH 2004
% -------------------------------------------------------
This section provides a brief overview of the
advantages of \jb{}{}.


\subsubsection{Early test creation}
\jb{} tests are created before the \gdaut{} is available. This is a radical advantage over capture-replay tools, which force testers to wait until an application is ready to begin with testing. The specification of modular, flexible GUI tests begins early (even as early as at the requirements stage) and continues alongside software development.\\ 
\textbf{Benefit:} every version of an \gdaut{} can be tested as soon as it becomes available. Testing keeps up with development, so you waste no time in your test process. Earlier testing lets you find issues when they are cheaper and easier to fix.

\subsubsection{Lower testing effort}
\jb{} tests are written without programming. This avoids the costs and effort associated with writing and maintaining tests using coding or scripting languages. \\
\textbf{Benefit:} the effort of test automation is reduced because tests do not have to be written and maintained in code or script. Specified tests have been shown to reduce the automated testing effort to 10\% of the whole project cost. 
 
\subsubsection{Flexible tests}
The smallest test element in \jb{} is a \gdstep{}. \gdsteps{} are created from interactive dialogs, based on three pieces of information:
\begin{enumerate}
\item The GUI-component to be tested (e.g. a textfield)
\item The action to execute on the component (e.g. enter text)
\item The parameters (data) for the action (e.g. hello)
\end{enumerate}
At first, only the action needs to be fixed. The component(s) to test, and the data for the action can be defined/changed later. There is a layer of abstraction between the action in a \gdstep{} and the component(s) and data it tests. \\
\textbf{Benefit:} your tests are flexible, which means they can be reused in a wide range of situations  on different components, with different data, in different languages, on different platforms and even for different implementations of a product. 

\subsubsection{Hierarchical test design}
Tests are hierarchically designed, with the structure being entirely up to the tester. \gdcases{} are reusable modules (keywords), and can contain \gdsteps{}, and other \gdcases{}. \jb{} lets you create \gdcases{} as generally as possible, but as specifically as necessary. More generic \gdcases{} are easier to reuse in a test. \\
\textbf{Benefit:} tests are easy and quick to create from reusable modules. Wide reuse of modules reduces the maintenance effort because a few changes update the whole test.
 
\subsubsection{Keyword-driven testing}
\jb{} supports keyword-driven testing. In essence, \gdcases{} can be seen as keywords. An advantage over other keyword-driven tools is that keywords in \jb{} do not have to be programmed, and are executable. They can be created by the testers, and are application and platform-independent. The test specification is the automation. \\
\textbf{Benefit:} build tests from keywords, without needing a middle man in the testing process, responsible for the actual automation. This reduces costs and saves time. 

\subsubsection{Data-driven testing}
Data for the test is kept separate from the actual test. The data can be entered in \jb{}, or can be read from Excel tables. Data can also be translated into any \gdaut{} supported languages. \\
\textbf{Benefit:} one specification can test different data, and different languages, without any changes to the specification. 

\subsubsection{Application structure}
\jb{} is a client-server application. The client is used to specify the tests. The server (\gdserver{}) can be installed on any machine (or platform) in the network where the \gdaut{} can run. \\
\textbf{Benefit:} cross-platform testing using the same test specification.

%\jb{}  is based on the Eclipse RCP. It runs as a standalone application.%or as an Eclipse plugin. \\
%%\textbf{Benefit:} developers can run tests in their IDE using the plugin. 

\subsubsection{Application support}
\jb{} supports the testing of Java (Swing, SWT/RCP) and Web (HTML) applications. Most of the \gdcases{} created with \jb{}  are application-independent. \\
\textbf{Benefit:} automate tests for different \gdaut{} implementations with the same specification.


\subsubsection{Test Project Portability}
Test \gdprojects{} are stored in a multi-user database.\\
\textbf{Benefit:} sharing of reusable modules across teams/projects/testers.
 
Test \gdprojects{} can be imported and exported to and from the database in XML format.\\
\textbf{Benefit:} Test Projects can be versioned with the corresponding code status. 

\subsubsection{Test execution}
Tests can be started by hand or from the command line. \\
 \textbf{Benefit:} automated tests as part of the build process

\subsubsection{Component recognition}
\jb{} treats components as objects. Various details about the components are stored, and are heuristically weighted and evaluated at runtime. \\

\textbf{Benefit:} reliable component recognition, even after changes in the \gdaut{}. Actions executable on individual table cells/tree nodes. 

\subsubsection{Error handling}
\jb{} handles several different types of error and can respond with user-defined \gdsteps{} as well as various ways to continue/stop the test execution. \\
\textbf{Benefit:} define how tests should react to errors. 

\subsubsection{Checkpoints}
Check existence, enablement, focus, selection, text, and properties of various components using \gdsteps{}.\\ 
\textbf{Benefit:} tests which can check different object properties. 

\subsubsection{Synchronization}
\jb{} offers various actions to synchronize the test automatically, including waiting for components to appear and waiting for windows to open or close.\\
\textbf{Benefit:} your tests will run even at different application speeds.

\subsubsection{Result analysis}
Test results can be produced as XML or HTML reports. The reports contain information on the status of the test, the start/end time and duration, how many steps were executed, and in which language. When errors have occurred in the test, they are marked in red, with information on the expected value and the actual value. \\
\textbf{Benefit:} analyse and compare test results over a period of time.


\subsubsection{Extensibility}
\jb{} allows user extensions to the supported components and actions. Write extensions to existing components and actions, or add entirely new components and actions for applications which use modified or unique components. \\
\textbf{Benefit:} test even applications with an altered look-and-feel or with unique implementations. 

 



