% $Id: layout.tex 8161 2009-04-06 14:07:39Z alexandra $
% Local Variables:
% ispell-check-comments: nil
% Local IspellDict: american
% End:
% --------------------------------------------------------
% User documentation
% copyright by BREDEX GmbH 2004
% --------------------------------------------------------
This manual is divided into the following chapters:

\begin{enumerate}
\item The \bxref[chapter ''Getting Started'']{Gettingstarted} provides a background and introduction to \jb{}{}.

\item The  \bxref[chapter ''Samples'']{Samples} exemplifies the capabilities of \jb{}{}, using 
\jb{}{} \gdproject{} examples.
We recommend this chapter if you learn best by example. 

\item The \bxref[chapter ''Tasks'']{Tasks} provides step-by-step instructions for performing most of the common tasks in \jb{}{}.
This chapter is best suited for those who learn by doing.

\item The \bxref[chapter ''Best Practices'']{BestPractices} details how to work with \jb{} for successful and easy-to-read test structures and automated test processes.

\item The \bxref[chapter ''User Interface'']{userinterface} presents the \jb{} user interface, describing its flexible design.
This chapter is useful if you are not yet familiar  with the GUI concept of the \bxname{Eclipse} platform.

\item The \bxref[chapter ''Concepts'']{Concepts} introduces the concepts behind
  \jb{}{}. It will familiarize you with \jb{}'s structure, components, and functionality. This chapter is useful (and recommended) for gaining in-depth knowledge of how \jb{}{} works.


\item Finally, the \bxref[chapter ''Glossary'']{Glossary} provides explanations of all
technical terms and abbreviations used throughout this manual.
\end{enumerate}
