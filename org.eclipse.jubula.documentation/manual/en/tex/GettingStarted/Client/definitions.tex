% $Id: definitions.tex 1720 2006-03-02 12:25:33Z dennis $
% Local Variables:
% ispell-check-comments: nil
% Local IspellDict: american
% End:
% --------------------------------------------------------
% User documentation
% copyright by BREDEX GmbH 2004
% --------------------------------------------------------
\begin{description}
\item[\gdstep] 
This is the smallest entity we use to specify a test.
It consists of three elements:
\begin{itemize}
\item a user interface \textbf{c}omponent
\item an \textbf{a}ction  to be performed with that component and 
\item \textbf{p}arameters for that action.
\end{itemize}
Because of these three elements we also use the term \bxname{CAP}.

\item[\gdcase]
A \gdcase may contain 
any number and combination of \gdsteps and
other \gdcases and will be displayed as a tree structure. This concept can be compared
with a file (\gdstep) and directory(\gdcase) structure. 
It is the main way of organizing and structuring the
tests. 

\item[\gdsuite] 
A \gdsuite may contain any number of \gdcases to be executed.
It must be linked or bound to an AUT. The execution is done by traversing
the tree from top to bottom.

\item[\bxname{Project}] 
Several \gdsuites can be stored and grouped together
in a project. 

\end{description}