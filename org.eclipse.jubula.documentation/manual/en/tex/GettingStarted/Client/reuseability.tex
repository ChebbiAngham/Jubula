% $Id: reuseability.tex 1720 2006-03-02 12:25:33Z dennis $
% Local Variables:
% ispell-check-comments: nil
% Local IspellDict: american
% End:
% --------------------------------------------------------
% User documentation
% copyright by BREDEX GmbH 2004
% --------------------------------------------------------
Just imagine two different ways of developing software-tests. On the one hand you
can  develop tests with concrete and fixed values. It is like generating a list
of values which are generally just useful for one \gdcase
(\bxname{TC}.  
On the other hand you can
develop test with references and use them like placeholders.
References can be fitted with individual values --  and 
references can also be fitted automatically. 
The tester just needs to fit \jb{} with data e.g. using data
tables. \jb{} will automatically run a \gdcase
until all test data have been tested.  
This approach is much favorable. Just compare the two
modus operandi: 

To following example (\tableref{comparison}) will distinguish the  advantages of
\jb{} over common  testing tools:
\begin{table}
\begin{center}
\begin{tabular}{|p{0.4\bxpicwidth}|p{0.5\bxpicwidthwidth}|}\hline 
                                 & \\
\underline{Common Test Tool}                & \underline{\jb{}} \\ 
\tt TF1 input 1                 & \bf TC1 \\
\tt TF1 input 2                 & \tt \hspace{1em}TCadder(P1, P2, P3) \\
\tt  button click               & \tt  \hspace{3em} TF1 input P1 \\ 
\tt TF3 verify 3                  & \tt  \hspace{3em} TF2 input P2 \\ 
\tt TF1 input 2                 & \tt  \hspace{3em} button click \\
\tt TF1 input 4                 & \tt   \hspace{3em} TF3 verify P3\\ 
\tt  button click               & \\  
\tt TF3 verify  6                & \underline{Reusing \jb{} \gdcases Examples:}\\ 
\tt TF1 input -2                & \tt TCadder(1, 2, 3)\\
\tt TF2 input 17                & \tt TCadder(Data Set1, Data Set2,
Data Set 3)\\
.....                           & \\
                                & \\ \hline
\end{tabular}
\caption{{\small Comparison between usual SW-Test and \jb{}}}
\label{comparison}
\end{center}
\end{table}

The tester just needs to specify 
 one \gdcase including
references like this and is able to reuse it
every time and every where it seems to be  necessary/useful. 

This approach is much more convenient because its easy and fast to
to create, flexible and therefore reusable at any time. 



  

