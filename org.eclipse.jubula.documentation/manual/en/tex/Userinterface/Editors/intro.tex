% $Id: intro.tex 8138 2009-04-02 15:43:28Z alexandra $
% Local Variables:
% ispell-check-comments: nil
% Local IspellDict: american
% End:
% --------------------------------------------------------
% User documentation
% copyright by BREDEX GmbH 2004
% --------------------------------------------------------
\index{Perspective}
\index{Browser}
\index{View}
\index{Editor}

There are four types of area in the \jb{} user interface:
\bigskip

\textbf{Perspectives}

There are three perspectives, specification, execution and modeling, each with a different function. A perspective is a collection of views, editors and browsers on the screen. 

\textbf{Browsers}

Browsers let you see the \gdcases{} and \gdsuites{} you have created in their hierarchical structures. There is also a browser for component names. 

\textbf{Editors}

Editors let you modify, add and delete items in \gdcases{} and \gdsuites{}. You can open editors by double-clicking on a node (e.g. \gdsuite{} or \gdcase{}) in a browser. In the modeling perspective, the editor is the canvas for your models. 


\textbf{Views}

Views show details about the currently selected item. If you single-click an item in a browser, the details in the views are read-only. If you select an item from within an editor, you can edit some of the values in the views.  

\bigskip

You can change the way your user interface looks by moving areas, changing the colors, and hiding/showing views \bxpref{adapting views}. 

To reset the default layout at any time, select:\\
\bxmenu{Window}{Reset Perspective}{}.

\subsection{Color coding}
Because \jb{} contains user interface elements whose contents change depending on what is selected, we use a color coding system to distinguish between areas used for specification and execution. 

The areas used for specification have brown borders by default. These are:
\begin{itemize}
\item The \gdtestcasebrowser{}
\item The \gdtestcaseeditor{}
\item The \gdpropview{}, \gdcompnamesview{} and \gddatasetsview{} when they are showing details about an item selected in one of the two areas above. 
\end{itemize}

The areas used for execution have blue borders by default. These are:
\begin{itemize}
\item The \gdtestsuitebrowser{}
\item The \gdtestsuiteeditor{}
\item The \gdtestresultview{}
\item The \gdpropview{}, \gdcompnamesview{} and \gddatasetsview{} when they are showing details about an item selected in one of the three areas above.
\end{itemize}

You can change the colors or switch the color coding off in the \bxcaption{appearance} preferences. 

 

