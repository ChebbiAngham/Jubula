% $Id: Extracting.tex 7785 2009-02-03 16:16:16Z alexandra $
% Local Variables:
% ispell-check-comments: nil
% Local IspellDict: american
% End:
% --------------------------------------------------------
% User documentation
% copyright by BREDEX GmbH 2005
% --------------------------------------------------------
% this command can be inserted multiple times
%\gdhelpid{}
% 
%\begin{gddescription}
%\end{gddescription}
%
%\begin{gdlist}
% use the \item command for single steps
%\end{gdlist}
% change <PATH> to the same directory, file is located in
% change <FILE> to the same filename you are editing
%\bxinput{<PATH>/Links/<FILE>}
%
% other usefull commands are
%   \bxtipp{}        to create a hint
%   \bxwarn{}        to describe a warning

\index{Test Case!Extracting}
\index{Extracting!Test Case}
\jb{} lets you \bxname{extract} \gdcases{} from other \gdcases{}. This lets you create keywords even after you have started specifying. 

\begin{enumerate}
\item Open the \gdtestcaseeditor{} by double-clicking on the \gdcase{} you want to edit in the \gdtestcasebrowser{}. 
\item Select the \gdcases{} you want to extract by single-clicking them. Use 
  \bxkey{Ctrl} to select more than one item. 
\item Right-click in the editor and  select: \\
\bxmenu{Refactor}{Extract Test Case}{}.
\item When prompted, enter a name for the new \gdcase{}. 
\item The \gdcases{} you selected will be extracted into this new \gdcase{}. 
\item The extracted \gdcase{} appears as a reused \gdcase{} in the current editor. It is marked with a small arrow to show that it is reused, and the \gdcase{} name is in angled brackets (\bxshell{< >}) to show that it is the same as the specification name. 
\item The \gdcase{} you just created is also visible in the \gdtestcasebrowser{}. 
\end{enumerate}
\bxtipp{Use this feature when you realize that you are planning on reusing one or more \gdcases{} for the same or a similar action again. You will save yourself time in test creation and maintenance.}

