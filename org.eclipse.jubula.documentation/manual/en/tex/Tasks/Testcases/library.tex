\index{Test Case!Library}
\index{Library of Test Cases}

Everything you need to create tests is already available for you in each \gdproject{}. The library of \gdcases{} which appears automatically in each new \gdproject{} contains all the actions supported by \gd{} as well as some examples of more complex keywords. For general information about the library, read the section later on \bxpref{LibraryInformation}. To learn how to use the library to create tests, read the next section \bxpref{UseLibrary}. 


\subsubsection{Using the library to create tests}
\label{UseLibrary}
Creating tests with \gd{} is really just a matter of:
\begin{itemize}
\item Deciding how to structure your tests (i.e. what keywords to create and how to combine them \bxpref{KeywordDesign}).
\item Choosing the \gdcases{} from the library (or from your own set of \gdcases{}) that you need to create these keywords. 
\end{itemize}

To use the \gdcases{} from the library, you will first need to create a \gdcase{} of your own \bxpref{TasksCreateTC}. 

\begin{enumerate}
\item Open the \gdtestcaseeditor{} by double-clicking on the \gdcase{} you want to edit in the \gdtestcasebrowser{}. 
\item In the \gdtestcasebrowser{}, browse to a \gdcase{} that you want to add. For help on finding your way around the library, see the later section \bxpref{UnderstandingLibrary}. 
\item You can add the \gdcase{} by dragging and dropping from the \gdtestcasebrowser{} to the \gdtestcaseeditor{} or you can right click on the \gdcase{} in the \gdtestcaseeditor{} and select:\\
\bxmenu{Reference Existing \gdcase{}}{}\\
to see a list of all \gdcases{} that you can add to this \gdcase{}. 
\item Once you have added the \gdcase{}, you will need to enter a component name in the \gdcompnamesview{} \bxpref{TasksReassignCompName} and you will need to enter data for the \gdcase{} in the \gdpropview{} \bxpref{WorkingWithData}. 
\bxtipp{See the Best Practices section for information on how to best structure your \gdcases{} and tests \bxpref{BPKeywordDesign}.}
\end{enumerate}


\subsubsection{Information about the library}
\label{LibraryInformation}
\begin{enumerate}
\item The \gd{} team uses a highly reusable library of \gdcases{} to specify tests. These \gdcases{} are installed with the software to reduce the effort spent creating your tests.  
\item The \gdprojects{} containing the \gdcase{} libraries are located under:\\
\bxname{examples/testcaseLibrary}.

\item The \gdprojects{} available are:
\begin{itemize}
\item unbound\_modules\_concrete
\item unbound\_modules\_web
\item unbound\_modules\_swt
\end{itemize}

\item These \gdprojects{} contain reusable \gdcases{} which have been created in advance so you do not have to specify them yourself.
\bxtipp{Refer to the chapter on Components, Actions, and Parameters
(\bxextref{\gdrefman}{ref,actparam}) for information on components, the
actions they support, and their parameters.}
\item The library is split into categories of \bxname{actions} on components. To select something in your \gdaut{}, open the \bxname{select} category and then open the category for the type of component you want to select something from. 

\bxtipp{The names for these \gdcases{} all begin with \bxcaption{ub}. This means that they are \bxname{unbound} -- they are not in any way dependent on an \gdaut{}. }
\item Each \gdcase{} in the library contains one \gdstep{} which corresponds to the action in the \gdcase{} name. The component name is a placeholder, and the parameters have been referenced so that you can enter your own data. 

\bxtipp{The unbound modules \gdprojects{} which correspond to your chosen \gdproject{} toolkit are automatically imported into the \gddb{} and reused in your \gdproject{}.}
 

\bxwarn{We do not recommend making changes to the installed unbound module \gdprojects{}, for compatibility reasons. If you have \gdcases{} you want to reuse in other \gdprojects{}, we advise creating your own library \gdproject{}.}


\end{enumerate}

\subsubsection{Tips and tricks for using the \gdcase{} library}
\label{UnderstandingLibrary}
The \gdcases{} in the library are organised into actions. In the \bxname{basic} category, you will find the various actions offered by \gd{}. The \bxname{complex} category contains some example keywords which are built up of more than one \gdcase{}. 

When specifying your tests, you need to find and choose which actions you will need. This takes some practice, but there are some hints which can help you:

\textbf{High-Level Actions}\\
\gd{} executes \bxname{high-level} actions. This means that if you want to select something from a menu using the menupath, for example, you need to look in the category:\\
\bxname{Actions (basic)/Select/Menu Bar}\\
and select the \gdcase{}:\\
\bxname{ub\_mbr\_selectEntry\_byTextpath}\\
The \gdcase{} finds the menu, opens it, and clicks the item you specify. \\

\textbf{Abstract components}\\
There are some actions which are executable on many different components. Clicks, for example can be executed on pratically all components in the interface. You can also check text on labels, combo boxes and text fields. Obviously, it makes sense to specify your \gdcases{} as abstractly as possible so that they can be reused in more places. This helps keep the maintenance low later. 

You will notice in the library that there is no \bxname{Button} category under the \bxname{Click} category. Instead, you will find various click actions specified for \bxname{Graphics Component}. This is because all components which support clicks belong to the \bxname{Graphics Component} group. 

In the same way, you will find the category \bxname{Check/Component with Text}. You can use the check actions from this category to check text on any component with text -- buttons, text fields, combo boxes, labels. \\

\textbf{Parameters}\\
Different actions require different data. Some \gdcases{} in the library have been pre-configured with data to make test specification easier (look in the category \bxname{Input via Keyboard/Application/Key Combination} for a long list). 

Some actions let you choose whether you want to enter data using an \bxname{indexpath} or a \bxname{textpath}. We recommend using the textpath so that you are not dependent on the order of e.g. menu entries or tabbed panes. 

To make text-based parameters more robust, \gd{} often lets you choose an \bxname{operator}. You can choose between \bxname{equals, matches, simple match, not equals}. Using matches lets you use regular expressions so that you don't have to hard-code the whole text parameter. 




