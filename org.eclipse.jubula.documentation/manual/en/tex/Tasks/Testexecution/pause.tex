\index{Pause on Error}
\index{Continue without Event Handler}

If you are executing tests interactively, you can use the \ite{} to help you analyze any errors that occur.

\subsubsection{Pause on Error}
On the toolbar in the \ite{}, click the \bxcaption{Pause on Test Execution Errors} button
%TODO
%\gdmarpar{../../../share/PS/}{ pause test execution on error}
or press \bxkey{F9} to cause the test to pause when it encounters an error. 

\bxtipp{Failed \gdsteps{} that have a retry \gdehandler{} will not be paused unless the final retry fails.}

\subsubsection{Continuing after an error}
Once you have analyzed the error in the \gdaut{} or in the test specification, you have two options:

\begin{description}
\item [Continue with the \gdehandler{}:]{Press the \bxcaption{Pause test execution} button on the toolbar \bxpref{PauseExec} to continue with the test. This will result in a \gdehandler{} (either a default \gdehandler{} or one of your own) being activated.}
\item [Continue without \gdehandler{}:]{Press the \bxcaption{Continue without \gdehandler{}} button or \bxkey{F8} to continue with the test \bxname{as if the failed \gdstep{} had been successful}. No \gdehandler{} is activated. This option is useful if the \gdehandler{} that would be activated would restart the \gdaut{} or exit the test, but you are either able to ignore the error for the moment or make changes in the \gdaut{} to ensure that the test will be able to continue. }
\end{description}
\bxwarn{Any changes you make to the test specification while the test is paused will not become valid until the next test execution start.}
