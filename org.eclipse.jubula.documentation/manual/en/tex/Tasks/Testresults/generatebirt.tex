\begin{itemize}
\item From the \gdtestsummaryview{}, you can generate reports of your tests using the BIRT reporting engine. 
\item \app{} offers a selection of example reports:
\begin{description}
\item[GUIdancerShort:]{This report shows a pie chart of the selected tests as an overview: how many were successful and how may failed. }
\item [GUIdancerHistory:]{This report shows a graph of the percentage completion for the selected \gdsuites{} for the dates given. There is also a list of the \gdauts{}, \gdsuites{} and test runs.}
\item [GUIdancerHistoryAbsolute:]{This report shows the same details as the GUIdancerHistory report, but instead of showing the test results in percentages, shows the actual amount of \gdsteps{} executed. It also shows the difference between expected and executed \gdsteps{}.}
\item [Testresult:]{This report gives out the full test details up to the given nesting level.}
\item [TestresultError:]{This report shows where errors in the selected tests occurred and displays any screenshots taken automatically alongside the results \bxpref{testresprefs}.}
\item [GUIdancerFULL:]{The full report is a combination of the short reports and history reports. } 
\end{description}

\item To start a report, click the arrow next to the \bxcaption{Create Report} button and select the report you want to generate.
\gdmarpar{../../../share/PS/createBirtReport}{create BIRT report}

\bxtipp{If you are not already connected to the \gddb{}, then a dialog will appear to create the connection \bxpref{tasksdblogin}. }

\item The BIRT report viewer starts (the first time it starts it may take some time).
\item The parameters for the report are displayed:
\begin{description}
\item [\gddb{} connection:]{The current \gddb{} connection, user and password are automatically placed into the field for the connection information.}
\item [Selection:]{In this section, specify which time frame the report should be generated for (either from a specific date, or using the options \bxname{yesterday, now, last week} etc.) as well as for which \gdproject{}, \gdsuite{}, and the operating system.  The details in the selection section are combined using \bxname{and}. SQL syntax can be used (e.g. \verb+%+ is used as a wildcard for any number of any characters, \verb+_+ is used as a wildcard for one character). }
\item [Detail Selection:]{For the Testresult and TestresultError reports, you must also specify the nesting level (how many levels in the hierarchy should be shown) and the ID of the test run you want to generate the report for. The test run ID for each run can be seen in the \gdtestsummaryview{}.}
\end{description}

\item Click \bxcaption{OK} to start the report generation. 
\item Once the report is ready, it will be shown in the BIRT viewer. 
\item You can click through the report's pages in the viewer and also export the report as a PDF. 
\bxwarn{The option \bxname{Auto} when exporting a report leads to the right-hand side of the report being cut off.}

\bxtipp{You can also create your own reports to execute with \app{} \bxpref{OwnBIRT}.}
\end{itemize}
