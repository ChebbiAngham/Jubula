\index{Reentry Properties}
\index{Event Handler!Reentry Properties}
\index{Break}
\index{Continue}
\index{Return}
\index{Pause}
\index{Exit}
\index{Retry}

\label{reentrytype}
The reentry type specifies how the test should continue once the \gdehandler{} \gdcase{} has been executed. 

\begin{description}
\item[Break]{The test execution leaves the \gdcase{} in which the error occurred and continues at the next \gdcase{} or \gdstep{}. } 
\item[Continue]{The test execution carries on at the next \gdcase{} or \gdstep{}. This is a good option when the error is relatively unimportant and does not affect the following \gdsteps{}. } 
\item[Exit]{The test execution is stopped. Use this when the error is so severe that the test cannot be continued.} 
\item[Return]{The test execution leaves the \gdcase{} in which the activated \gdehandler{} is nested. This could be the current \gdcase{} or one higher up in the hierarchy. This option is useful if you have a use case, which contains \gdcases{} to test a particular area or function. You can decide to leave this part of the test, and carry on at the next use case. \bxname{Return} will continue at the next \gdsuite{} in a \gdjob{} if it is activated as a default \gdehandler{} or if there are no further steps in the current \gdsuite{}.} 
\item[Pause]{The test execution is paused. To restart the test, click the \bxname{pause} button in the toolbar in the \jbclient{}.} 
\item[Retry]{The failed \gdstep{} is repeated as many times as you specify in the \gdpropview{}. If the \gdstep{} is successful on repeating, it is marked as successful after retrying. If the \gdstep{} fails on all retries, the error type is passed on to the next parent \gdcase{} and this \gdstep{} (and therefore the test) is marked as failed. }
\end{description}
