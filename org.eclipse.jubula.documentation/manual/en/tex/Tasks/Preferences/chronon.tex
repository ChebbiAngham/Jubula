\index{Preferences!Chronon}
\index{Chronon!Preferences}


You can find the Chronon preference page under:\\
\bxname{Chronon/ITE Chronon}\\
in the preferences dialog.

On this page, you can configure the following:
\begin{description}
\item [Output directory:]{Use this to configure where the recording files for Chronon are written.}
\item [Include patterns:]{Enter a comma-separated list of packages that you want to be covered by the monitoring. The packages must adhere to the patterns as defined in the Chronon documentation, for example \bxshell{com.myorg.**} selects the whole com.myorg namespace. \bxshell{com.myorg.*} selects only classes in the com.myorg package. The documentation for the patterns is located at \\
\href{https://chronon.onconfluence.com/display/DOC/Include+and+Exclude+patterns}{https://chronon.onconfluence.com/display/\\
DOC/Include+and+Exclude+patterns}\\ 
The default include patterns are to ensure that code from \app{} is recorded. You should leave the patterns as they are unless otherwise specified e.g. by the support or development team.}
\item [Exclude patterns:]{Enter a comma-separated list of packages that you want to be excluded by the monitoring. The packages must adhere to the patterns as defined in the Chronon documentation, for example \bxshell{com.myorg.**} selects the whole com.myorg namespace. \bxshell{com.myorg.*} selects only classes in the com.myorg package. The documentation for the patterns is located at \\
\href{https://chronon.onconfluence.com/display/DOC/Include+and+Exclude+patterns}{https://chronon.onconfluence.com/display/\\
DOC/Include+and+Exclude+patterns}\\
 You should leave the patterns as they are unless otherwise specified e.g. by the support or development team.}
\end{description}
\bxtipp{You must restart the Chronon recording (via stop and then start) before the changes will take place. }
