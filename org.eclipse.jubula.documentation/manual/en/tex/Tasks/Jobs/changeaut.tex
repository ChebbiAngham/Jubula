You can use \app{} to test multiple \gdauts{} in one test run.

The \gdauts{} can be the same actual \gdaut{} which has been started multiple times (to test refresh aspects, for example). 

You can test \gdauts{} that were started independently, or \gdauts{} that are launched by other \gdauts{}.


\subsubsection{Testing independently started \gdauts{}}
\index{Test Job}

To be able to test multiple \gdauts{} that are \bxname{not} started by each other, the following criteria must be met:

\begin{itemize}
\item The \gdauts{} are either written with the same toolkit (e.g. Swing) or,
\item you have specified your \gdproject{} at the \bxname{concrete} level, and will only be testing areas of the \gdauts{} that can be tested with the actions that are valid for all \gdaut{} types (i.e. no RCP-specific components are involved in the test) \bxpref{projtoolkit}. 
\item The \gdauts{} are all defined in the same \gdproject{}.
\item The first \gdaut{} can either be started using the \bxname{autrun} command \bxpref{autrun} or via an \gdaut{} configuration \bxpref{configuringaut}. Any other \gdauts{} required for the \gdjob{} must have been started with the \bxname{autrun} command.
\end{itemize}

\bxtipp{To run \gdjobs{} from the test executor, all \gdauts{} for the test run must be started with the \bxname{autrun} command.}

\subsubsection{Testing \gdauts{} that are launched by other \gdauts{}}
\label{TasksLenientTest}
\index{Test Job}

If your \gdaut{} starts other \gdauts{} which you also want to test, then the following criteria must be met:

\begin{itemize}
\item The \gdagent{} must be running in \bxname{lenient} or \bxname{non-strict} mode \bxpref{TasksAgentLenient}.
\item The \gdauts{} must be written with the same toolkit (e.g. Swing) \bxpref{projtoolkit}.
\item The \gdauts{} have all been defined for this \gdproject \bxpref{Defineaut}.
\item The order in which the launched \gdauts{} will appear and be tested must be known.
\bxtipp{When an \gdaut{} is launched by another \gdaut{}, the \gdaut{} ID for the new \gdaut{} is formed as \bxshell{AUT ID + 1}. The next \gdaut{} to be started receives the ID \bxshell{AUT ID + 2}, and so on. You can enter these \gdaut{} IDs in the \gdsuites{} in the \gdjob{}. }
\end{itemize}

\textbf{Behavior of \gdauts{} when being started by other \gdauts{}}\\
\begin{description}
\item {RCP starting RCP:}[The newly started RCP \gdaut{} receives the ID \bxshell{ID+1}.]
\item {Swing Jar or main class starting Swing Jar or main class:}[This is currently not possible.]
\item {Swing Jar or main class starting Swing executable:}[This is currently not possible.]
\item {Swing executable starting Swing Jar or main class:}[The newly started Swing \gdaut{} receives the ID \bxshell{ID+1}.]
\item {Swing executable starting Swing executable:}[The newly started Swing \gdaut{} receives the ID \bxshell{ID+1}.]
\end{description}

\bxwarn{If the \gdagent{} is not running in lenient mode, then the newly started \gdaut{} will shut down.}
