You can start the \dash{} server with default parameters or with custom parameters. 

\begin{enumerate}
\item In the installation directory, open the \bxname{dashboard} folder. 
\item In this folder, start the \bxname{dashboardserver}. 
\item If you start the server without any arguments, it will use the properties file \bxname{dashboardserver.properties} from its current working directory. You can see this properties file in the same folder as the \bxname{dashboardserver} on Windows and Linux, and in the folder:\\
\bxname{dashboard/dashboardserver.app/Contents/MacOS}\\
on Mac systems. 
\item You can, alternatively, specify a different properties file using \bxshell{-c <PROPERTIES\_FILE>}.
\item The properties file must contain the following properties with the values you require for them (do not use line breaks within the properties in the properties file):
\begin{verbatim}
org.eclipse.jubula.dashboard.
port=<PORTNUMBER>

org.eclipse.jubula.dashboard.
jdbc_driver=<JDBC_DRIVER>

org.eclipse.jubula.dashboard.
jdbc_url=<JDBC_URL>

org.eclipse.jubula.dashboard.
database_username=<DB_USERNAME>

org.eclipse.jubula.dashboard.
database_password=<DB_PASSWORD>
\end{verbatim}
\item Once the \dash{} server has started, users will be able to connect to it via the web application \bxpref{TasksDashConnect}.
\end{enumerate}
