Once you have started the \dash{} application in your browser \bxpref{TasksDashCustom}, you will see the \reportpersp{} you are familiar with from the \ite{} \bxpref{TestResSumView}.

\subsubsection{Features available in the \dash{}}
Many of the features available for viewing test results in the \ite{} are also available here:
\begin{itemize}
\item You can reopen the \gdtestresultview{}  for a test run whose details are still in the \gddb{} \bxpref{TasksReopenTestResult}.
\item You can refresh the \gdtestsummaryview{} \bxpref{TasksRefreshSummaryView}.
\item You can filter and sort in the \gdtestsummaryview{} \bxpref{TasksFilterTestSummary}.
\item When a test result is open, you can jump to the next or previous error \bxpref{testresultview} and view properties details / any screenshots taken for a selected node. 
\item You can export test results as HTML and XML reports from the \dash{}. 
\end{itemize}

\subsubsection{Features unavailable in the \dash}
\begin{description}
\item [\gddb{} changes:]{You cannot perform any actions in the \dash{} that alter the data in the \gddb{}, e.g. deleting test result summaries, entering comments or toggling relevance.}
\item[Preferences]{You also cannot change the preferences via the workspace, so the default period of time for which you see test result summaries is 30 days, and the label decoration for execution duration time and parameter values are displayed \bxpref{TasksPrefsDecoration}.}
\item [Code coverage]{Code coverage reports cannot be opened in the \dash{}. You can, however, see the code coverage values in the \gdtestsummaryview{}.}
\item [BIRT reports]{ are not available in the \dash{}.}
\end{description}
