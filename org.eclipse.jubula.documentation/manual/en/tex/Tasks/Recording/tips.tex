\index{Observation Mode!Tips}

We have designed the \jb{} observation mode to help you get started with your tests and to help you understand what sort of user actions correspond to \jb{} actions. 

\jb{} is first and foremost a keyword-driven tool, and the library of \gdcases{} installed with \jb{} means that you have various benefits over simple recording:

\begin{itemize}
\item You can create tests without needing the \gdaut{} - you don't have to wait for each version of the \gdaut{} to begin test specification. 
\item Your tests aren't so dependent on the actual implementation of the \gdaut{}, so they can be a lot more general in terms of data, component implementations etc. and therefore a lot more robust and maintainable. 
\item Using keywords encourages you to think about your test structure a lot more, which also helps maintenance later. 

\end{itemize}


Nevertheless, we understand that you may want to observe some \gdcases{}. Here are some tips that might help you:
\begin{itemize}
\item Think about your test structure at the beginning: what modules will you want to reuse? Start by observing small \gdcases{}, e.g. a \gdcase{} to login, a \gdcase{} to open a dialog etc. 
\item Check while you are observing that the actions you carry out are observed in the way you meant. 
\item Supplement your observed \gdcases{} with other \gdcases{} from the library of \gdcases{} in \jb{}. 
\item Refactor as you go! If you have recorded a \gdcase{} with specific data, but you will want to use it for any type of data, then add a reference for the data. 
\end{itemize}
