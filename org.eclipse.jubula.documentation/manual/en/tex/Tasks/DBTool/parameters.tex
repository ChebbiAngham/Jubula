\index{dbtool}
\begin{enumerate}
\item Once you have browsed to the \app{} installation directory and entered \bxshell{dbtool}, you can enter the parameters for the \gddb{} actions.
\item The dbtool uses the parameters described in the table \bxfigref{dbtoolparams}:


\begin{table}[h]
\label{dbtoolparams}
	\centering
	\begin{tabular}{|l|l|}

	\hline
	\textbf{Detail}&\textbf{Parameter}%&\textbf{Example}
\\
		\hline
                Help 
                &\bxshell{-h}\\
                & Gives parameter help\\
                \hline
                  Delete Project
                  & \bxshell{-delete <project-name project-version>}\\
		  &e.g. \emph{-delete ''ExampleProject'' 1.0}\\
                  \hline
                  Delete All
                  & \bxshell{-deleteall}\\
		  &e.g. \emph{-deleteall}\\
                  \hline
                  Keep test result summaries
                  & \bxshell{-keepsummary} (optional)\\
		  &e.g. \emph{-keepsummary}\\
                 \hline
                  Directory 
                  & \bxshell{-directory <directory path>}\\
		  &e.g. \emph{-directory ''D:/Test/Projects/''}\\
                  & The directory for imports and/or exports\\
                 \hline
                  Export Project
                  & \bxshell{-export <project-name project-version>}\\
		  &e.g. \emph{-export ''ExampleProject'' ''1.0''}\\
                 \hline
                  Export All
                  & \bxshell{-exportall}\\
		  &e.g. \emph{-exportall}\\
                  &The directory for the export all must be empty\\
                 \hline
                  Import Project
                  & \bxshell{-import <import-file>}\\
		  &e.g. \emph{-import <ExampleProject.xml>}\\
                  &Existing files with the same name will be overwritten\\
		\hline
                Workspace
                  & \bxshell{-data <path to workspace>}\\
                   &e.g. \emph{-data ''C:/Users/Test''}\\
                \hline
                Database scheme
                & \bxshell{-dbscheme <scheme>}\\
		&e.g. \emph{-dbscheme ''Oracle''}\\
                \hline
		Database username
                  & \bxshell{-dbuser <username>}\\
		&e.g. \emph{-dbuser ''myusername''}\\
		\hline
		Database password
                  & \bxshell{-dbpw <password>}\\
		&e.g. \emph{-dbpw ''mypassword''}\\
		\hline
		Database URL (optional)
                  & \bxshell{-dburl <URL>}\\
                  &e.g. \emph{-dburl ''db.example.de''}\\
		&If no URL is given, the default will be used.\\
		\hline
	\end{tabular}
	\caption{Parameters for the dbtool}
\end{table}

\item You can use the parameter \bxshell{-keepsummary} to specify that the test result summaries should not be deleted when the \gdproject{} or \gdprojects{} are deleted. This is useful for continuous integration processes, where the test results over time should be kept, but the \gdprojects{} are reimported into the \gddb{} (for example from the version control system) each night. If you do not enter this parameter, the summaries will be deleted with the \gdprojects{}.

\bxtipp{If you are using the embedded \gddb{}, see the section on using the embedded \gddb{} with the Test Executor for information on which username and password to use \bxpref{TasksTestExecEmbedded}}.
\item Once you have entered all the necessary parameters, press \bxkey{enter}. 

\end{enumerate}
  
