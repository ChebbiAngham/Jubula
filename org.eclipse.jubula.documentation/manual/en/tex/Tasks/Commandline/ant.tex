\app{} tests can be performed as part of an \bxname{Apache Ant} build
process. If you are already familiar with ant tasks, integrating \app{}
tests into your production process is straightforward. The following is a snippet of an example \bxshell{build.xml}
file:

\footnotesize
\begin{verbatim}
<project name="GDTestProject" basedir="[install directory]">
    <taskdef name="guidancer"
      classname="com.bredexsw.guidancer.client.cmd.ClientTask">
        <classpath>
            <fileset dir="${basedir}">
                <include name="guidancer/ClientCmd.jar" />
            </fileset>
        </classpath>
    </taskdef>
    <target name="test" depends="">
        <guidancer 
          project="MyProject" 
          server="gdserver.example.com" 
          port="5556" 
          dburl="jdbc:oracle:thin:@db.example.com:1521:dbname" 
          dbuser="mydblogin" 
          dbpw="mydbpassword"
          testsuites="MyTestSuite1,MyTestSuite2"
          autconfig="MyAUTConfig"
          language="en_US"
          resultdir="."
         />
    </target>
</project>
\end{verbatim}
\normalsize

The above \bxshell{basedir} attribute of the \bxshell{project} element should be set to the \app{}
installation directory, i.e. 
\bxshell{C:$\backslash$Program Files$\backslash$guidancer}. A relative
path can also be used, as seen above. The contents of the
\bxshell{taskdef} section should be left as is. These set up \app{}'s
running environment and include necessary libraries and resource files.

\bxwarn{Changing the \bxshell{taskdef} settings may prevent your tests
from being able to run!}

The \bxshell{target} definition needs to be adjusted to your
environment. The \bxshell{name} property within the \bxshell{target}
tag can be set to your desired name for test execution. The properties
enclosed in the \bxshell{guidancer} tag should be set as follows:

\begin{description}
  \item{\textbf{project}} The name of the \gdproject{} containing the
    \gdsuites{} that you wish to run.
    \item{\textbf{version}} The version of the \gdproject{} containing the
    \gdsuites{} that you wish to run.
 \item{\textbf{autconfig}} The name of the \gdaut{} configuration, as entered
  in your \gdproject. If your \gdproject{} contains only one \gdaut,
  this option may also be omitted.
  \item{\textbf{server}} The name of the computer that \gdagent{} is running
  on.
  \item{\textbf{port}} The port number that the \gdagent{} is running on.
  \item{\textbf{dburl}} The JDBC URL of the \gddb{} containing your \app{}
  \gdprojects. This option may be omitted, in which case the \gddb{}
  configured using \app{}'s configuration tool will be used.
  \item{\textbf{dbuser}} The login name to the \gddb. When using the \gdintdb{} (default installation), this should be set to ''sa''.
  \item{\textbf{dbpw}} The password to login to the \gddb. When using the
  \gdintdb{} (default installation), this should left blank
  (e.g. \verb+dbpw=+'''' ).
  \item{\textbf{testsuites}} The names of the \gdsuites{} that you wish to
  be executed. This option may be omitted, but only if your
  \gdproject{} only contains a single \gdsuite{}.
  \item{\textbf{language}} The language that the test(s) are to be executed
  in. If this property is omitted, the default language for your
  \gdproject{} will be used.
  \item{\textbf{datadir}} The path to the directory where external data are stored. 
\item{\textbf{resultsdir}} The path to the directory where the test results should be stored. 
\end{description}

Once your \bxname{Ant} task and target are set up, your test can be
executed from the command line, using the name given to your target:

\bxshell{ant test}

or called as part of a build script:

\begin{verbatim}
<project name="MyProject">
...
<ant antfile="<location_of_my_gd_build_script>/<name_of_my_gd_ant_file>"
     target="<mytargetname>"
     inheritAll="false"/>
...
</project>
\end{verbatim}
