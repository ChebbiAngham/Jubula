% BREDEX LaTeX Template
%  \documentclass is either ``bxreport'' or ``bxarticle''
%                 option is bxpaper
%% \documentclass{bxarticle}
%% % ----------------------------------------------------------------------
%% \begin{document}
%% \title{}
%% \author{}
%% % \author*{Hauptautor}{Liste der Nebenautoren}
%% \maketitle
%% % ----------------------------------------------------------------------
%% \bxversion{0.1}
%% %\bxdocinfo{STATUS}{freigegeben durch}{freigegeben am}{Verteilerliste}
%% \bxdocinfo{DRAFT}{}{}{}
%% % ----------------------------------------------------------------------

%% \end{document}
\index{Command Line Client}
\begin{enumerate}
\item Once you have browsed to the \jb{} installation directory and entered \bxshell{GUIdancerCMD}, you can enter the parameters for the test execution.
\item The command line client has various parameters \bxfigref{cmdlineparams}:
\bxtipp{A list of language codes is available in the reference manual (\bxextref{\jb{}efman}{ref,langcodes}).}

\label{cmdlineparams}
\tablehead{\hline\textbf{Detail}&\textbf{Parameter}\\\hline}
\begin{supertabular}{|p{4.0cm}p{8.0cm}|}
\hline

                Help 
                &\bxshell{-h}\\
                & Gives parameter help\\
                \hline
                  Project name
                  & \bxshell{-project <project name>}\\
		  &e.g. \emph{-project ''ExampleProject''}\\
                  \hline
                  Project version
                  & \bxshell{-version <project version>}\\
		  &e.g. \emph{-version ''1.3''}\\
		\hline
		\gdaut{} configuration name 
                  & \bxshell{-autconfig <configuration name>}\\
		&e.g. \emph{-autconfig ''localconfiguration''}\\
                \hline
		\gdaut{} ID
                  & \bxshell{-autid <ID>}\\
		&e.g. \emph{-autid ''SimpleAdder1''}\\
		\hline
                Configuration file
                  & \bxshell{-c <path to configuration file>}\\
		&e.g. \emph{-c ''C:/Program Files/guidancer/config''}\\
	        \hline
		Database scheme
                  & \bxshell{-dbscheme <scheme>}\\
		&e.g. \emph{-dbscheme ''Embedded''}\\
		\hline
		Database username
                  & \bxshell{-dbuser <username>}\\
		&e.g. \emph{-dbuser ''joebloggs''}\\
		\hline
		Database password
                  & \bxshell{-dbpw <password>}\\
		&e.g. \emph{-dbpw ''mypassword''}\\
		\hline
	         Server
                  & \bxshell{-server <AUT Agent hostname>}\\
                 &e.g. \emph{-server ''localhost''}\\
		\hline
		Port number
                  & \bxshell{-port <port number>}\\
                   &e.g. \emph{-port ''60000''}\\
		\hline
		Language
                  & \bxshell{-language <language code>}\\
                   &e.g. \emph{-language ''en\_US''}\\
		\hline
		Test Suite
                  & \bxshell{-testsuite <testsuite name>}\\
                   &e.g. \emph{-testsuite ''suite1''}\\
                \hline
                Test Job
                  & \bxshell{-testjob <testjob name>}\\
                   &e.g. \emph{-testjob ''job1''}\\
                \hline
		Data directory
                  & \bxshell{-datadir <path to external} \\
                  & \bxshell{test data directory>}\\
                   &e.g. \emph{-datadir ''C:/Program Files/guidancer/data''}\\
		\hline
		Result directory
                  & \bxshell{-resultdir <path to directory>}\\
                 &e.g. \emph{-resultdir ''C:/Program Files/guidancer/results''}\\
		\hline
		No run option (optional)
                  & \bxshell{-n}\\
                  &e.g. \emph{-n}\\
		&Checks if the \gdsuite{} is startable.\\
		\hline
		Quiet option (optional)
                  & \bxshell{-q} \\
                  &e.g. \emph{-q}\\
		&Does not give out test information.\\
		\hline
		Database URL (optional)
                  & \bxshell{-dburl <URL>}\\
                  &e.g. \emph{-dburl ''db.example.de''}\\
		&If no URL is given, the default will be used.\\
                \hline
		Timeout (optional)
                  & \bxshell{-timeout <timeout in seconds>}\\
                  &e.g. \emph{-timeout ''3600''}\\
		&Enter an optional timeout for the command line client.\\
                \hline
		No screenshot (optional)
                  & \bxshell{-s }\\
                  \hline
		Test results not relevant (optional)
                  & \bxshell{-r }\\
                  &\\
		&Flags the test results as not relevant \\
                & in the test result summary \bxpref{testresprefs}.\\
		\hline
%\caption{Parameters for Command Line Client}
\end{supertabular}

%% \centering
%% \label{cmdlineparams}
%% \begin{longtable}{|l|l|}
%% %\begin{table}[h]
	
%% 	%\begin{tabular}{|l|l|}

%% 	\hline
%% 	\textbf{Detail}&\textbf{Parameter}\\
%% 		\hline
%%                 Help 
%%                 &\bxshell{-h}\\
%%                 & Gives parameter help\\
%%                 \hline
%%                   Project name
%%                   & \bxshell{-project <project name>}\\
%% 		  &e.g. \emph{-project ''ExampleProject''}\\
%%                   \hline
%%                   Project version
%%                   & \bxshell{-version <project version>}\\
%% 		  &e.g. \emph{-version ''1.3''}\\
%% 		\hline
%% 		\gdaut{} configuration name 
%%                   & \bxshell{-autconfig <configuration name>}\\
%% 		&e.g. \emph{-autconfig ''localconfiguration''}\\
%%                 \hline
%% 		\gdaut{} ID
%%                   & \bxshell{-autid <ID>}\\
%% 		&e.g. \emph{-autid ''SimpleAdder1''}\\
%% 		\hline
%%                 Configuration file
%%                   & \bxshell{-c <path to configuration file>}\\
%% 		&e.g. \emph{-c ''C:/Program Files/guidancer/config''}\\
%% 	        \hline
%% 		Database scheme
%%                   & \bxshell{-dbscheme <scheme>}\\
%% 		&e.g. \emph{-dbscheme ''Embedded''}\\
%% 		\hline
%% 		Database username
%%                   & \bxshell{-dbuser <username>}\\
%% 		&e.g. \emph{-dbuser ''joebloggs''}\\
%% 		\hline
%% 		Database password
%%                   & \bxshell{-dbpw <password>}\\
%% 		&e.g. \emph{-dbpw ''mypassword''}\\
%% 		\hline
%% 	         Server
%%                   & \bxshell{-server <AUT Agent hostname>}\\
%%                  &e.g. \emph{-server ''localhost''}\\
%% 		\hline
%% 		Port number
%%                   & \bxshell{-port <port number>}\\
%%                    &e.g. \emph{-port ''60000''}\\
%% 		\hline
%% 		Language
%%                   & \bxshell{-language <language code>}\\
%%                    &e.g. \emph{-language ''en\_US''}\\
%% 		\hline
%% 		Test Suite
%%                   & \bxshell{-testsuite <testsuite name>}\\
%%                    &e.g. \emph{-testsuite ''suite1''}\\
%%                 \hline
%%                 Test Job
%%                   & \bxshell{-testjob <testjob name>}\\
%%                    &e.g. \emph{-testjob ''job1''}\\
%%                 \hline
%% 		Data directory
%%                   & \bxshell{-datadir <path to external test data directory>}\\
%%                    &e.g. \emph{-datadir ''C:/Program Files/guidancer/data''}\\
%% 		\hline
%% 		Result directory
%%                   & \bxshell{-resultdir <path to directory>}\\
%%                  &e.g. \emph{-resultdir ''C:/Program Files/guidancer/results''}\\
%% 		\hline
%% 		No run option (optional)
%%                   & \bxshell{-n}\\
%%                   &e.g. \emph{-n}\\
%% 		&Checks if the \gdsuite{} is startable.\\
%% 		\hline
%% 		Quiet option (optional)
%%                   & \bxshell{-q} \\
%%                   &e.g. \emph{-q}\\
%% 		&Does not give out test information.\\
%% 		\hline
%% 		Database URL (optional)
%%                   & \bxshell{-dburl <URL>}\\
%%                   &e.g. \emph{-dburl ''db.example.de''}\\
%% 		&If no URL is given, the default will be used.\\
%%                 \hline
%% 		Timeout (optional)
%%                   & \bxshell{-timeout <timeout in seconds>}\\
%%                   &e.g. \emph{-timeout ''3600''}\\
%% 		&Enter an optional timeout for the command line client.\\
%%                 \hline
%% 		No screenshot (optional)
%%                   & \bxshell{-s }\\
%%                   \hline
%% 		Test results not relevant (optional)
%%                   & \bxshell{-r }\\
%%                   &\\
%% 		&Flags the test results as not relevant in the test result summary \bxpref{testresprefs}.\\
%% 		\hline
%% \caption{Parameters for Command Line Client}
%%         \end{longtable}
	
 
\item You can either enter a \gdsuite{} to be executed or a \gdjob{}. Only one of these two commands is accepted for the command line client. 
\item If the \gdaut{} you want to test was started with the \bxname{gdrun} command, you must enter an \gdaut{} ID. If your \gdaut{} will be started with an \gdaut{} configuration, then enter the configuration name. 
%For example, two \gdsuites{} called \bxname{Test,Part1} and \bxname{Test,Part2} would have to be entered as:\\
%\bxshell{-testsuites ''Test$\backslash$Part1,Test$\backslash$Part2''}
%\item If you have used quotes in your \gdsuite{} name, and these are a special symbol for your operating system, then the quotes must be masked with backslashes. 

\bxtipp{The \jb{} Command Line Client also accepts arguments to pass on to the Java Virtual Machine. This means that you can, for example, increase the initial and maximum amount of system memory allocated to the JVM with the parameters \bxshell{-Xms<memory\_size>} and \bxshell{-Xmx<memory\_size>}, respectively. For example, the parameter \bxshell{-Xmx128M} would make a maximum of 128 MB of system memory available to the Command Line Client. When entering the standard parameters for the Command Line Client, you may add \bxshell{-J<JVM\_parameter>} for each JVM parameter you wish to set. For example, \bxshell{-J-Xmx128M}. Multiple parameters, like standard parameters, are separated by spaces. \\
Here is an example of defining mutliple JVM parameters: \bxshell{-J-Xmx128M -JXms128M}.}
\item Using the no run option will check the completeness of the \gdprojects{}, test data and the validity of the \gddb{} connection. The \gdserver{} connection is not checked. 
\item Once you have entered all the necessary parameters, press \bxkey{enter}. 
\item The client will connect to the \gdserver, connect to the \gddb{}, open the \gdproject{}, start the \gdaut{} and then execute the \gdsuites{} you specified, in the order you wrote them.
\item Once the test has finished, the client will show an exit code.
\begin{itemize}
  \item \bxcaption{Exit code: 0} indicates that the test was successful.
  \item \bxcaption{Exit code: 1} indicates that the test contained an error.
\end{itemize}
\bxtipp{To stop the test execution, use \bxkey{Ctrl+C}}
\end{enumerate}
