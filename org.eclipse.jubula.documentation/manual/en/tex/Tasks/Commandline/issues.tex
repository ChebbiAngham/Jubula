
The following describes aspects concerning the Command Line Client
which should be considered when specifying your tests.

\subsubsection{Batch vs. Interactive}

Specifying tests with the \jb{}{}  GUI is necessarily an \emph{interactive}
process, as the user is actively working with the program, providing
input and responding to feedback. 

Test execution, however, can be
carried out either in an interactive or a \emph{batch}
process. \jb{}{} provides some test actions that are meant for interactive
use, such as the \bxcaption{Pause} action for the
\bxcaption{Application}  component. This action pauses test
execution, waiting for the user command to either continue or stop
test execution.

However, when tests are intended for \emph{batch} execution, as is the
case when using the Command Line Client, such
actions lose their meaning, as no user is expected to be present to
respond. \jb{} thus ignores interactive actions in this case.

\bxtipp{We recommend starting tests with a \bxname{wait for component} action. This eliminates timing problems with batch runs. }

\subsubsection{When Errors Occur}

If an error occurs in a \gdsuite{}, such that it can no longer be
executed to the finish, the Command Line Client will automatically
begin testing with the next specified \gdsuite{}, if one exists.
