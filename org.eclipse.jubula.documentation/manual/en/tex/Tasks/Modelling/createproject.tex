\index{Project!Create for Model}
\index{Create Model Project}

The first thing to do when you want to work with models in \gd{} is to create an Eclipse Project in which the models will be managed and saved. The \gdproject{} will be saved in your workspace. 

\begin{enumerate}
\item Switch to the modeling perspective if you have not already done so. You can do this by selecting:\\
\bxmenu{Window}{Open Perspective}{Modeling Perspective}
\item In the \gdnavview{}, click the \bxcaption{Create New Project} button
or select:\\
\bxmenu{New}{Project}{}\\
from the context-sensitive menu. 
\item In the wizard that appears, select \bxname{Project} from the \bxname{General} category. 
\item Click \bxcaption{Next}
\item Enter a \gdproject{} name. You can decide whether to keep the \gdproject{} in your workspace or link to it from your workspace. 
\item Click \bxcaption{Finish}. The \gdproject{} you just created will appear in the \gdnavview{}. It contains the file \bxname{.project}. Do not delete this file!
\item Now you have created a \gdproject{}, you can create a new model diagram \bxpref{TasksMBTCreateModel} or import models \bxpref{TasksMBTImport}. 
\end{enumerate}
