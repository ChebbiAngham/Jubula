\app{} also lets you create manual tests to be executed and analyzed in the \app{} software. 

A \gdcase{} containing manual \gdsteps{} is the same as an automated \gdcase{} for all intents and purposes. You can add data, reuse it in other \gdcases{}, and add it to \gdsuites{} to be executed. The only difference is during execution: manual tests are executed in an interactive mode \bxpref{TasksManualExec}. 

To create a manual test:
\begin{enumerate}
\item Create a \gdcase{} \bxpref{TasksCreateTC}
\item Add the module for \bxname{manual \gdcase} from the library of \gdcases{} (category: Manual \gdstep{}).
\item In the \gdpropview{}, you can enter three pieces of data:
\begin{description}
\item[The action(s) to perform:]{Enter a short description which should appear when the test is being executed. You can also enter referenced parameters  \bxpref{TasksTestdataReferences} or variables \bxpref{TasksVariables} here to specify different data for the test.}
\item [The expected behaviour:]{Enter the description (any any data) of the expected outcome of the \gdstep{}.}
\item [Timeout:]{Enter how long the test execution should wait before receiving the information that the \gdstep{} passed or failed. We recommend setting this timeout value high enough for a manual tester to be able to complete the \gdstep{} and enter any comments.}
\end{description}
\item Once your manual \gdcase{} is finished, then you can add it to a \gdsuite{} \bxpref{TasksEditorAdd} to be executed \bxpref{TasksManualExec}.
\end{enumerate}

\bxtipp{We recommend keeping manual \gdcases{} in a separate category to automated \gdcases{}. Although it is possible to combine manual and automated \gdcases{} in one test, make sure you know which \gdsuites{} can be run completely automatically for your unattended tests!}
