Once a test with code coverage has run, the  code coverage information must first be processed in order to show it in the \gdtestsummaryview{}. 

The following columns are available in the \gdtestsummaryview{} to show you code coverage information:

\bxtipp{ If a column is not visible, you can show it by selecting it from the context-sensitive menu on the header row for the \gdtestsummaryview{}.}

\begin{itemize}
\item The \bxname{Profiling Agent} column displays which code coverage agent was selected for the test run.
\item The \bxname{Measured Value} column shows the Instruction Coverage for the test run. You can see the values for other types of coverage in the \gdpropview{} when you select the test run. An overview of the meanings of the coverage types is available on the JaCoCo website: \\
\bxname{http://www.eclemma.org/jacoco/trunk/doc/counters.html}. 
\bxtipp{The measured value will become visible once the code coverage information has been processed. You may need to refresh the \gdtestsummaryview{}. }
\item The \bxname{Profiling Details} column indicates whether the full details for the code coverage report are still in the \gddb{}. Full profiling details are removed from the \gddb{} at the same time as test run details \bxpref{TasksReopenTestResult}. The period of time that details remain in the \gddb{} can be specified in the \gdproject{} properties \bxpref{ProjPropertiesGeneral}.
\end{itemize}

\textbf{Opening and viewing the code coverage report}\\

\begin{enumerate}
\item In the \gdtestsummaryview{}, select the test run whose code coverage you want to see and click the \bxcaption{Open Profiling Report} button at the top of the view. 
\bxwarn{The details about code coverage are available in the \gddb{} for the same amount of time as the test result details \bxpref{TasksReopenTestResult}. After this time, the code coverage details are deleted along with the test run details.}
\item In the editor that opens, you can see the details for the code coverage for the test run. 
\item If you specified a source file directory for your \gdaut{} \bxpref{CCConfigure} and compiled your \gdaut{} classes with debug information, you will be able to navigate through your classes to see more detailed information about the code coverage through the whole \gdaut{}. 
\item Once you have opened a code coverage report, it is saved into your workspace. You can reopen it from the \gdnavview{}. 
\bxtipp{You should regularly remove old code coverage reports from your workspace to avoid overfilling it.}
\end{enumerate}



