% $Id: objectmapping.tex 6021 2007-11-19 15:03:56Z alexandra $
% Local Variables:
% ispell-check-comments: nil
% Local IspellDict: american
% End:
% --------------------------------------------------------
% User documentation
% copyright by BREDEX GmbH 2005
% --------------------------------------------------------
% this command can be inserted multiple times
%\gdhelpid{}
% 
%\begin{gddescription}
%\end{gddescription}
%
%\begin{gdlist}
% use the \item command for single steps
%\end{gdlist}
% change <PATH> to the same directory, file is located in
% change <FILE> to the same filename you are editing
%\bxinput{<PATH>/Links/<FILE>}
%
% other usefull commands are
%   \bxtipp{}        to create a hint
%   \bxwarn{}        to describe a warning

\index{objectmapping}

\textbf{Green border does not appear in Java \gdauts{}}\\
If you are in the \gdomm{}  and cannot see a green border around components, then this could mean the following:
\begin{itemize}
\item The border cannot be drawn -- try collecting the component anyway and see if the technical name appears in the \gdomeditor{}. 
\item The component is not supported -- find out whether the component in your \gdaut{} is a standard component or a custom component. If it is a custom component, you may need to extend \app{} to be able to test it. A simple mechanism for adding basic support is available directly in the \gdproject{} settings \bxpref{TasksBasicExtension}.
\end{itemize}


\textbf{Problems with GTK}\\
The GTK graphics toolkit (often used with Linux) can make object mapping with the default \app{} settings impossible for certain components. If the \gdaut{} that you are mapping is running under Linux, please change the object mapping shortcut key combination to anything \emph{other than} \bxkey{Ctrl+Shift+[0-9 or A-F]}.
