% $Id: adapt.tex 7860 2009-02-25 17:10:42Z alexandra $
% Local Variables:
% ispell-check-comments: nil
% Local IspellDict: american
% End:
% --------------------------------------------------------
% User documentation
% copyright by BREDEX GmbH 2004
% --------------------------------------------------------
\label{adapting views}
\index{User Interface!Adapting}
\index{Preferences!Reset}
\index{View!Show}
\index{Perspective!Specification}
\index{Perspective!Execution}
\index{Execution!Perspective}
\index{Specification!Perspective}
You can individualize the \jb{} user interface in various ways. 

\subsection{Color-coding the user interface}
%appearancehelpid
\begin{enumerate}
\item To change the color-coding on the borders of related browsers, views and 
editors, select:\\
\bxmenu{Window}{Preferences}{} \\
and select \bxcaption{Appearance} from the list on the left-hand side.
\item Use the \bxcaption{color} buttons to select a colors for your client.

\item Choosing a color for \bxcaption{Specification} will make the borders of the \gdtestcasebrowser{}, \gdtestcaseeditor{} and their three support views all have the color you choose.

\item By choosing a color for \bxcaption{Execution}, the borders of the \gdtestsuitebrowser{}, \gdtestsuiteeditor{} and their three support views will all be in the color you choose.

\item You can also check the box to switch off the color-coding option. 
\end{enumerate}

\subsection{Moving Browsers, Views and Editors}
You can move items in the user interface  in two ways:

\begin{itemize}
\item Drag-and-drop views or browsers. While the mouse button
 is held, the target location is marked by a gray rectangle.
\item Right mouse-click on the tab area and select
\bxcaption{Move} and then either \bxcaption{View} or \bxcaption{Tab Group}.
Single-click to drop the item. 
\end{itemize}

\subsection{Resizing in the user interface}

To change the size of the views and browsers:
\begin{itemize}
\item Double-click on the tab of a view, browser or editor to maximize it. Double-click again to minimize it.
\item  Use the buttons in the  top right-hand corner of the view or browser to 
minimize, maximize or restore it.  
\item Drag the borders of the view or browser to enlarge or reduce it.
\item Right mouse-click on the name of the view or browser and select
\bxcaption{Size} and then the side to be adjusted. The chosen side 
appears as a blue line which can be dragged and dropped. 
\item To turn the view or browser into a separate window, select 
\mbox{\bxcaption{Detached}}
 from
the context-sensitive menu of the tab area. 
\end{itemize}  

%\subsection{Ordering items in browsers}
%\gdhelpid{testExecViewContextId}{Test Suite Browser}
%\gdhelpid{testSpecificationViewContextId}{Test Case Browser}
%% To alter the way the items in a view or browser are shown:
%% \begin{itemize}
%% \item Select the icon in the right-hand corner of a view or browser
%%  to  entirely collapse the trees.
%% \gdmarpar{../../../share/PS/collapse}{collapse tree}
%% \item Right-click on a node in the browser and select \bxcaption{expand} to 
%% expand that node alone. 
%% \item To arrange items alphabetically, activate the leftmost icon
%% in the right-hand 
%% corner of the view or browser.
%%  To reorder the icons as they were previously, click
%% the icon again. 
%% \gdmarpar{../../../share/PS/sort}{ arrange/reorder}
%% \end{itemize}

\subsection{Restoring user interface defaults}
\begin{enumerate}
\item You can restore the default perspective at any time: \\ 
\bxmenu{Window}{Reset Perspective}{}.
\item To show or restore individual views or browsers, choose either the \bxcaption
{restore} icon in the tab for the view or  select: \\
\bxmenu{Window}{Show View}{}\\
 and then choose which view to display.
\end{enumerate}

\subsection{Changing perspectives}
\begin{enumerate}
\item To change between perspectives, select:\\
\bxmenu{Window}{Open Perspective}{}.
\item Select which perspective to 
open. 

\item Alternatively, you can use the icons in the top left-hand
corner of the current perspective to toggle between perspectives. 
\end{enumerate}

\subsubsection{Automatically changing perspective}
%\gdhelpid{prefPageBasicContextId}{GUIdancer Preferences}
By selecting:\\
 \bxmenu{Window}{Preferences}{},\\
you can choose to automatically change to the Execution perspective when  test execution starts, 
or to be asked each time a \gdsuite{} is started. 








