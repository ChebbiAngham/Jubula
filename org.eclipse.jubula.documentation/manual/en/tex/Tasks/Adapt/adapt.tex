% $Id: adapt.tex 7860 2009-02-25 17:10:42Z alexandra $
% Local Variables:
% ispell-check-comments: nil
% Local IspellDict: american
% End:
% --------------------------------------------------------
% User documentation
% copyright by BREDEX GmbH 2004
% --------------------------------------------------------
\label{adapting views}
\index{User Interface!Adapting}
\index{Preferences!Reset}
\index{View!Show}
\index{Perspective!Specification}
\index{Perspective!Execution}
\index{Execution!Perspective}
\index{Specification!Perspective}
You can individualize the  user interface in various ways. 

\subsection{Moving Browsers, Views and Editors}
You can move items in the user interface  in two ways:

\begin{itemize}
\item Drag-and-drop views or browsers. While the mouse button
 is held, the target location is marked by a gray rectangle.
\item Right mouse-click on the tab area and select
\bxcaption{Move} and then either \bxcaption{View} or \bxcaption{Tab Group}.
Single-click to drop the item. 
\end{itemize}

\subsection{Resizing in the user interface}

To change the size of the views and browsers:
\begin{itemize}
\item Double-click on the tab of a view, browser or editor to maximize it. Double-click again to minimize it.
\item  Use the buttons in the  top right-hand corner of the view or browser to 
minimize, maximize or restore it.  
\item Drag the borders of the view or browser to enlarge or reduce it.
\item Right mouse-click on the name of the view or browser and select
\bxcaption{Size} and then the side to be adjusted. The chosen side 
appears as a blue line which can be dragged and dropped. 
\item To turn the view or browser into a separate window, select 
\mbox{\bxcaption{Detached}}
 from
the context-sensitive menu of the tab area. 
\end{itemize}  

\subsection{Restoring user interface defaults}
\begin{enumerate}
\item You can restore the default perspective at any time: \\ 
\bxmenu{Window}{Reset Perspective}{}.
\item To show or restore individual views or browsers, choose either the \bxcaption
{restore} icon in the tab for the view or  select: \\
\bxmenu{Window}{Show View}{}\\
 and then choose which view to display.
\end{enumerate}

\subsection{Changing perspectives}
\begin{enumerate}
\item To change between perspectives, select:\\
\bxmenu{Window}{Open Perspective}{}.
\item Select which perspective to 
open. 

\item Alternatively, you can use the icons in the top left-hand
corner of the current perspective to toggle between perspectives. 
\end{enumerate}

\subsubsection{Automatically changing perspective}

By selecting:\\
 \bxmenu{Window}{Preferences}{},\\
you can choose to automatically change to the \execpersp{} when  test execution starts, 
or to be asked each time a \gdsuite{} is started. 

\subsection{Keyboard shortcuts}

The \ite{} offers various keyboard shortcuts to make working with the keyboard easier and quicker. 

Most of the shortcuts can be changed in the \bxname{Preferences} in the \ite{}. Shortcuts for object mapping and observation are in the preferences. Other shortcuts can be seen and changed in the \bxname{General/Keys} preferences.

\bxtipp{Press \bxkey{Ctrl+Shift+l} twice to see the list of shortcuts used in the \ite{} and Eclipse. }





