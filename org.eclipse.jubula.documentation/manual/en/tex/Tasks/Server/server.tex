% $Id: server.tex 12283 2010-09-23 10:34:38Z alexandra $
% Local Variables:
% ispell-check-comments: nil
% Local IspellDict: american
% End:
% --------------------------------------------------------
% User documentation
% copyright by BREDEX GmbH 2005
% --------------------------------------------------------
% this command can be inserted multiple times
%\gdhelpid{}
% 
%\begin{gddescription}
%\end{gddescription}
%
%\begin{gdlist}
% use the \item command for single steps
%\end{gdlist}
% change <PATH> to the same directory, file is located in
% change <FILE> to the same filename you are editing
%\bxinput{<PATH>/Links/<FILE>}
%
% other usefull commands are
%   \bxtipp{}        to create a hint
%   \bxwarn{}        to describe a warning

\index{AUT Agent!Starting}
\index{Start!AUT Agent}
\label{otherportnumber}
The \gdagent{} is the server component of \jb{}. It runs on the same machine as the \gdaut{} and allows \jb{} to communicate with the \gdaut{} and control it during test execution. 

\subsection{Windows users}
\begin{enumerate}
\item Start the \gdagent{} via the start menu:\\
\bxmenu{Start}{\jb{}}{Start \gdagent{}}. 
\item The  \gdagent{} is started on port number 60000 unless you enter a different port number as a parameter \bxpref{tasksagentcmd}. 



You can see and stop the \gdagent{} in the system tray.  

\end{enumerate}

\subsection{Linux users}

Use the script:\\
 \bxshell{./autagent (-p <port number>)}. 

You can see and stop the \gdagent{} in the system tray. 

\subsection{Starting the \gdagent{} from the command line: options and parameters}
\index{AUT Agent!Quiet Mode}
\index{AUT Agent!Parameters}
\index{AUT Agent!Quiet Mode}
\index{AUT Agent!Parameters}
\label{tasksagentcmd}
\label{TasksAgentLenient}

\begin{enumerate}
\item Start the \gdagent{} from the Windows command line with this command:\\
\bxshell{autagent.exe (-p <port number>)}

\item If no port number argument is given, the \gdagent{}  will start on the default port 60000. 

\item You can use the following parameters when starting the \gdagent{}:

\begin{description}
\item [\bxshell{-p}:]{Port number. Enter the port number you wish to start the \gdagent{} on. }
\bxtipp{You can also use the environment variable \bxcaption{GD\_AUT\_AGENT\_PORT} to set the port number that should be used as a default when you start the \gdagent{}.}
\item[\bxshell{-l}:]{Lenient mode. The lenient mode for the \gdagent{} allows \gdauts{} launched  by other \gdauts{} to be tested \bxpref{TasksLenientTest}. You can also change the mode of the currently running \gdagent{} via the system tray. Right-click the \gdagent{} icon and (de)select \bxname{Strict \gdaut{} Management} from the menu.}
\bxtipp{The default mode for the \gdagent{} is \bxname{strict}.}
\item [\bxshell{-v}:]{Verbose mode. You will see a dialog to tell you whether the \gdagent{} has started successfully or not.}
\item [\bxshell{-q}:]{Quiet mode. You will see no dialog if the \gdagent{} starts successfully. If the \gdagent{} does not start successfully, the error is written to the console.}
\item [\bxshell{-h}:]{Help. Enter this parameter to see a list of options for the \gdagent{}.}
\end{description}

\item The \gdagent{} can be started multiple times -- either on the same machine (using a different port number for each instance) or on multiple machines for distributed testing. 

\end{enumerate}

