% $Id: server.tex 12283 2010-09-23 10:34:38Z alexandra $
% Local Variables:
% ispell-check-comments: nil
% Local IspellDict: american
% End:
% --------------------------------------------------------
% User documentation
% copyright by BREDEX GmbH 2005
% --------------------------------------------------------
% this command can be inserted multiple times
%\gdhelpid{}
% 
%\begin{gddescription}
%\end{gddescription}
%
%\begin{gdlist}
% use the \item command for single steps
%\end{gdlist}
% change <PATH> to the same directory, file is located in
% change <FILE> to the same filename you are editing
%\bxinput{<PATH>/Links/<FILE>}
%
% other usefull commands are
%   \bxtipp{}        to create a hint
%   \bxwarn{}        to describe a warning

\index{AUT Agent!Starting}
\index{Start!AUT Agent}
\label{otherportnumber}
The \gdagent{} is the server part of \gd{}. It runs on the same machine as the \gdaut{} and allows \gd{} to communicate with the \gdaut{} and control it during test execution. 

\subsection{Windows users}
\begin{enumerate}
\item Start the \gdagent{} via the start menu:\\
\bxmenu{Start}{\GD{}}{Start \gdserver{}}. 

\item This starts the \gdserver on the 
default port number given during installation (usually 60000).
\bxtipp{You can find out/change the default port number in the Configuration Tool.}

You can see and stop the \gdagent{} in the system tray.  

\item  For information on using the command line interface to start the \gdserver, and which parameters can be used to start it, see the later section \bxpref{tasksservercmd}. 
\end{enumerate}

\subsection{Linux users}

Use the script:\\
 \bxshell{./autagent (-p <port number>)}. 

You can see and stop the \gdagent{} in the system tray. 

\subsection{Starting the \gdserver from the command line}
\label{tasksservercmd}
\begin{enumerate}
\item Start the \gdserver from the command line with this command:\\
\bxshell{autagent.exe (-p <port number>)}

\item If no port number argument is given, 
the \gdserver will start on the default port, specified during installation. 

\item  For information which parameters can be used to start the \gdserver{}, see the later section \bxpref{tasksservercmd}. 

\end{enumerate}

\subsubsection{Parameters for the \gdserver}
\index{AUT Agent!Quiet Mode}
\index{AUT Agent!Parameters}
\begin{enumerate}
\item If you are starting the \gdserver from the command line interface or shell you can start it in one of two modes: verbose and quiet by entering the following parameters after the \gdserver{} command:\\
\begin{description}
\item [Verbose: \bxshell{-v}]{You will see a dialog to tell you whether the \gdserver{} has started successfully or not.}
\item [Quiet \bxshell{-q}]{You will see no dialog if the \gdserver{} starts successfully. If the \gdserver{} does not start successfully, the error is written to the console.}
\end{description}

\end{enumerate}
\bxtipp{The parameter \bxshell{-h} displays a list of options for the \gdserver{}}. 

\subsubsection{Starting the \gdagent{} in lenient mode}
\label{TasksAgentLenient}
\index{AUT Agent!Quiet Mode}
\index{AUT Agent!Parameters}

The lenient mode for the \gdagent{} allows \gdauts{} launched  by other \gdauts{} to be tested \bxpref{TasksLenientTest}.

To start the \gdagent{} in the lenient mode, ener \bxshell{-l} as a command line parameter when starting the \gdagent{}. 

You can also change the mode of the currently running \gdagent{} via the system tray. Right-click the \gdagent{} icon and (de)select \bxname{Strict \gdaut{} Management} from the menu. 

\bxtipp{The default mode for the \gdagent{} is \bxname{strict}.}


\subsection{Starting the \gdserver more than once}
\begin{enumerate}
\item You can start the \gdserver{} multiple times using a different
port number for each instance. 
\item You can start the \gdserver on multiple machines, as long as the \gdserver is installed on them.
\item Ask your system administrator for available  port numbers.
\item Give the port number for each \gdserver either as an environment variable
called \bxcaption{GD\_AUT\_AGENT\_PORT} or as a command line argument. 
\end{enumerate}

