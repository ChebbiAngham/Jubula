\index{Connect!AUT Agent}
\index{AUT Agent!Connect}
\index{AUT Agent!Embedded}
\index{Embedded AUT Agent}

You can connect to an \gdagent{} you have started \bxpref{TasksAgentExternal} on a local or remote machine, or you can connect an embedded \gdagent{} on your local machine. 

\bxtipp{When using a separate \gdagent{}, you must wait for the \gdagent{} to be running before you can connect to it.}
\begin{enumerate}
\item On the toolbar, click on the arrow next to the \bxcaption{Connect to \gdagent} button. 
\gdmarpar{../../../share/PS/startServer}{ connect to \gdagent}
\item In the drop down list, you can choose an \gdagent{} to connect to:
\begin{itemize}
\item Select the embedded \gdagent{} option to connect to an \gdagent{} that is automatically started for you on your machine.  
\bxtipp{The embedded \gdagent{} uses port 60000 by default in the \ite{}. You can change the port that should be used for the embedded \gdagent{} in the preferences \bxpref{TasksPrefsEmbeddedAgent}.}
\item Select an \gdagent{} host and port number to connect to from the list. The list of hosts and ports available in this list can be configured in the preferences \bxpref{TasksPrefsAgent}. If you have set the environment variable \bxname{TEST\_AUT\_AGENT\_PORT}, then you will also see the port number you set for that variable in the drop-down list. 
\bxwarn{If you are not working with the embedded \gdagent{}, you must have started an \gdagent{} on a machine in the network \bxpref{TasksAgentExternal} to be able to connect to it.}
\end{itemize}
\end{enumerate}

\bxtipp{To work with the embedded \gdagent{}, you do not need to start an \gdagent{} on your machine. However, you can only work with the embedded \gdagent{} for local testing (i.e. on the same machine as the \ite{}). The embedded agent is started in \bxname{lenient} mode \bxpref{TasksLenientTest}.}
