You can add a task ID to \gdcases{}, \gdsuites{} and \gdjobs{} in your \gdproject{}. 

The task ID should be a valid ID in the repository that you have specified as the repository for this \gdproject{} \bxpref{TasksALMConfigureProject}. Adding the task ID to an item in your \gdproject{} means that this item is the relevant test for that task in your repository. When you activate the option, any test results for this item will be added as a comment to the task in the repository. The comment will include a link to the dashboard, in which the test result report can be viewed.

To add a task ID to a \gdcase{}, \gdsuite{} or \gdjob{}:
\begin{enumerate}
\item Open the item in the editor by double-clicking it.
\item In the \gdpropview{}, in the cell for \bxname{Task ID}, enter the task ID from the external repository. You can only enter task IDs at the place of specification -- you cannot overwrite them when you reuse the item.
\item Save the editor. 
\item When you have added a task ID to a node, you can open the task for this node from the browser by selecting:\\
\bxmenu{Open with}{Mylyn Task Editor}{}
\end{enumerate}

\bxtipp{You should ensure that you add task IDs to the right node-level to provide you with the relevant amount of information for the tasks in your repository. This will usually be at the level of Use Cases within a \gdsuite{}. }
