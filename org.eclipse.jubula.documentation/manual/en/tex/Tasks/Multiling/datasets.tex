% BREDEX LaTeX Template
%  \documentclass is either ``bxreport'' or ``bxarticle''
%                 option is bxpaper
%% \documentclass{bxarticle}
%% % ----------------------------------------------------------------------
%% \begin{document}
%% \title{}
%% \author{}
%% % \author*{Hauptautor}{Liste der Nebenautoren}
%% \maketitle
%% % ----------------------------------------------------------------------
%% \bxversion{0.1}
%% %\bxdocinfo{STATUS}{freigegeben durch}{freigegeben am}{Verteilerliste}
%% \bxdocinfo{DRAFT}{}{}{}
%% % ----------------------------------------------------------------------

%% \end{document}
%% \label{multilingusingds}
%% \begin{itemize}
%% \item The \gddatasetsview{} allows you to translate data, and to create data sets for your \gdcases{}.
%% \item A \gdcase{} repeats for each data set in the \gddatasetsview{}.
%% \item In this way, you can reuse the same \gdcase{} with different data.
%% \item Data in the master template (originally specified \gdcase{}) are used as default data whenever you reuse the \gdcase{}.
%% \item You can overwrite this data when you reuse the \gdcase{}. 
%% \item If you have overwritten data in a reused \gdcase{}, the data will not be affected by changes to data in the maaster template. 
%% \item If you do not overwrite data in a reused \gdcase{}, changes to data in the master template will also appear in the reused \gdcase{}. 
%% \end{itemize}

%% \textbf{Using the Combo Boxes in the \gddatasetsview{}:}
%% \begin{enumerate}
%% \item The three combo boxes in the \gddatasetsview{} let you see the data you have entered, and enter new data from different perspectives.
%% \item The standard setting is to select a language from the combo box on the right.
%% \item This shows your parameters and the data for them for the selected language.
%% \item You can also select the middle combo box. This shows the parameters and their values in  all the languages for the data set you selected. 
%% \item Selecting the combo box on the left selects one parameter for this \gdcase{} and shows you all of its data sets in all of the languages. 
%% \end{enumerate}

