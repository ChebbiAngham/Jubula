% BREDEX LaTeX Template
%  \documentclass is either ``bxreport'' or ``bxarticle''
%% %                 option is bxpaper
%% \documentclass{bxarticle}
%% % ----------------------------------------------------------------------
%% \begin{document}
%% \title{}
%% \author{}
%% % \author*{Hauptautor}{Liste der Nebenautoren}
%% \maketitle
%% % ----------------------------------------------------------------------
%% \bxversion{0.1}
%% %\bxdocinfo{STATUS}{freigegeben durch}{freigegeben am}{Verteilerliste}
%% \bxdocinfo{DRAFT}{}{}{}
%% % ----------------------------------------------------------------------

%% \end{document}
%% \begin{itemize}
%% \item To be able to translate data, you must use \bxname{references} in your \gdsteps{}.
%% \item This means that you can enter parameters and data sets at the \gdcase{} level. 
%% \item These data can then be translated in the \gddatasetsview{}. 
%% \item For more information on using references as placeholders, see later in this chapter \bxpref{referencetasks}.
%% \item Once you have set up your \gdcase{} so that you can overwrite data (i.e. you have used references in the \gdsteps{} it contains) you can enter data for each language in the \gddatasetsview{}.
%% \item You can add data either in the master template of the \gdcase{}, or when you reuse it.
%% \item If you add data in the master template, you will be able to add data for all \gdproject{} languages.
%% \item If you add data when you reuse the \gdcase{} in a \gdsuite{}, you will only be able to add data for the languages that the \gdaut{} for this \gdsuite{} supports. 
%% \item Like any other parameter values which have been moved up using references, you can overwrite the data every time you reuse the \gdcase{}. 
%% \item For more information on using the \gddatasetsview{}, see the following section \bxpref{multilingusingds}. 
%% \end{itemize}

%% \textbf{To enter data for a language:}

%% \begin{enumerate}
%% \item Open the editor for the \gdcase{} you want to add data to.
%% \item Single-click on the \gdcase{} to select it.
%% \item In the \gddatasetsview{}, select the language from the combo box you are entering data for. The current working language will automatically be selected in the \gddatasetsview{}. It is a good idea to start with this language.
%% \item Select \bxcaption{Add} to add a data set.
%% \item You can now enter parameter values for all the parameters you have moved up to this \gdcase{}. 
%% \end{enumerate}
%% \bxtipp{Bear in mind that the  parameter values for the first  data set will be shown in the \gdpropview{} in the \bxname{default language}. You cannot translate data in the \gdpropview{}, only in the \gddatasetsview{}.}

%% \textbf{Translating data:}

%% \begin{enumerate}
%% \item Any data sets you have entered for one language will automatically appear as empty data sets for the other languages.
%% \item In the \gddatasetsview{} for the \gdcase{} whose data you want to translate, select the language you want to translate into from the combo box.
%% \item Enter the parameter values for this language in the fields. 
%% \end{enumerate}

%% \bxwarn{At the moment, \gd{} only allows the same number of data sets for each language. Deleting a row in one language will delete it in all of the languages for this instance of the \gdcase{}. }
