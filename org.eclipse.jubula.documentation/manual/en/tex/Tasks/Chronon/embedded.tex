The embedded support for Chronon in \app{} lets you record debug information while you are working with \app{}. Should an error occur, the Chronon report can be added to any bug reports or support conversations to help our team analyse the problem. 

\subsubsection{Turning on the recording}
\begin{enumerate}
\item First, you must make the Chronon embedded recorder available for \app{}. To do this, you must open the \app{}.ini file and remove the comment symbol from the last two lines. Then you must (re)start \app{}. 
\item You can activate the Chronon recording in \app{} by selecting:\\
\bxmenu{Help}{Start Chronon Recorder}{}\\
\item The Progress View will open and you will see the status of the recording. The progress bar continues to run once the recorder has started. The status changes from \bxname{starting} to \bxname{running}. When the recorder is running, you will be able to see the job running in the Progress View and also in the status bar. Do not stop the job -- this will stop the recording. 
\item Once the recorder is running, your actions in \app{} are written to the recording file.
\bxtipp{Bear in mind that any data contained within \app{} may be visible by our development team when analysing any Chronon files}. 
\end{enumerate}

\bxtipp{We do not recommend running the Chronon recorder constantly, as the collection of additional information can lead to detractions in performance in \app{}. Instead, we would recommend turning Chronon on when you specifically want to reproduce a problem, and turning it off afterwards. Additionally, you should ensure that you have increased the heap space to avoid running into memory problems.}
 
\subsubsection{Turning off the recording}
You can deactivate the Chronon recording by opening the Progress View and clicking the \bxcaption{Stop} button. 

\subsubsection{Accessing the Chronon report}
The Chronon reports are saved to your home directory under:\\
\bxname{.jubula/Chronon}\\

When working with Chronon in \app{} itself, a new folder is created for each recording session (one session = starting - recording - stopping). The relevant folder for a bug report can be zipped and attached to help our analysis. 


\subsubsection{Increasing the heap space to improve working with Chronon}
Having the Chronon recorder running is very memory-intensive. You will almost certainly have to increase the heap space allocated to \app{}. You can do this in the .ini file for \app{}. 



