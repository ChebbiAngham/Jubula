\index{Chronon!Support for AUTs}

You can configure your \gdaut{} to run with Chronon recording so that you can analyze any errors that occur during your automated test runs. See the section later \bxpref{TasksChrononTools} on analysing the reports for information on the tooling required to do this.

\bxtipp{\jb{} Feature users: The support for Chronon recording in \gdauts{} is currently only available in the standalone software.}

\subsubsection{Adding Chronon information in the \gdaut{} configuration}
\gdhelpid{autConfigSettingWizardPagePageContextId}{Configuring an AUT}
You can select Chronon as a monitoring agent and configure it in the Expert Settings in the \gdaut{} configuration. 

\begin{enumerate}
\item Open the \gdaut{} configuration dialog from the \gdproject{} properties \bxpref{projectproperties}.
\item Select the \bxname{Expert} configuration. 
\item Select \bxname{Chronon (separate installation)} as a monitoring agent \bxpref{TasksChrononAUTSeparate}. 
\item You can then enter the configuration details for the monitoring.
\end{enumerate}

\subsubsection{Configuring the separate Chronon installation for use with your \gdaut{}}
\gdhelpid{autConfigSettingWizardPagePageContextId}{Configuring an AUT}
\label{TasksChrononAUTSeparate}

Enter the following configuration details:

\begin{description}
\item [Java Agent JAR:]{Enter the path to the Chronon recorder JAR you have installed.}
\item [Native Agent File:]{The path to the platform-specific native recorder agent library.}
\item [Config file:]{Enter the absolute path to the configuration file for the recorder. The configuration file has the same setup as a Java properties file. It must contain the following properties:\\
\bxshell{servermode} (set to true)\\
\bxshell{name} (the \gdaut{} name as it will be displayed in the Chronon Recording Server UI)\\
\bxshell{autostartwithconfig} (if you want the recording to start automatically (recommended) then the recording parameters need to be set in the Chronon application.)\\ 
\bxshell{port} (optional -- if none is entered, 8042 is used.). \\
The documentation for the configuration file is available on the Chronon site:\\
\href{http://chronon.onconfluence.com/\\
display/DOC/Recorder+Configuration+File}{http://chronon.onconfluence.com/\\
display/DOC/Recorder+Configuration+File}}
\end{description}

\bxtipp{You can also refer to the Chronon documentation for information on the required parameters to run your tests with the Recording Server. :\\
\href{http://chronon.onconfluence.com/display/\\
DOC/Recording+outside+Eclipse}{http://chronon.onconfluence.com\\
/display/DOC/Recording+outside+Eclipse}\\}

\subsubsection{Adapting tests to improve data collection}
You should bear the following in mind when using Chronon for recording information in automated tests.\\

\textbf{Performance in the \gdaut{} may be affected}
\begin{itemize}
\item The recordings that Chronon performs are very memory-intensive. For this reason, you may notice performance differences in your \gdaut{}. It may also be necessary to increase the step delay for your tests, and / or add extra synchronization to compensate for the performance differences when Chronon is running. 
\item For these reasons, we do not recommend having Chronon configured as a part of your standard \gdaut{} configuration. Instead, we suggest running tests with Chronon monitoring when needed. 
\item We also strongly suggest ensuring that your \gdaut{} and the machine it is running on have sufficient memory to cope with the increased monitoring activity.
\end{itemize}

\textbf{Stopping or restarting the \gdaut{} will cause the previously recorded information to be lost}
\begin{itemize}
\item The recording files are written when the \gdaut{} is stopped. This means either stopping the \gdaut{} by hand, using the \bxcaption{Stop AUT} button in on the toolbar in the \ite{} or when you use the \bxname{restart} action. 
\item Because of this, we recommend executing individual \gdcases{} (use cases) in \gdsuites{} that you want to analyze with Chronon. You should ensure that any \gdehandlers{} in the \gdsuite{} will not cause the \gdaut{} to restart.
\end{itemize}




\subsubsection{Analyzing the generated reports}
\label{TasksChrononTools}
To analyze the reports generated, you will require the Chronon Time Travelling Debugger from Chronon Systems. You can download a trial version of this tool. The link to the trial version is provided in the expert \gdaut{} configuration.

Open source projects may contact Chronon Systems for free licenses. 


