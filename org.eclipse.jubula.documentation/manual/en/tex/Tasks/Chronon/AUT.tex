You can configure your \gdaut{} to run with Chronon recording so that you can analyze any errors that occur during your automated test runs. 

\subsubsection{Adding Chronon information in the \gdaut{} configuration}
\gdhelpid{autConfigSettingWizardPagePageContextId}{Configuring an AUT}
You can select Chronon as a monitoring agent and configure it in the Expert Settings in the \gdaut{} configuration. 

\begin{enumerate}
\item Open the \gdaut{} configuration dialog from the \gdproject{} properties \bxpref{projectproperties}.
\item Select the \bxname{Expert} configuration. 
\item Select Chronon as the monitoring agent. 
\item You can then enter the configuration details for the monitoring. 
\begin{description}
\item [Package patterns:]{Enter a comma-separated list of packages that you want to be covered by the monitoring.The patterns must be  valid regular expressions. If you enter no patterns, no recording file will be written.}
\item [Target directory:]{Enter the path to a directory where the recording files should be written. Any non-existing folders will be automatically created. This target directory and all items it contains are deleted when the \gdaut{} is started or restarted. Once the \gdaut{} is started, the target directory is recreated ready for the next run. This means you must ensure you have copied any relevant files before starting or restarting the \gdaut{}. When the \gdaut{} is stopped after a test, the recording file is written to this target directory. The target directory can be relative to the \gdaut{} working directory, if you have entered one.}
\item [Operating System:]{Enter \bxshell{win,} \bxshell{lin}, \bxshell{sol} or \bxshell{mac} (without quotes) to specify your operating system. }
\item[JRE Architecture:]{Enter \bxshell{32} or \bxshell{64} (without quotes) to specify your architecture. If you have entered \bxshell{mac} as an operating system, you can only enter \bxshell{64}.}
\end{description}
\end{enumerate}

\subsubsection{Adapting tests to improve data collection}
You should bear the following in mind when using Chronon for recording information in automated tests.
\textbf{Performance in the \gdaut{} may be affected}
\begin{itemize}
\item The recordings that Chronon performs are very memory-intensive. For this reason, you may notice performance differences in your \gdaut{}. It may also be necessary to increase the step delay for your tests, and / or add extra synchronization to compensate for the performance differences when Chronon is running. 
\item For these reasons, we do not recommend having Chronon configured as a part of your standard \gdaut{} configuration. Instead, we suggest running tests with Chronon monitoring when needed. 
\end{itemize}

\textbf{Stopping or restarting the \gdaut{} will cause the previously recorded information to be lost}
\begin{itemize}
\item The recording files are written when the \gdaut{} is stopped. This means either stopping the \gdaut{} by hand, using the \bxcaption{Stop AUT} button in \app{} or when you use the \bxname{restart} action. 
\item Because of this, we recommend executing individual \gdcases{} (use cases) in \gdsuites{} that you want to analyze with Chronon. You should ensure that any \gdehandlers{} in the \gdsuite{} will not cause the \gdaut{} to restart.
\end{itemize}

\subsubsection{Analysing the generated reports}
To analyze the reports generated, you will require the Chronon Time Travelling Debugger from Chronon Systems. 


