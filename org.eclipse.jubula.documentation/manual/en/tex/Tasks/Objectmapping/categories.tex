% BREDEX LaTeX Template
%  \documentclass is either ``bxreport'' or ``bxarticle''
%                 option is bxpaper
%% \documentclass{bxarticle}
%% % ----------------------------------------------------------------------
%% \begin{document}
%% \title{}
%% \author{}
%% % \author*{Hauptautor}{Liste der Nebenautoren}
%% \maketitle
%% % ----------------------------------------------------------------------
%% \bxversion{0.1}
%% %\bxdocinfo{STATUS}{freigegeben durch}{freigegeben am}{Verteilerliste}
%% \bxdocinfo{DRAFT}{}{}{}
%% % ----------------------------------------------------------------------

%% \end{document}
\textbf{Default categories in the \gdomeditor{}}\\
\index{Object Mapping!Unassigned component names}
\index{Object Mapping!Unassigned technical names}
\index{Object Mapping!Assigned names}
\index{Assigned names}
\index{Unassigned names!Technical}
\index{Unassigned names!Component}
\label{TasksOMDefaultCats}
The \gdomeditor{} tree view displays the following categories by default:
\begin{description}
\item [Unassigned component names:]{these are the names you have used in your \gdcases{} or component names that you have created \bxpref{TasksCreateNewCompName}. They are unassigned because they have not yet been mapped to a technical name. }
\item [Unassigned technical names:]{these are the names that you have collected from the \gdaut{} \bxpref{TasksOMCollect}, but not yet assigned to component names. }
\item [Assigned names:]{there are pairs of names that have been mapped to each other. Each technical name can be mapped to one or more component names. This mapping tells \app{} which actual components you are referring to in your \gdcases{}.}
\end{description}

\textbf{Creating categories in the \gdomeditor{}}\\
\index{Object Mapping!Categories}
\index{Categories!Object Mapping}
\label{omcats}
We recommend creating categories in the \gdomeditor{} to make your mapping work easier \bxpref{BPOMCategories}. 
\begin{itemize}
\item You can create categories and subcategories in the \gdomeditor{} (tree view) by:
\begin{enumerate}
\item Selecting the category you want your new category to appear in (e.g. in \bxcaption{assigned names}).
\item Selecting \bxcaption{create category} from the context-sensitive menu. 
\item Entering a name for the category
\bxwarn{You can't have two categories with the same name at the same level.}
\end{enumerate}
\item When you are mapping, you can choose which category to map into. See the next section \bxpref{mapcat} for details.
\end{itemize}

\bxtipp{It is a good idea to create categories in the \gdomeditor{}. See the section on best practices \bxpref{BPOMCategories} for more details.}

\textbf{Mapping into categories in the \gdomeditor{}}\\
\index{Object Mapping!in categories}
\label{mapcat}
Once you have created categories in the \gdomeditor{}, you can choose to map technical names collected from the \gdaut{} directly into a category. This can help if you have created a category for each dialog/window, and you want to map all of the components from it into one category.  

\begin{enumerate}
\item When you are in the \gdomm{}, right-click onthe category you want to map into and select:\\
\bxmenu{Map components into this category}{}{}\\ 
to make the technical names you collect from the \gdaut{} appear in this subcategory. 
\bxtipp{If you have already mapped the technical name, the name will be shown in the \gdomeditor{}, but not moved into the category.}
\bxtipp{The status bar displays which category you are mapping into.}
\end{enumerate}
%pictures
