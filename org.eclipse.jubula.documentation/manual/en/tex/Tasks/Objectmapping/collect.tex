\index{Object Mapping!Collect components}

Once your test specification is ready, you can  \bxname{collect} components (technical names) from the \gdaut{} to \bxname{map} (assign) to the component names you used in your tests. 

To collect components from the \gdaut{}, you must have: 

\begin{itemize}
\item started and connected to the \gdagent{} \bxpref{Agent}
\item created a \gdcase{} containing \gdsteps{} \bxpref{TasksCreateTC}
\item added \gdcases to a \gdsuite{} \bxpref{WorkingWithSuites}
\item specified an \gdaut{} \bxpref{Defineaut} 
\item configured an \gdaut{} \bxpref{configuringaut} if you want to start the \gdaut{} via \app{}
\item chosen an \gdaut{} for the \gdsuite{} \bxpref{confsuite}
\item checked that your current working language is supported by the chosen \gdaut{}. 
\item started the \gdaut{} to be mapped \bxpref{startaut}
\item opened the object mapping editor \bxpref{TasksOpenOME}
\end{itemize}


\begin{enumerate}
\item Start the \gdomm{} by clicking the arrow next to the\bxname{Start \gdomm{}} on the toolbar and selecting which \gdaut{} (based on the \gdaut{} ID) you want to map. \gdmarpar{../../../share/PS/startOM}{start \gdomm}

\bxtipp{If you have the same \gdaut{} running more than once, you will only be able to collect components from the \gdaut{} whose ID you chose. The object map for \gdauts{} that are the same is, however, identical.}

The status bar will show that the \gdomm{} is active. 
\item Bring the \gdaut{} into focus by clicking on its titlebar. 

\textbf{For Java \gdauts{}:}
\begin{itemize}
\item In the \gdaut{} for which the \gdomm{} was started, move the cursor over components. They will be highlighted with a green border (see \bxfigref{greenborders}). 
\item To collect a technical name for a component, hover the cursor over the component whose name you want to collect. 
\item Press \bxkey{Ctrl+Shift+Q}.
\bxtipp{You can change the key combination for the object mapping in the object mapping preferences \bxpref{TasksPrefsOMM}. This is a good idea if the current key combination has a specific meaning in your \gdaut{}. You can also set the object mapping combination to a mouse click if your \gdaut{} does not accept key combinations.}
\item If no component is collected, then you may need to extend \app{} to recognize and test the component. More information on extending \app{} is available in the Extension Manual.
\end{itemize}

\textbf{For HTML \gdauts{}:}
\begin{itemize}
\item While the \gdomm{} is active, the \gdaut{} cannot be used. 
\item To collect a technical name for a component, click the component whose name you want to collect. 
\end{itemize}

\item In the \gdomeditor{}, the technical name for this component will appear in the \bxname{unassigned technical names} category. 
\bxtipp{If you have already mapped this component, it will be highlighted in the \gdomeditor{}.}
\item When you collect a technical name, it is displayed with a colored dot in the \gdomeditor{}. The color of the dot indicates the strength of the component recognition for this component, \bxname{in the current state of the \gdaut{}} \bxpref{TasksOMStateColor}.
\item Collect all the names you need from the \gdaut{} and then click the \bxname{Stop \gdomm{}} button on the toolbar. 
\gdmarpar{../../../share/PS/stopOM}{ stop \gdomm}
\item You can now map (assign) the component names you used in your \gdcases{} to the technical names you have collected from the \gdaut{} \bxpref{TasksOMMap}.
\end{enumerate}

\textbf{For other \gdauts{}}
Specifics on object mapping for other \gdauts{} can be found in the relevant section in the chapter on toolkit-specific information \bxpref{Toolkit}. 

\subsubsection{Understanding the colored dots when collecting component names in the \gdomeditor{}}
\gdhelpid{objectMapEditorContextId}{Object Mapping}
\label{TasksOMStateColor}

When technical names are collected from the \gdaut{}, they appear in the \gdomeditor{} with a colored dot. The color of the dot corresponds to the strength of the component recognition for this component \bxname{at the time of collecting}. 

\begin{description}
\item [A green dot]{signifies that the component can be found as an exact match, and that only this component was above the threshold \bxpref{TasksOMConfiguration} (i.e. only this component was considered as possible.}
\item [A yellow dot]{means that the component can be found as an exact match, but that other components were also above the threshold, i.e. this was not the only component considered possible.}
\item [A red dot]{means that the component can not currently be found if a test is executed. The recognition value for the component is below the current threshold.}
\end{description}

You can use this information to identify components that will not be recognized in the current state of the \gdaut{} before running the test. 

\bxwarn{The colored dot disappears after saving. It is not a measurement of the component state over time, but only at the moment when the component was collected.}
