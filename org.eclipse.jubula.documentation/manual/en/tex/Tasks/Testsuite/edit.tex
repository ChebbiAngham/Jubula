\subsubsection{Configuring \gdsuites{} in the \gdpropview{}}
\index{Test Suite!Configuration}
\index{Configuration!Test Suite}
\index{Event Handler!Error Types}
\index{Error Types}
\index{Action Error}
\index{Component not found}
\index{Check failed}
\index{Configuration error}
\index{Reentry Properties}
\index{Event Handler!Reentry Properties}
\index{Break}
\index{Continue}
\index{Return}
\index{Pause}
\index{Exit}
\index{Default Event Handler}
\index{Event Handler!Default}
\index{AUT ID}
\label{confsuite}

To configure a \gdsuite{}, you must first create one \bxpref{TSeditor}.

\bxtipp{If you are editing the \gdsuite{} configuration, the working language (which is specified via the globe button on the toolbar) must be set to a language supported by the chosen \gdaut{} for the \gdsuite{}. Otherwise the \gdsuite{} will be uneditable. }

\begin{enumerate}
\item Open the \gdtestsuiteeditor{} by double-clicking on the \gdsuite{} you want to configure. 
\item In the \gdpropview{}, you can:
\begin{enumerate}
\item Change the \gdsuite{} name by entering a new name in the \bxname{\gdsuite{} name} field.  
\item Add a comment to the \gdsuite{}. 
\item Enter a value in the \bxname{step delay} field. 

\bxtipp{The step delay is the time \app{} leaves between each \gdstep{} during test execution. The default is 0 milliseconds. }

\item Select the \gdaut{} for this \gdsuite{}. To be able to select an \gdaut{} (and object map, and execute your test) you must have added at least one \gdaut{} to the \gdproject{} \bxpref{Defineaut}.



\bxtipp{You don't have to choose an \gdaut{} for a \gdsuite{} as soon as you have created it, but you will have to choose one before object mapping, for example.}

\item Choose a default reentry type for each of the four error types in \app{} from the combo-boxes. 

\gdehandlers{} are \gdcases{} used to deal with errors during test execution. When an error occurs, the current \gdcase{} is searched for an \gdehandler{} for that error type. If none is found, the parent \gdcase{} is searched, and so on. If no \gdehandler{} for the test is found, then a default \gdehandler{} (specified in the \gdsuite{} properties)is activated.  

See the sections on \gdehandlers{} for information on the event types \bxpref{eventtype} and reentry types \bxpref{reentrytype}.
\item Save the changes in the \gdtestsuiteeditor{}.
\end{enumerate}
\end{enumerate}


\subsubsection{Adding \gdcases{} to a \gdsuite{}} 
\label{addtestsuite}
\index{Test Suite!Add Test Case}
To add \gdcases{} to a \gdsuite{}, you must first create one \bxpref{TSeditor}.
\bxtipp{Look at the section on Keyword Design \bxpref{KeywordDesign} for useful information on how to structure your \gdsuites{}.}
\begin{enumerate}
\item Open the \gdtestsuiteeditor{} by double-clicking on the \gdsuite{} you want to edit.
\item Select a \gdcase{} to add from the context-sensitive menu by selecting:
\bxmenu{Reference Existing \gdcase{}}{}{}\\

\bxtipp{You can also add \gdcases{} to a \gdsuite{} in the \gdtestsuiteeditor{}
using drag-and-drop or by pressing \bxkey{ENTER} on a selected node in the \gdtestsuiteeditor{}.}

  \item Choose a \gdcase{} or \gdcases{} to add from the dialog which appears.
  \item Click \bxcaption{OK}. 
  \item The \gdcase{} or \gdcases{} you selected appear(s) in the \gdtestsuiteeditor{}. They are marked with a small arrow to show that they are reused here. The name of the \gdcase{} is contained in angled brackets (\bxshell{< >}) to show that it is the same name that you used when you specified the \gdcase{}. 

\item Save the changes in the editor.
\bxtipp{Once you have added \gdcases{} to a \gdsuite{}, the next step is to do the object mapping \bxpref{objectmappingtasks}.}
\end{enumerate}

\subsubsection{Deleting \gdcases{} from a \gdsuite{}}
\index{Delete!Test Cases}
\index{Test Case!Delete}
\label{TasksDeleteTCsFromSuite}
\begin{enumerate}
\item Open the \gdtestsuiteeditor{} by double-clicking on the \gdsuite{} you want to edit.
\item  Single-click the \gdcase{} you want to delete. Use  \bxkey{Ctrl} to select multiple \gdcases{}. 
\item Select \bxcaption{delete} from the context-sensitive menu to delete the \gdcases{}.
\item \gdcases{} can also be deleted using \bxkey{DELETE}.
\item Confirm that you want to delete the \gdcases{} in the dialog which appears.
\item Save the changes in the \gdtestsuiteeditor{}. 

\end{enumerate}

\subsubsection{Renaming \gdsuites{}}
\gdhelpid{dialogRenameContextId}{Rename Dialog}
\index{Rename!Test Suites}
\index{Test Suite!Renaming}
\label{tasksrenametsuite}
\begin{enumerate}
\item \gdsuites{} can be renamed in one of two ways:
\begin{enumerate}
\item Select the \gdsuite{} you want to rename in the \gdtestsuitebrowser{}. Select:\\
\bxmenu{Rename}{}{}\\
from the context-sensitive menu. In the dialog which appears, enter a new name and click \bxcaption{OK}. 
\item Open the \gdtestsuiteeditor{} by double-clicking on the \gdsuite{} you want to edit in the \gdtestsuitebrowser{}.  In the \gdpropview{}, change the \gdsuite{} name. Save the changes in the editor.  
\end{enumerate}
\end{enumerate}
