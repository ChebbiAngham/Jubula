\index{Test Suite!Configuration}
\index{Configuration!Test Suite}
\index{Event Handler!Error Types}
\index{Error Types}
\index{Action Error}
\index{Component not found}
\index{Check failed}
\index{Configuration error}
\index{Reentry Properties}
\index{Event Handler!Reentry Properties}
\index{Break}
\index{Continue}
\index{Return}
\index{Pause}
\index{Exit}
\index{Default Event Handler}
\index{Event Handler!Default}
\index{AUT ID}
\label{confsuite}

To configure a \gdsuite{}, you must first create one \bxpref{TSeditor}.

\bxtipp{If you are editing the \gdsuite{}, the working language (which is specified via the globe button on the toolbar) must be set to a language supported by the chosen \gdaut{} for the \gdsuite{}. Otherwise the \gdsuite{} will be uneditable. }

\begin{enumerate}
\item Open the \gdtestsuiteeditor{} by double-clicking on the \gdsuite{} you want to configure. 
\item In the \gdpropview{}, you can:
\begin{enumerate}
\item Change the \gdsuite{} name by entering a new name in the \bxname{\gdsuite{} name} field.  
\item Add a comment to the \gdsuite{} \bxpref{TasksEditorAddComment}. 
\item Enter a value in the \bxname{step delay} field. 

\bxtipp{The step delay is the time \app{} leaves between each \gdstep{} during test execution. The default is 0 milliseconds. You can also increase the speed of test execution by altering your system settings for double click speed \bxpref{TasksExecSpeed}.}

\item Select the \gdaut{} for this \gdsuite{}. To be able to select an \gdaut{} (and object map, and execute your test) you must have added at least one \gdaut{} to the \gdproject{} \bxpref{Defineaut}.



\bxtipp{You don't have to choose an \gdaut{} for a \gdsuite{} as soon as you have created it, but you will have to choose one before object mapping, for example.}

\item Choose a default reentry type for each of the four error types in \app{} from the combo-boxes. 

\gdehandlers{} are \gdcases{} used to deal with errors during test execution. When an error occurs, the current \gdcase{} is searched for an \gdehandler{} for that error type. If none is found, the parent \gdcase{} is searched, and so on. If no \gdehandler{} for the test is found, then a default \gdehandler{} (specified in the \gdsuite{} properties)is activated.  

As a general rule, you should avoid default \gdehandlers{} being executed.
See the sections on \gdehandlers{} for information on the event types \bxpref{eventtype}, reentry types \bxpref{reentrytype} and creating your own \gdehandlers{}.
\item Save the changes in the \gdtestsuiteeditor{}.
\end{enumerate}
\end{enumerate}
