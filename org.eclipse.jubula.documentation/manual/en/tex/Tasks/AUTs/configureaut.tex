% BREDEX LaTeX Template
%  \documentclass is either ``bxreport'' or ``bxarticle''
%% %                 option is bxpaper
%% \documentclass{bxarticle}
%% % ----------------------------------------------------------------------
%% \begin{document}
%% \title{}
%% \author{}
%% % \author*{Hauptautor}{Liste der Nebenautoren}
%% \maketitle
%% % ----------------------------------------------------------------------
%% \bxversion{0.1}
%% %\bxdocinfo{STATUS}{freigegeben durch}{freigegeben am}{Verteilerliste}
%% \bxdocinfo{DRAFT}{}{}{}
%% % ----------------------------------------------------------------------

%% \end{document}

\index{Add!AUT Configuration}
\index{AUT Configuration!Add}
\index{Edit!AUT Configuration}
\index{AUT Configuration!Edit}
\index{AUT!Configuration}
\index{Configuration!AUT}
\index{Activation}
\index{Application activation}

Once you have created a \gdproject{} \bxpref{newproject} and defined an \gdaut{} \bxpref{Defineaut}, you can add and edit \gdaut{} configurations. 

The details in the \gdaut{} configuration provide information on how to start the \gdaut{}, e.g. on which machine. 

An \gdaut{} can have multiple configurations (for example, for local and remote testing). A configuration contains all the information required to start the \gdaut{}, and may contain platform- or installation-specific information such as paths to working directories, \gdaut{} arguments, Java versions, browser choices and activation methods. 

\bxtipp{If you want to start your Java \gdaut{} yourself, and have the \ite{} connect to it, then use the \bxname{autrun} command to start the \gdaut{} \bxpref{autrun}. In this case, you do not need to create an \gdaut{} configuration.}

\subsubsection{\gdaut{} activation}
\label{TasksAUTActivation}

Activation makes sure that the  \gdaut{} is in focus at the beginning of test execution. This is acheived by clicking somewhere in the \gdaut{} window. You can specify the activation method (i.e. where to click) as part of a configuration for an \gdaut{}, or you can create a \gdstep{} within a test to do the same thing \bxextref{\gdrefman}{ref,activate}. 

The advantage of specifying an activation method here is that it is central and affects each test execution started on this \gdaut{} with this configuration. 

Bear in mind that you may need to activate your \gdaut{} in order for tests to work, especially if the \gdaut{} runs on the same machine as the \ite{}. 
