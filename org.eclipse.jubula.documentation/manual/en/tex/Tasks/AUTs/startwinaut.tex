\index{Add!AUT Configuration}
\index{AUT Configuration!Add}
\index{Edit!AUT Configuration}
\index{AUT Configuration!Edit}
\index{AUT!Configuration}
\index{Configuration!AUT}
\index{Activation}
\index{Application activation}
\index{AUT ID}

When starting Windows \gdauts{},  you must specify whether you are testing a desktop Windows \gdaut{} or an \gdaut{} that is designed to run on a modernUI desktop (i.e. a Windows Store App). 

\subsubsection{\gdaut{} configuration for Windows desktop \gdauts{}}
To be able to start a Windows desktop \gdaut{}, you should select the \bxname{win} toolkit as the \gdaut{} toolkit in the \gdaut{} definition dialog. 

You must enter the following information to be able to start a \bxname{win} \gdaut{}:
\begin{enumerate}
\item Enter the basic configuration details as described earlier \bxpref{TasksBasicConfigurationInfo}.
\item Enter the executable file name in the \bxname{Executable File Name} field. This path can be relative if you define a working directory \bxpref{TasksWorkingDir}).
\item Enter any necessary command-line arguments for the \gdaut{} in the
 \bxname{AUT Arguments} field. 
  
\end{enumerate}

\subsubsection{\gdaut{} configuration for Windows Store \gdauts{}}
To be able to start a Windows Store \gdaut{}, you should select the \bxname{winApps} toolkit as the \gdaut{} toolkit in the \gdaut{} definition dialog. 

You must enter the following information to be able to start a \bxname{winApps} \gdaut{}:
\begin{enumerate}
\item Enter the basic configuration details as described earlier \bxpref{TasksBasicConfigurationInfo}.
\item Enter the name of the app. This is the name that is shown when the \gdaut{} is installed. This name is locale-dependent -- you will have to have separate \gdaut{} configurations to test your app in different languages. 
\item Enter any necessary command-line arguments for the \gdaut{} in the
 \bxname{AUT Arguments} field. 
  
\end{enumerate}
