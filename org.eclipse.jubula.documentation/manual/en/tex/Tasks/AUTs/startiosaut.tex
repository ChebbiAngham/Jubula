\index{AUT Agent!iOS}
\index{iOS!AUT Agent}
\index{AUT Configuration!iOS}
\index{iOS!AUT Configuration}
\index{Start!iOS}
\index{iOS!Start AUT}


\subsubsection{Connecting to the \gdagent{}}
The \gdagent{} does not need to be started on the simulator or device for testing iOS \gdauts{}. It does, however, need to be started to ensure that the communication can take place.  The actual communication with the simulator or the device is accomplished using a port that is defined in the test \gdaut{} \bxpref{ToolkitiOSSetup} and configured in the \gdaut{} configuration \bxpref{TasksiOSAUTConfig}.

\bxtipp{Since the place where the \gdagent{} is started is not important, we recommend starting it on localhost.}

\subsubsection{Configuring an iOS \gdaut{}}
\label{TasksiOSAUTConfig}
\begin{itemize}
\item Enter the basic configuration details as described earlier \bxpref{TasksBasicConfigurationInfo}.
\item The working directory currently has no effect.
\item Enter the iOS Device Host: this is the address of the device or simulator in the network that the \gdaut{} will run on. The host will either be a hostname or an IP address. 
\item Enter the iOS Device Port: this is the port number that is available for communication between the \gdaut{} and the \ite{}. It is defined as a part of the setup of the \gdaut{}. If no port number has been specified in the \gdaut{}, the default of 11022 will be used, and you should enter this number. 
\end{itemize}
\bxwarn{Starting iOS \gdauts{} with \bxname{autrun} is not supported.}

\subsubsection{Starting and connecting to iOS \gdauts{}}
Unlike other \gdauts{}, iOS \gdauts{} are not \textit{started} via the \ite{}, nor by the testexec. Instead, the \gdaut{} must have been made testable \bxpref{ToolkitiOSSetup}, had component naming added \bxpref{ToolkitiOSDFT} and also be started on the simulator or device that it will be tested on. 
Once these prerequisites have been completed, you can connect to the \gdaut{} from the \ite{} by selecting the \gdaut{} from the list in the \bxcaption{Start \gdaut{}} button on the toolbar. This does not start the \gdaut{} but connects to it. 
