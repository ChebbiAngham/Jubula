
\index{Add!AUT Configuration}
\index{AUT Configuration!Add}
\index{Edit!AUT Configuration}
\index{AUT Configuration!Edit}
\index{AUT!Configuration}
\index{Configuration!AUT}
\index{Activation}
\index{Application activation}
\index{ID attribute name}
\index{AUT ID}

The \gdaut{} configuration dialog for HTML and RAP has three different levels of detail: basic, advanced and expert. 

See the sections below for information on the different levels. 

\subsubsection{Basic HTML and RAP \gdaut{} configuration}

Use the basic setting to specify the URL and Browser you wish to start this \gdaut{} configuration on. 

\begin{enumerate}
\item A suggested name for this \gdaut{} configuration is generated automatically based on your \gdaut{} name and the default \gdagent{} host. You can change this name if you wish. 
\item The default \gdagent host is also automatically selected. You can select a new \gdagent{} from the combo box or add a new one by clicking \bxcaption{New}. For more information on adding and editing an \gdagent{}, see the section later \bxpref{TasksPrefsAgent}.
\item Enter the \gdaut{} ID that will be given to this \gdaut{} when it is started.  
\item You can optionally create a working directory to store files in. The server directory of the \app{} installation is selected as default. To create a working directory elsewhere, browse to and select the location.  For more information on relative paths, read the section in the reference manual \bxextref{\gdrefman}{ref,relativepath}. 
\item Enter the URL of your \gdaut{}.
\bxwarn{Relative paths to the URL cannot be used!}
\item Select the browser you want to start the \gdaut{} in.
\end{enumerate}
For information on the advanced properties for the \gdaut{} configuration, see the next section \bxpref{AdvancedAUTConfigHTML}. 

\subsubsection{Advanced HTML and RAP \gdaut{} configuration}
\label{AdvancedAUTConfigHTML}

You can use the advanced dialog to enter the browser path for your browser. This lets you use a specific version of the browser (not available for Internet Explorer). 

For information on the expert properties for the HTML and RAP \gdaut{} configuration, see the next section \bxpref{ExpertAUTConfigHTML}. 

\subsubsection{Expert HTML and RAP \gdaut{} configuration}
\label{ExpertAUTConfigHTML}

You can use the expert dialog to enter an \bxname{ID attribute name} \bxpref{DFTWeb}. If you have used a specific tag to name components in your application, enter the tag in the Expert Configuration area. \app{} will then use this information instead of the \bxname{name} attribute in the object recognition. 

You can also select an activation method for your \gdaut{}. See the section on \gdaut{} activation \bxpref{TasksAUTActivation} for more details. 




