%% % ----------------------------------------------------------------------
%% \bxversion{0.1}
%% %\bxdocinfo{STATUS}{freigegeben durch}{freigegeben am}{Verteilerliste}
%% \bxdocinfo{DRAFT}{}{}{}
%% % ----------------------------------------------------------------------

%% \end{document}

\index{autrun command!Define AUT}
\index{Define AUT!autrun command}

The \bxname{autrun} command can be used as an alternative to starting your \gdaut{} via \app{} (i.e. with an \gdaut{} configuration \bxpref{configuringaut}). It can only be used if your \gdaut{} can be started by an executable file (e.g. .bat, .exe, .cmd, .sh etc.) and if it is written in Java 1.5 or above, and you are using a Sun VM. 


\bxwarn{The \bxname{autrun} command cannot be used for HTML or pure SWT \gdauts{}.}

The command allows you to start your \gdaut{} independently of \app{}, on a machine where the \gdagent{} is running. The \app{} \ite{}, when connected to this \gdagent{} will then recognize the running \gdaut{} as a testable application. 

To use the \bxname{autrun} command:
\begin{enumerate}
\item Ensure that the \gdagent{} is installed on the machine where you will be starting the \gdaut{}. 
\item Start your \gdaut{} via the command line by entering \bxshell{autrun.exe} then the following parameters: 
\begin{table}[h]
\label{autrunparams}
	\centering
	\begin{tabular}{|l|l|}

	\hline
	\textbf{Detail}&\textbf{Parameter}%&\textbf{Example}
\\
		\hline
                -h 
                &\bxshell{-h}\\
                & Gives parameter help\\
                \hline
                -w, \verb+--+workingdir
                  & \bxshell{-w <directory>}\\
		  &Enter the working directory for the \gdaut{}\\
                  \hline
                  -a, \verb+--+autagenthost
                  & \bxshell{-a <hostname>}\\
		  &Enter the hostname for the \gdagent{}\\
                  \hline
                  -p, \verb+--+autagentport
                  & \bxshell{-p <port number>}\\
		  &Enter the port number for the \gdagent{}\\
                  \hline
                  -swing
                  & \\
		  &If the \gdaut{} is a Swing \gdaut{}\\
                  \hline
                  -rcp
                  & \\
		  &If the \gdaut{} is an RCP \gdaut{}\\
                  \hline
                  -swt
                  & \\
		  &If the \gdaut{} is an SWT \gdaut{}\\
                  \hline
                  -k, \verb+--+kblayout
                  & \bxshell{-k <en\_US>}\\
		  &Enter the keyboard layout for SWT/RCP \gdauts{}\\
                  \hline
                  -i, \verb+--+autid
                  & \bxshell{-i <ID>}\\
		  &Enter the ID to give to this \gdaut{}\\
                  \hline
                  -e, \verb+--+exec
                  & \bxshell{-e <AUT.exe>}\\
		  &Enter the executable file for the \gdaut{}\\
                  \hline
                  -g, \verb+--+generatenames (optional)
                  & \bxshell{-g <true/false>}\\
		  &For RCP \gdauts{}, enter whether \\& to generate technical names.                  \bxpref{Defineaut}\\
                  \hline
                 
	\end{tabular}
	\caption{Parameters for autrun command}
\end{table}
\end{enumerate}

\subsubsection{Creating an \gdaut{} definition from a running \gdaut{}}
\label{createAUTDef}
Once you have started an \gdaut{} using the \bxname{autrun} command, you can automatically generate an \gdaut{} definition \bxpref{Defineaut} for this \gdaut{}:

\begin{itemize}
\item In the \gdrunautview{}, select the \gdaut{} you want to define (it will be marked as an unknown \gdaut{} ID).
\item Select:\\
 \bxmenu{Create AUT Definition}{}{}\\
from the context menu.
\item The \gdaut{} definition window will appear. Complete the dialog \bxpref{Defineaut} and click \bxcaption{OK}.
\end{itemize}
