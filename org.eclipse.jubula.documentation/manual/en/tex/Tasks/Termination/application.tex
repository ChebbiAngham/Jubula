%% % $Id: application.tex 11355 2010-06-16 14:48:31Z alexandra $
%% % Local Variables:
%% % ispell-check-comments: nil
%% % Local IspellDict: american
%% % End:
%% % --------------------------------------------------------
%% % User documentation
%% % copyright by BREDEX GmbH 2005
%% % --------------------------------------------------------
%% % this command can be inserted multiple times
%% \gdhelpid{}
%% % 
%% \begin{bxdescription}
%% \end{bxdescription}
%% %
%% \begin{bxsteps}
%% % use the \item command for single steps
%% \end{bxsteps}
%% % change <FILE> to the same filename you are editing
%% \bxinput{Links/<FILE>}
%% %
%% % other usefull commands are
%% %   \bxhint{}        to create a hint
%% %   \bxwarn{}        to describe a warning

\index{Stop!AUT Agent}
\index{AUT Agent!Stop}

Once the \gdsuite{}, \gdaut{} and \gdagent{}  have been stopped or disconnected, 
make sure any changes are saved and select: \\ \bxmenu{File}{Exit}{}. 

\subsection{Stopping the \gdagent}
\gdhelpid{problemViewContextId}{Problem View}
\begin{itemize}
\item Stop the \gdagent{} by selecting: \\ \bxmenu{Start}{Programs}{\app{}} \\
and then \bxcaption{Stop \gdagent{}} 
from the Windows Menu. 
\item Under Linux, use  the script:\\
\bxshell{stopautagent}. 
\bxtipp{You can also stop the \gdagent{} from the system tray.}
\item You can use the parameter \bxshell{-p <port number>} to specify which port number should be stopped. 
\item You can also enter the hostname directly after the \bxshell{stopautagent} command to specify which hostname should be stopped.
\bxtipp{Stopping the \gdagent{} will stop any \gdauts{} that are connected to this \gdagent{}.}
\end{itemize}

