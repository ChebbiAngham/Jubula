You can create and manage central test data sets for a \gdproject{} which can be reused in \gdcases{}. 

\subsubsection{Creating and editing central test data sets}
\gdhelpid{guidancerCentralTestDataEditorContextId}{Central Test Data}
\gdhelpid{newTestDataCubeNameContextId}{New Central Data Set}
\label{TasksCentralDataCreate}

To create a central test data set:
\begin{enumerate}
\item Open the  \gddataeditor{} by clicking the \bxcaption{Central Test Data Editor} button on the toolbar or selecting:\\
\bxmenu{Open with}{Central Test Data Editor}{}\\
from the \gdtestsuitebrowser{}.
\item In the \gddataeditor{}, select:\\
\bxmenu{Add new Data Set}{}{}\\
from the context-sensitive menu or press \bxkey{Insert}.
\item In the dialog that appears, enter a name for the new data set and click \bxcaption{OK}.
\item The new data set appears in the \gddataeditor{}. You can now add parameters to the data set \bxpref{TasksCentralDataEditParams}.
\item You can rename the data set by pressing\bxkey{F2} or selecting:\\
\bxmenu{Rename}{}{}\\
from the context sensitive menu.
\item You can categorize your central data sets using the \bxname{Add category} function from the context-sensitive menu \bxpref{WorkingWithCategories}.
\end{enumerate}

\subsubsection{Deleting central test data sets}
\gdhelpid{guidancerCentralTestDataEditorContextId}{Central Test Data}
You can delete a central test data set if the data set has not yet been reused (referenced) in a \gdcase{} \bxpref{TasksCentralDataReference}. 
\begin{enumerate}
\item Select:\\
\bxmenu{Delete}{}{}\\
from the context-sensitive menu or press \bxkey{DELETE}.

\item A dialog will appear if the data set has been reused and cannot be deleted.
\item You can use the search \bxpref{TasksShowWhereUsedDataSet} to show where the data set has been used.
\end{enumerate}

\subsubsection{Adding and modifying parameters for central test data sets}
\gdhelpid{guidancerCentralTestDataEditorContextId}{Central Test Data}
\label{TasksCentralDataEditParams}

Once you have created a central test data set \bxpref{TasksCentralDataCreate}, you can add parameters to the data set using the \bxname{Edit Parameters} dialog. 
\begin{enumerate}
\item Open the edit parameters dialog for the central data set by double-clicking on the data set in the \gddataeditor{}. You can also select \bxmenu{Edit parameters}{}{} from the context-sensitive menu. 
\item In the \bxname{Edit Parameters} dialog, you can see any parameters you have already added for this data set, and what type of parameters they are. 
\item Use the \bxcaption{Add} button to create a new parameter for the data set. 
\item Enter a name for the parameter and select the type of parameter it should be (e.g. String, Integer, ...). The type of parameter it should be will depend on which actions are using it. A list of actions and their parameters (and types) is available in the reference manual (\bxextref{\gdrefman}{ref,actparam}).
\bxtipp{Names for referenced parameters may only consist of letters, numbers and underscores. You cannot use spaces.}

\item You can also change the order the parameters appear in, edit their types and names, and delete them completely using this dialog. 

\end{enumerate}


\subsubsection{Entering data for central test data sets}
\gdhelpid{guidancerCentralTestDataEditorContextId}{Central Test Data}
\label{TasksDSVCentral}

Once you have created a central test data set \bxpref{TasksCentralDataCreate} and have added parameters to the central test data set \bxpref{TasksCentralDataEditParams} then you can enter data for these parameters in the \gddatasetsview{}.

To enter data sets for a central test data set:
\begin{enumerate}
\item Open the  \gddataeditor{} by clicking the \bxcaption{Central Test Data Editor} on the toolbar or selecting:\\
\bxmenu{Open with}{Central Test Data Editor}{}\\
from the \gdtestsuitebrowser{}.
\item In the editor, single-click the central test data set you want to add data to.
\item In the \gddatasetsview{}, make sure the language in the combo box on the right is the right language for your data. 
\item Select \bxcaption{Add} to add a row. 
\item Enter the values for the parameters in the row. 
\bxtipp{You cannot add referenced parameters (i.e. reference names preceded by the equals sign) in the \gddatasetsview{} for a central test data set.}
\item Use the buttons in the \gddatasetsview{} to  add more rows, delete rows and insert rows above the currently selected row. 
\end{enumerate}

\subsubsection{Reusing central test data sets in \gdcases{}}
\gdhelpid{guidancerCentralTestDataEditorContextId}{Central Test Data}
\gdhelpid{guidancerPropertiesViewContextId}{Properties View}
\label{TasksCentralDataReference}

You can reuse a central test data set in a \gdcase{} to provide the concrete data for the parameters required by the \gdcase{}. 

\begin{enumerate}
\item In the \gdpropview{} for the \gdcase{}, enter the name of the central test data set you want to use in the  \bxname{Central Test Data Set} field. 
\bxtipp{Press \bxkey{Ctrl+Space} to see a list of possible data sets for this \gdcase{}. You will only be shown data sets that contain the correct parameters with the correct types.}
\item When you have entered a central test data set, then the \gdpropview{} shows \bxname{central test data set} as the data type. You will see the data from the central test data set in read-only form in the \gddatasetsview{}.
\bxtipp{If data is missing from the central test data set, you will receive the error that test data is incomplete for any  \gdsuites{} this \gdcase{} is used in.}
\item You can delete the central test data set used by removing it from the \bxname{central test data set} field. The data type reverts to \bxname{local data}.
For more information on the data sources, see the earlier section \bxpref{TasksDataSources}.
\bxtipp{You can use central test data sets that contain more parameters than your \gdcase{}. For example, if your \gdcase{} requires the parameters \bxshell{NAME, ADDRESS} and your central test data set contains \bxshell{NAME, ADDRESS, POSTCODE}, you can still use the central test data set.}
\end{enumerate}

\subsubsection{Importing Excel files as central test data}
\gdhelpid{guidancerCentralTestDataEditorContextId}{Central Test Data}
\gdhelpid{guidancerPropertiesViewContextId}{Properties View}
\label{TasksCentralDataImport}

If you have existing Excel files that you want to convert into central test data, then you can import them via the \gddataeditor{}:

\begin{enumerate}
\item In the \gddataeditor{}, right-click and select \bxmenu{Import}{}{} from the context-sensitive menu.
\item In the dialog that appears, browse to the directory containing your Excel file(s).
\item Either select the whole directory (if it contains all Excel files) on the left, or select the individual Excel files on the right.
\item Click \bxcaption{Finish} to start the import.
\bxwarn{Your Excel files must contain the correct amount and type of \gdproject{} languages.}
\end{enumerate}

\subsubsection{Changing the column used in a central test data set for multiple \gdcases{}}
\gdhelpid{searchRefactorChangeCtdsColumnUsageWizardPagePageContextId}{Changing Central Test Data Columns}
\gdhelpid{guidancerCentralTestDataEditorContextId}{Central Test Data}
\gdhelpid{searchResultViewContextId}{Search Result View}
\label{TasksChangeCTDSColumn}
If you have used a central test data set in multiple \gdcases{} and later realize that you have two columns in the central test data set that contain the same information, then you can change all \gdcases{} that use this central test data set to just use one column. Once you have done this, you can remove the unnecessary column from the central test data set. 

\bxwarn{In order to perform this action, all \gdcases{} to be changed must not be in use by anyone else using the \gdproject{} -- you should ensure that this is the case before performing the action, otherwise the action cannot be carried out.}


\begin{enumerate}
\item Search for all places where the central test data set whose column usage you want to change is used. Use \bxname{Show where used} on the central test data set to see all places \bxpref{TasksShowWhereUsedDataSet}. 
\item In the \gdsearchresultview{}, you will see all places where the selected central test data set is reused in this \gdproject{}, including any original specifications of \gdcases{} that use it, and any reused \gdcases{} that use it (as \gdcases{} or as \gdehandlers{}. We recommend selecting all entries to perform the action, otherwise you may have incomplete test data after only changing a subset. 
\item You will only be able to perform the changes if:
\begin{itemize}
\item All the selected \gdcases{} use the same central test data set.
\item All the selected \gdcases{} have their original specification in this \gdproject{}, i.e. you do not have a \gdcase{} from a reused \gdproject{} that uses the central test data set in this \gdproject{}.
\item The \gdproject{} is not protected.
\item The \gdcases{} you have selected are present (i.e. they are not missing from e.g. reused \gdprojects{}. 
\item None of the selected \gdcases{} uses two different central test data sets (e.g. one central test data set on the originally specified \gdcase{}, and another one at a place where it is reused).
\item The selected \gdcases{} have at least one referenced parameter and at least one of the referenced parameters has the same type as another parameter in the central test data set.
\end{itemize}
\item If the \gdproject{} is protected, or \gdcases{} are missing, the context-menu entry will be disabled. If any of the other prerequisites are not fulfilled, you will see an information dialog that lets you automatically deselect any invalid \gdcases{}. 
\item From the context menu, select:\\
\bxmenu{Change central test data set column usage}{}{}\\
\item In the dialog that appears, you can select a parameter to change on the left-hand side, and the column to change it to on the right-hand side. You can only change the usage of columns whose types are the same. If you have e.g. TEXT1 (string) and TEXT2 (string)  in your central test data set and \gdcases{}, but you only require TEXT1, then select TEXT2 on the left-hand side and TEXT1 on the right hand side.
\item Click \bxcaption{Finish} to perform the action. 
\item All selected \gdcases{} will be altered to use the newly chosen column in place of the previously used column. 
\item If the newly chosen column was already used in one or more of the selected \gdcases{}, then all places within the \gdcase{} that referenced the old column are changed to reference the newly chosen column. Using the example from above, all places where TEXT2 was referenced in the selected \gdcase{} will be changed to TEXT1. The changed name remains at the interface of the \gdcase{}; TEXT2 is not deleted from the \gdcase{}, but it is no longer used. If the newly chosen column was not yet used in one or more of the selected \gdcases{}, the old parameter name of the \gdcase{} is renamed to the newly chosen column. 
\end{enumerate}
