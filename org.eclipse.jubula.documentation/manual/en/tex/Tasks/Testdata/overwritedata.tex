\index{Overwrite data}
\index{Parameter!Overwrite}

When you reuse \gdcases{}, you reuse them with any concrete values they contain, and with any default values that you have entered for their referenced parameters.  

Data which has been entered for referenced parameters can be overwritten when you reuse a \gdcase{}.

Reusing \gdcases{} happens in two ways:
\begin{itemize}
\item by adding the \gdcase{} to another \gdcase{} \bxpref{TasksEditorAdd}. The \gdcase{} is then \bxname{nested} in the \gdcase{}. 
\item by adding the \gdcase{} to a \gdsuite{} \bxpref{TasksEditorAdd}. The \gdcase{} is then \bxname{nested} in the \gdsuite{}. 
\end{itemize}

\begin{enumerate}
\item Single-click the reused \gdcase{} in the editor for the \gdcase{} or \gdsuite{} where you have reused it. 

\item In the \gdpropview{}, you can see the parameters you referenced from in this \gdcase{}. The data source for this \gdcase{} is shown as \bxname{referenced \gdcase{}} to denote that the data have not been changed after reusing the \gdcase{}. 

\bxtipp{A grey diamond next to the \bxname{parameter value} field means that the values in it were entered in the original specification of the \gdcase{}.}

\item You can enter parameter values here or reference the parameters again. They would then become parameters of the \bxname{parent} \gdcase{}.

\bxtipp{If you enter values here, you can see a yellow diamond next to the \bxname{parameter value} field. This means that the original data have been overwritten. The data source changes to read \bxname{local \gdcase{}}} 

If you add references here, you will be able to enter or overwrite data when you reuse the parent \gdcase{}. 

\bxtipp{Once you change the parameter values of a reused \gdcase{}, any changes to the parameters in the original specification of that \gdcase{} will not affect your new values. You can reset any local changes to the  data of the \gdcase{} by removing all Excel files or Central Test Data Sets and then selecting \bxname{Referenced \gdcase{}} from the data source combo box.}
\end{enumerate}


