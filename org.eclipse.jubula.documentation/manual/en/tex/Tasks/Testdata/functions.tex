\index{Functions}
\index{Parameter!Function}

You can let \app{} calculate specific values for you without having to enter the results yourself by using \bxname{functions}. There are specific functions that work out-of-the-box with \app{}, and additional functions can be added as well. 
\subsubsection{Syntax for functions}
\index{Functions!Syntax}
The sign used to introduce a function is the question mark: \bxshell{?} (without quotes). 

After the sign, you must enter the name of the function followed by the arguments the function requires:

\verb+?add(arg1,arg2)+

The arguments are separated by commas and are placed within round brackets. 

\subsubsection{Pre-defined functions}
The following functions are available out of the box:
\begin{description}
\item [add]{Adds the two following numbers: \bxshell{?add(1,2)}}
\item [sub]{Subtracts the second number from the first: \bxshell{?sub(3,2)}}
\item [mult]{Multiplies the second number by the first: \bxshell{?mult(2,4)}}
\item [div]{Divides the first number by the second: \bxshell{?div(2,1)}}
\item [trunc]{}
\item [round]{}
\item [now]{}
\item [formatDate]{}
\item [parseDate]{}
\item [modifyDate]{}
\end{description}

\subsubsection{Embedding functions in other functions}

\subsubsection{Concatenating functions with other data}

\subsubsection{Useful examples for functions}

\subsubsection{Adding your own functions}

