\index{Functions}
\index{Parameter!Function}

You can let \app{} calculate specific values for you without having to enter the results yourself by using \bxname{functions}. There are specific functions that work out-of-the-box with \app{}, and additional functions can be added as well. 

\subsubsection{Syntax for functions}
\index{Functions!Syntax}
The sign used to introduce a function is the question mark: \bxshell{?} (without quotes). 

After the sign, you must enter the name of the function followed by the arguments the function requires:

\verb+?add(arg1,arg2)+

The arguments are separated by commas and are placed within round brackets. 

\subsubsection{Pre-defined functions}
The following functions are available out of the box:

\textbf{Mathematical functions}\\
The following functions give their results as decimal numbers, e.g. 1.0, 1.2 etc. 
\begin{description}
\item [add]{Adds 0 or more numbers to 0, e.g.: \bxshell{?add(1,2)}.}
\item [sub]{Subtracts the second number from the first: \bxshell{?sub(3,2). This function only accepts two numbers.}}
\item [mult]{Multiplies 0 or more numbers by 1 e.g.: \bxshell{?mult(2,4)}.}
\item [div]{Divides the first number by the second: \bxshell{?div(2,1)}. This function only accepts two numbers.}
\item [trunc]{Takes two arguments, the decimal to be truncated and the precision (as an integer) to truncate the decimal to. Use \bxshell{0} to cut off the number to no decimal places (i.e. to receive a plain integer), and use \bxshell{1} to cut off the decimal to one decimal place etc: \bxshell{?trunc(2.396,0)} gives \bxshell{2} and \bxshell{?trunc(2.789,1) gives \bxshell{2.7}}.}
\item [round]{Takes two arguments, the decimal to be rounded and the precision (as an integer) to round to. This function uses \bxname{half-up} rounding to round the number so that if the final decimal place after rounding is 5 or higher, the final number will be incremented by 1 e.g.: \bxshell{?round(2.56,1)} gives \bxshell{2.6}. If the final number after rounding is 4 or less, there is no incrementation, eg. \bxshell{?round(2.46,1)} gives \bxshell{2.4}. }
\bxwarn{It is currently only possible to use numbers formatted with the decimal mark \bxname{period} or \bxname{fullstop} (\bxshell{.}). Thousands separators may not be used. For example, \bxshell{1.5} is accepted, but \bxshell{1,5} is not. \bxshell{1000} can be entered but \bxshell{1,000} cannot. Entering \bxshell{1.000} is equivalent to entering \bxshell{1}.}
\end{description}

%% \textbf{Date functions}\\
%% \begin{description}
%% \item [now]{}
%% \item [formatDate]{}
%% \item [parseDate]{}
%% \item [modifyDate]{}
%% \end{description}

\subsubsection{Embedding functions in other functions}
Functions can be added as arguments to other functions. If, for example, you want to use the result of a subtraction as the first argument of your addition, you could write it like this:\\
\bxshell{?add(?sub(2,1),1)}\\
Results in \bxshell{1.0 + 1}, i.e. \bxshell{1.0}\\

%\subsubsection{Useful examples for functions}

\subsubsection{Adding your own functions}
You can also add your own functions using an extension point. This is described in the Extension Manual. 




