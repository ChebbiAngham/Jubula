\index{Variables}
\index{Parameter!Variable}

\begin{itemize}
\item \jb{} lets you store values from the \gdaut{} to use later.
\item Some actions (e.g. store value) let you save a value as a variable. You specify the name of the variable, e.g. \bxshell{USERNAME}. 
\item When you want to use the variable, you can enter it as a parameter by preceding it with a dollar sign: \bxshell{\$USERNAME}. 
\end{itemize}
For more information on using variables, see the following sections:
\begin{itemize}
\item Working with variables in tests \bxpref{TasksAUTVariables}.
\item Working with system variables \bxpref{TasksSystemVariables}.
\item Working with the \jb{} pre-defined variables \bxpref{TasksPredefinedVariables}.
\end{itemize}

\subsubsection{Reading and using values (variables) from the \gdaut{}}
\index{Variables!Storing}
\index{Storing Variables}
\label{TasksAUTVariables}

You can store values read from the \gdaut{} to use as data in other \gdcases{}. 

\begin{enumerate}
\item Use one of  the \bxname{store value} actions on the various components to  reads a value from a component in the \gdaut{}. 
\bxtipp{You can also use the \bxname{store value} action on the \bxname{application} component to store a value you enter.}

\item In the parameter field, enter a name for this variable (e.g. USERNAME). 

\bxtipp{Variable names may only contain letters, numbers and underscores}
\item When you want to use this value as data for a parameter, enter the variable name preceded by a dollar sign (\$) as the parameter value (e.g. \$USERNAME).

Bear in mind that the variable has to be stored before it can be used as a parameter value.  

\end{enumerate}
Read the following sections for more information on:
\begin{itemize}
\item Using system variables in tests \bxpref{TasksSystemVariables}. 
\item Using the pre-defined \jb{} variables in tests \bxpref{TasksPredefinedVariables}
\end{itemize}

\subsubsection{Using system variables in tests}
\label{TasksSystemVariables}
\index{Variables!System}
\index{System variables}
\index{Parameter!JRE}
\index{JRE parameters}


\jb{} lets you add variables to your operating system, which can be used in your tests. 

You will need to set environment variables which have the form:\\

\bxshell{GDUDV\_<variablename>}\\

\bxname{GDUDV} stands for \bxcaption{\jb{} user-defined variable}. 

To use the variable in your tests, enter the variable name (everything after the underscore) preceded by a dollar sign. Do not enter the \bxcaption{GDUDV\_} part.

\bxwarn{After entering or changing a system variable, you will need to restart \jb{}.}

Your system administrator will be able to help you with operating-system specific ways of setting environment variables. 



\subsubsection{Using the \jb{}pre-defined test execution variables}
\label{TasksPredefinedVariables}
\index{Variables!Pre-defined}
\index{Predefined Variables}

\begin{enumerate}
\item \jb{} contains pre-defined test execution variables which you can use in your tests. 
\item The following variables are automatically initialized when executing a \gdsuite{}:
 \begin{description}
 \item [GD\_TESTSUITE:]{The \gdsuite{} name.}
 \item [GD\_USERNAME:]{The account name you are logged into your computer under.}
 \item [GD\_DBUSERNAME:]{The \gddb{} user.}
 \item [GD\_AUTSTARTER:]{The hostname for the \gdserver{} the test is running on.}
 \item [GD\_PORTNUMBER:]{The port number for the \gdserver{} the test is running on.}
 \item [GD\_AUT:]{The \gdaut{} name.}
 \item [GD\_AUTCONFIG:]{The \gdaut{} configuration name.}
 \item [GD\_CLIENTVERSION:]{The version of the \jb{} client you are using.}
 \item [GD\_LANGUAGE:]{The language the AUT and the test are running in, e.g en\_US. }
 \end{description}
\item To use the value of one of these variables in your test, enter: \\
\verb+${VARIABLE_NAME}+\\
as the parameter value. 
\end{enumerate}
\bxtipp{For a list of language codes, see the section in the reference manual (\bxextref{\jb{}efman}{ref,langcodes}) for details. }

