\index{References}
\index{Parameter!Reference}
\begin{itemize}
\item If there are parameters in a \gdcase{} that you want to change when you reuse it, you can enter a reference instead of a concrete value. Deciding which data to parametrize is an important decision in test creation. See the Best Practices section for more information on this \bxpref{BPKeywordData}.
\bxtipp{Press \bxkey{Ctrl+SPACE} to get content assist in the \gdpropview{} for certain parameters. }
\item To enter a reference for a parameter, in the \gdpropview{}, enter a reference name, preceded by an equals sign: \bxshell{=REF\_NAME}. 

\bxtipp{Reference names may only consist of letters, numbers and underscores. You cannot use spaces in reference names.}
\item Press \bxkey{Enter}. 
A yellow arrow will appear next to the \bxname{parameter} field in the \gdpropview{}.
\gdmarpar{../../../share/PS/refArrow}{ reference symbol}
\bxtipp{It helps to choose names that are meaningful so that you know what sort of data to enter later. Instead of \bxname{=TEXT}, you could use \bxname{=CATEGORY\_NAME}, for example \bxpref{BPParamNames}.}
\item You will see that the reference becomes a parameter of the parent \gdcase{}. It appears in the \gdtestcasebrowser{} next to the parent \gdcase{} in square brackets. 
\item You will be able to enter data for this parameter when you reuse the \gdcase{}. 
\bxtipp{You can enter data for it now if you want to -- this is essentially like having default data. They appear when you reuse the \gdcase{}, but you can overwrite them \bxpref{TasksOverwriteData}.}
\end{itemize}

For information on editing and deleting referenced parameters from \gdcases{}, see the following section \bxpref{TasksTestdataEditParams}. 



