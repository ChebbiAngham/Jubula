\index{Parameter!Concatenation}
\index{Concatenation}

The \ite{} lets you use various different types of parameter:
\begin{itemize}
\item Concrete values \bxpref{TasksTestdataConcrete}.
\item Referenced parameters \bxpref{TasksTestdataReferences}.
\item Variables \bxpref{TasksVariables}.
\item Functions \bxpref{TasksDataFunctions}.
\end{itemize}

You can use these parameters separately, or you can combine them to create a parameter value. This is useful if a value you want to enter or check consists of parts that change and parts that stay the same. 

To combine different types of parameter to make one value, you must write them in a specific way:
\begin{enumerate}
\item Referenced parameters must be written with curly brackets around the reference name:\\
\bxshell{=\{REF\_NAME\}}
\item Variable names must also be written with curly brackets around them:\\
\bxshell{\$\{VAR\_NAME\}}
\item Concrete values are written as normal. 
\item For example, you can build a data string that contains all four types of data:\\
\bxshell{test\_=\{PROJECTNAME\}\_\$\{CUSTOMERNUMBER\}\_?now()}\\

\end{enumerate}

