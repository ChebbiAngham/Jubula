\index{Empty String}
\index{Escape Character}
\index{Parameter!Empty String}
\index{Parameter!Escape Character}

\textbf{Entering empty strings as parameters}\\
If you want the parameter you enter to be an \bxname{empty string} (i.e. nothing), use two single quote marks: ''\\

You can use this with the \bxname{equals}, \bxname{matches} or \bxname{simple match} operators.

You can also use '\verb+^$+' or '\verb+^\s*$+' with the operator \bxname{matches} to check that a text area is empty.  

You can also use '\verb+^$+' with the operator \bxname{matches} to check that a text area is empty.  

\bxtipp{If a component looks empty, but entering an empty string doesn't work, it may be worth asking a developer what is actually in the component.}

\textbf{The escape character}\\
Some symbols have a special meaning for test execution. If you want to use the symbol without the special function, you have to \bxname{escape} it. The symbol to negate any special function of the following symbol is a backslash: ($\backslash$). 
                          
See the Reference Manual (\bxextref{\gdrefman}{ref,specialchar}) for more details on special symbols and escaping them.

\bxtipp{When you are using regular expressions, you will also need to think about which symbols need neutralising. Sometimes more than one backslash is necessary.}

