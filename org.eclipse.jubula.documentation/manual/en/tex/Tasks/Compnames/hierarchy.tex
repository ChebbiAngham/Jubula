\index{Component!Names!Hierarchy}
\index{Hierarchy of Component Names}
\index{Abstract components}
The component hierarchy  is designed to allow flexible test specification and robust tests. For a graphical overview of the component hierarchy, see the reference manual \bxextref{\gdrefman}{ref,overviewabstcomp}. 
\bigskip

\textbf{Abstract components}\\

You can write tests very abstractly at the beginning, only adding detail later. You will notice that the library contains categories such as \bxname{Component with Text Input} and \bxname{Graphics Component}. 

These are \bxname{abstract components} -- actions in these categories can be used on a wide range of actual components in the \gdaut{}. You can use a \bxname{Click} action on the \bxname{Graphics Component} to click any component in the \gdaut{}. You just need to enter different component names for it in the \gdcompnamesview{}. 
\bigskip
\textbf{Using the same component name for different component types}\\

What happens if you want to specify a test that clicks in a table and then selects a cell in the table?

The click action is on the \bxname{Graphics Component} and the select cell action is on the \bxname{Table} component -- but you don't want to have two different component names. 

This isn't a problem. You can use the same component name for different components as long as these are compatible. So, in this case, the \bxname{Graphics Component} and the  \bxname{Table} component can both use the component name e.g. \bxname{TableView\_MainTable\_tbl}. 

\bxwarn{You cannot use the same component name for incompatible types, for example, trees and tables.}

An overview of the component hierarchy can be seen in: \bxextref{\gdrefman}{ref,overviewabstcomp}.
