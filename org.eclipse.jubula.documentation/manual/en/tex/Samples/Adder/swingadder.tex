You can start the Swing SimpleAdder using the configurations for the \bxname{SimpleAdder} \gdaut{}. For Windows and Linux, you can choose whether you want to use the JRE installed with the \ite{} or your system JRE. Mac users can use the Mac configuration to start the SimpleAdder with their default Java version.

\subsubsection{Sample 1.1: creating a \gdcase{} from \gdsteps{}}
\label{samplesteststeps}

This category contains one \gdcase{}. The \gdcase{} contains four \gdsteps{}, which test an addition in the Simple Adder. The parameter values in the \gdsteps{} have been referenced, and a data set has been added. 

This is an example of a test written with \gdsteps{}. However, we recommend using the library of \gdcases{} to write tests, as shown in the next examples. 


\subsubsection{Sample 1.2: creating a \gdcase{} using the library}
\label{sampleslibrary}

This category contains one \gdcase{}. The \gdcase{} has referenced another \gdcase{}, with four reused \gdcases, which have been reused from the \gdproject{} \bxname{unbound\_modules\_concrete}. 
\bxtipp{Press \bxkey{F6} to find where a particular \gdcase{} was originally specified. }

The \gdcases{} carry out the same steps as in the previous example \bxpref{samplesteststeps}. The differences here are:\\

\begin{itemize}
\item The steps to enter a value both reuse the same \gdcase{}, with different referenced parameters, and a different component name. 
\item The components used in the reused \gdcases{} are \bxname{abstract} components \bxpref{TasksCompNameType}. This means that the \gdcases{} are easier to reuse, making tests more robust and maintainable. 
\end{itemize}

This \gdcase{} is reused in the executable \gdcase{} \bxname{1.2\_SIMPLE\_ADDER\_TEST}, which is nested in the \gdsuite{} of the same name. 

\subsubsection{Sample 1.3: using \gdehandlers{}}

This category has four subcategories. Each subcategory contains a test which reuses a \gdcase{} to execute a calculation in the Simple Adder which will cause an error. After the error, a reset is carried out. 

An \gdehandler{} has been specified in the \bxname{bound\_modules\_samples} \gdproject{}. The \gdehandler{} has been added to the \gdcase{}, and checks that text in the result field is \bxname{jackpot}. 

The four tests are as follows:\\

\textbf{Continue}\\
The \gdehandler{} in this test has the reentry property \bxname{continue}. When the error occurs, the \gdehandler{} is activated. Once the check has been carried out, the test continues, and the reset is performed. \\
\textbf{Exit}\\
The \gdehandler{} in this test has the reentry property \bxname{exit}. When the error occurs, the \gdehandler{} is activated. Once the check has been carried out, the test finishes. The reset is not performed.  \\
\textbf{Pause}\\
The \gdehandler{} in this test has the reentry property \bxname{pause}. When the error occurs, the \gdehandler{} is activated. Once the check has been carried out, the test pauses. By un-pausing the \gdsuite{} in the client, the test continues.   \\
\textbf{Retry}\\
The \gdehandler{} in this test is different to the \gdehandler{} in the other tests. It contains the same steps as the test itself, but the parameter references have been switched. This essentially changes the order in which the numbers are entered into the Simple Adder. The \gdehandler{} has the reentry property \bxname{retry}. When the error occurs, the \gdehandler{} is activated. The calculation is redone with the switched values. The failed \gdstep{} (i.e. the original check) is retried, and there is no error. The test is marked as successful.  \\

