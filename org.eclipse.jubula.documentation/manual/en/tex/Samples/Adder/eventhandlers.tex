
% $Id: eventhandlers.tex 8104 2009-03-31 13:26:29Z alexandra $
% Local Variables:
% ispell-check-comments: nil
% Local IspellDict: american
% End:
% --------------------------------------------------------
% User documentation
% copyright by BREDEX GmbH 2004
% --------------------------------------------------------

This category has four subcategories. Each subcategory contains a test which reuses a \gdcase{} to execute a calculation in the Simple Adder which will cause an error. After the error, a reset is carried out. 

An \gdehandler{} has been specified in the \bxname{bound\_modules\_samples} \gdproject{}. The \gdehandler{} has been added to the \gdcase{}, and checks that text in the result field is \bxname{jackpot}. 

The four tests are as follows:\\

\textbf{Continue}\\
The \gdehandler{} in this test has the reentry property \bxname{continue}. When the error occurs, the \gdehandler{} is activated. Once the check has been carried out, the test continues, and the reset is performed. \\
\textbf{Exit}\\
The \gdehandler{} in this test has the reentry property \bxname{exit}. When the error occurs, the \gdehandler{} is activated. Once the check has been carried out, the test finishes. The reset is not performed.  \\
\textbf{Pause}\\
The \gdehandler{} in this test has the reentry property \bxname{pause}. When the error occurs, the \gdehandler{} is activated. Once the check has been carried out, the test pauses. By un-pausing the \gdsuite{} in the client, the test continues.   \\
\textbf{Retry}\\
The \gdehandler{} in this test is different to the \gdehandler{} in the other tests. It contains the same steps as the test itself, but the parameter references have been switched. This essentially changes the order in which the numbers are entered into the Simple Adder. The \gdehandler{} has the reentry property \bxname{retry}. When the error occurs, the \gdehandler{} is activated. The calculation is redone with the switched values. The failed \gdstep{} (i.e. the original check) is retried, and there is no error. The test is marked as successful.  \\


