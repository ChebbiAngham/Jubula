

The test for the meters tool contains three use cases. 
\begin{description}
\item [Taking a reading:]{ this use case takes a meter reading for a chosen flat. The actual steps of entering the reading and clicking \bxcaption{OK} or \bxcaption{Next} in the dialog are contained in the \gdcase{} \bxname{uc1} as they are reused elsewhere in the test.}
\item [Moving a tenant in:]{ this use case selects an empty flat from the hierarchy and selects the option to move a tenant into the flat. It reuses use case 1 because a reading needs to be taken for each of the three meters.}
\item [Moving a tenant out:]{ this use case selects a flat and selects the option to move a tenant out. It reuses use case 1 again to enter the readings. } 
\end{description}

In the \bxname{bound\_modules\_samples} \gdproject{}, you can see the building blocks that these use cases are made up of in the \bxname{AUT bound modules} category. 

\subsubsection{Concatenated parameters}
The meters application has a fairly complex tree structure in terms of the data in it. Look in the \gdcase{} \bxname{mtr\_tre\_selectNode} in the \bxname{bound\_modules\_samples} \gdproject{} to see how the treepath has been concatenated out of the concrete value \bxname{Building} and then two further parameters for the location of the meter and the meter number. This makes entering the test data easier. 
