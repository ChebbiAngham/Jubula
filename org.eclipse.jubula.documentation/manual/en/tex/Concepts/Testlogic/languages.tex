% BREDEX LaTeX Template
%  \documentclass is either ``bxreport'' or ``bxarticle''
%                 option is bxpaper
%% \documentclass{bxarticle}
%% % ----------------------------------------------------------------------
%% \begin{document}
%% \title{}
%% \author{}
%% % \author*{Hauptautor}{Liste der Nebenautoren}
%% \maketitle
%% % ----------------------------------------------------------------------
%% \bxversion{0.1}
%% %\bxdocinfo{STATUS}{freigegeben durch}{freigegeben am}{Verteilerliste}
%% \bxdocinfo{DRAFT}{}{}{}
%% % ----------------------------------------------------------------------

%% \end{document}
\subsection{\gdproject{} and \gdaut{}  languages}
\index{Languages!Project}
\index{Languages!AUT}
\index{Project!Languages}
\index{AUT!Languages}
\index{Default Language}
\index{Working language}
\index{Languages!Working}

With \gd{}, you can test different language versions of your \gdaut{} with the same test. All you have to do is translate the data for the test. 

To make multilingual testing possible, you have to tell \gd{} which languages you are using: 

\begin{description}
\item [\gdproject language(s):]{These languages are specified during \gdproject{} creation and are valid for all \gdauts{} in this \gdproject{}. You will be able to have \gdauts{} which support one or more of these languages.}
\item[Default language:]{One of the \gdproject{} languages is the default language. This is the language which will automatically be set as the working language when you open the \gdproject{}. }
\item[\gdaut language(s):]{When you are defining an \gdaut{}, you choose one or more \gdaut{} languages from the selection of \gdproject{} languages. These languages are the languages the \gdaut{} supports. You can start the \gdaut{} in these languages and write test data for these languages. }
\item[The working language:]{In the \gd{} toolbar, you can change the working language to any of the \gdproject{} languages. The working language determines which \gdsuites{} you can edit (only those whose \gdaut{} supports this language) and the language the \gdaut{} is started in. }
\end{description}

