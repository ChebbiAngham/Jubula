% $Id: recording.tex 8161 2009-04-06 14:07:39Z alexandra $
% Local Variables:
% ispell-check-comments: nil
% Local IspellDict: american
% End:
% --------------------------------------------------------
% User documentation
% copyright by BREDEX GmbH 2004
% --------------------------------------------------------
\index{Observation}
\index{Top-Down Specification}
\index{Specification!Approaches!Top-Down}

Specifying tests by hand is certainly a professional way of 
creating tests; it makes you develop ideas about which parts of a test can  
be reused, and also allows test specification to begin before components
have been developed. However, allowing the program to observe user actions 
can have its advantages.

\jb{} lets you observe \gdcases{} in a running \gdaut{}, using a method which differs considerably from  traditional recording options. You can record actions and checks in your \gdaut{}, but the output for these actions is \gdsteps{}, not code. 

You can create the same tests with observing as you can with specification. The only real differences are:
\begin{itemize}
\item object mapping is carried out automatically in observing mode. Because of this, you do not provide a component name, but one is created by \jb{}. 
\item you must use concrete values for parameters in the observation mode. You can, however, change these to references in the specification perspective later so that your tests are more reusable. 
\end{itemize}

You can reuse observed \gdcases{} in the same ways as you can reuse specified \gdcases{}. 

\bxtipp{The observation mode supports high level actions. To find out
exactly how an action works, look up the component and its action in
\bxextref{\jb{}efman}{ref,actparam}}. 

To  execute an observed test, you just need to nest the observed
\gdcase in a \gdsuite{}. This is an example of a top-down approach to testing.
     



