% $Id: recording.tex 8161 2009-04-06 14:07:39Z alexandra $
% Local Variables:
% ispell-check-comments: nil
% Local IspellDict: american
% End:
% --------------------------------------------------------
% User documentation
% copyright by BREDEX GmbH 2004
% --------------------------------------------------------
\index{Observation}

As explained in the introduction \bxpref{IntroRecording} and in the section on how to record \bxpref{TasksObsModeTips}, we believe that recording tests has various disadvantages in any tool. 

However, you can  observe \gdcases{} in a running Java \gdaut{}. You can record actions and checks in your \gdaut{}, and the output for these actions is \gdsteps{}. 

You can create the same tests with observing as you can with specification. The real differences are:
\begin{itemize}
\item object mapping is carried out automatically in observing mode. Because of this, you do not provide a component name, but one is created automatically. 
\item you must use concrete values for parameters in the observation mode. You can (and should), however, change these to references in the specification perspective later so that your tests are more reusable. 
\end{itemize}

You can reuse observed \gdcases{} in the same ways as you can reuse specified \gdcases{}. 

\bxtipp{The observation mode supports high level actions. To find out
exactly how an action works, look up the component and its action in
\bxextref{\gdrefman}{ref,actparam}}. 


     



