% $Id: execution.tex 8161 2009-04-06 14:07:39Z alexandra $
% Local Variables:
% ispell-check-comments: nil
% Local IspellDict: american
% End:
% --------------------------------------------------------
% User documentation
% copyright by BREDEX GmbH 2004
% --------------------------------------------------------
\index{Results!XML and HTML Reports}
To be able to execute a test, you must have:
\begin{itemize}
\item created a \gdsuite{} and added \gdcases{} to it.
\item defined and configured an \gdaut{} for the \gdproject{}. 
\item object mapped the components from the \gdsuite{}. 
\item added any necessary test data for the language you want to test. 
\end{itemize}

You can follow the execution in the \gdtestresultview{} and you can opt to produce HTML and XML result reports. 

\subsection{\gdstep{} execution}
\jb{} executes tests on the following basis: 
\begin{itemize}
\item A user-defined component is located in the user interface of the \gdaut{}. 
\item The specified action is executed on this component, with the parameters you gave for the action. 
\end{itemize}

\jb{} bases the success of an action on whether it was carried out or not. There are no implicit checks which verify the state of the application to check if the execution of an action actually had an effect. For example, if an action is carried out to click a button, this \gdstep{} will be marked as successful even if the button is disabled. From \jb{}'s perspective, a click was executed on the button component. 

It is worth bearing this in mind if a test has failed, especially because of a ''component not found'' error. What may have happened is that a further component was not found due to a dialog still being in focus, because the \bxcaption{close} button  was disabled, for example. Although a click was executed on this button, it did not have the desired effect. The test continued, but ran into problems. 
