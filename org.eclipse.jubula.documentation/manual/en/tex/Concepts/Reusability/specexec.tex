% $Id: specexec.tex 4493 2007-02-08 09:58:18Z mike $
% Local Variables:
% ispell-check-comments: nil
% Local IspellDict: american
% End:
% --------------------------------------------------------
% User documentation
% copyright by BREDEX GmbH 2004
% --------------------------------------------------------
\index{Specification!Names}
\index{Execution!Names}
\index{Names!Specification}
\index{Names!Execution}
\index{Component!Names!Reassigning}
\index{Reassigning!Component Names}
\index{Names!Component!Reassigning}
\index{Color Coding}
\index{Master Template}
\label{specexec}

Specification occurs in the \gdtestcasebrowser{}, \gdmarpar{../../../share/PS/testCase}{specified \gdcase{}}, \gdtestcaseeditor{} and its support views. A specified \gdcase{} is a \emph{master template}. 

Master templates can be reused in two ways: 

\begin{itemize}
\item  They can be added to a \gdsuite{} (in the \gdtestsuiteeditor{}). 
\item They can be nested in another \gdcase{} (in the \gdtestcaseeditor{}). 
\end{itemize}

\gdcases{} show a small arrow when they are reused to show that they are only a reference to the master template.  
\gdmarpar{../../../share/PS/testCaseRef}{reused \gdcase}

%See the diagram in the \bxcaption{User Interface} chapter (\bxfigref{specexec}) 
%for details on where specification and reuse occur. 

\subsubsection{Modifying Reused \gdcases}
You can modify reused \gdcases{} in three ways:
\begin{itemize}
\item You can change the data they test by overwriting any previously entered parameter values \bxpref{referencetasks}. This is only possible if you have moved the parameters up using references in the \gdsteps{}. 
\item You can change the component they test by reassigning the component names \bxpref{reass}. In this way, a single \gdcase{} can execute the same action on different components.  
\item You can change the \gdcase{} name. \gdcases{} have both \emph{specification} and \emph{execution} 
names. The specification name is the name of the master template, and you can overwrite this each time you reuse a \gdcase{}.  
\end{itemize}

Until the data or \gdcase{} name is overwritten in a reused \gdcase{}, any changes to the data or name in the master template will be carried through to all places where you have reused the \gdcase{}. If you have overwritten data or the \gdcase{} name in a reused \gdcase{}, the reused \gdcase{} will no longer be affected when the data or name in the master template changes. 


\subsubsection{Foundations}
\GD is a flexible program, allowing data values to be changed even after 
specification. However, in order to preserve test functionality, you cannot alter the names of references in a 
\gdstep{} or \gdcase{} if its parent \gdcase{} is nested in another \gdcase{}. 

That is, the references in the child \gdcase{} are part of the  \textbf{foundation} of the parent
\gdcase{}. This means that the  references in the nested \gdcase{} 
may not be changed, since modifications will change the foundation on which 
the parent \gdcase{} is built. This may adversely 
affect the test by making the other  \gdcase{} unusable.  

However, there is a way of altering \gdsteps{} and references in reused 
\gdcases{}. See the section on modifying reused \gdcases{} 
\bxpref{overwritingreused} in \bxcaption{Tasks} for more details. 

%Reference values, however, may be edited at any time, as long as their 
%eference has  been carried through the hierarchy to the \gdcase to be 
%edited. In the \gdtestsuitebrowser{}, for example, only values in the 
%top-level \gdcase can be edited. 

