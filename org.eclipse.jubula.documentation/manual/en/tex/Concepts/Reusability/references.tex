% $Id: references.tex 8161 2009-04-06 14:07:39Z alexandra $
% Local Variables:
% ispell-check-comments: nil
% Local IspellDict: american
% End:
% --------------------------------------------------------
% User documentation
% copyright by BREDEX GmbH 2004
% -------------------------------------------------------- 
\index{References}
\index{Parameter!References}
\label{referencesconcepts}

The first way to make \gdcases{} reusable is to use placeholders for the data in them. Instead of specifying concrete parameter values in \gdsteps{}, you can use references for values that will change each time you reuse the \gdcase{}. These references act as placeholders for values to be inserted later. 

The referenced parameter becomes a parameter of the parent \gdcase{}, and a value for it can be entered at the \gdcase{} level. 

The advantage of using references is that a value is not fixed for a particular \gdcase{}. Instead of creating a new \gdcase{} for each different username, for example, you can have one \gdcase{} which can contain all the usernames you want to test as its data. 

Using references also means that you can translate your data, you can use external data from an Excel file, and you can run the same \gdcase{} multiple times with different \bxname{data sets}. 

