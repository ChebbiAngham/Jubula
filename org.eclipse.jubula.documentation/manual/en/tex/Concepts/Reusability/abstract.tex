
\index{Components!Abstract}
\index{Abstract components}

Another way to make your \gdcases{} more reusable is to increase the amount of components they can test. For each action, you choose a component type.  \gd{} has three levels of components:
\begin{description}
\item [Abstract components]{are general, high-level components from which other components are derived. They are described in terms of what features a component has, e.g. graphics component, component with text. They group actions together which can all be executed on components of this type.}
\item [Concrete components]{are components which are available to all graphical toolkits, but which are restricted to a certain component type, e.g. combo box, list.}
\item [Toolkit specific components]{are the most specific components in \gd{}. They are only available for certain toolkits. For example, a HTML link is a component which is only available in Web applications.}
\end{description}

We recommend that you specify your tests as abstractly as possible, and as concretely as necessary. If you want to create a \gdcase{} to click a button, it is better to use the abstract component \bxname{graphics component}. The \bxname{graphics component} also contains the \bxname{click} action, and has the advantage that the \gdcase{} can then be used on other components than buttons. If you want to select a cell from a table, however, you will have to use the concrete component \bxname{table}, because this is the highest level which offers this action. 

The more abstract your specification, the more reusable your \gdcases{} are. Abstract specifications are also easier to maintain after changes in the user-interface. If the object mapping needs to be adapted to a change in the GUI, there is a larger chance that the new GUI-component can be mapped to the original specification if abstract components were used. 



