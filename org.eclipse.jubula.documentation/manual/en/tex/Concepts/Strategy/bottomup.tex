% $Id: bottomup.tex 6515 2008-02-18 11:33:01Z alexandra $
% Local Variables:
% ispell-check-comments: nil
% Local IspellDict: american
% End:
% --------------------------------------------------------
% User documentation
% copyright by BREDEX GmbH 2004
% --------------------------------------------------------
Bottom-up design approaches the testing process in a modular fashion. 
Small ''building blocks'' (the \gdcases{} in \jb{}) are specified in a general or abstract way, making them easy to adapt and reuse. 
These small units are used to construct other modules or building blocks, which are more adapted to the  task at hand. Concrete details can be added as needed.

One of the advantages of this approach is that tests are flexible and easy to maintain, since a change made at a central point can update many places where that \gdcase{} has been reused. Tests are also generally more efficient, because time is not wasted specifying what is essentially the same \gdcase{} over and over again. 

The difficulty with bottom-up testing is that it requires a good knowledge of the structure of the test -- you have to know which \gdcases{} are going to be required multiple times, and this is often not easy to recognize at the beginning of testing. 





