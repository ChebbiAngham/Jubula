 % $Id: refactoring.tex 6515 2008-02-18 11:33:01Z alexandra $
% Local Variables:
% ispell-check-comments: nil
% Local IspellDict: american
% End:
% --------------------------------------------------------
% User documentation
% copyright by BREDEX GmbH 2004
% --------------------------------------------------------
\index{Refactoring}
As well as supporting both bottom-up and top-down testing, \GD{} lets you change your specification from top-down to bottom-up. If you realize that a certain part of a \gdcase{} could actually be used as a \gdcase{} of its own -- i.e. it would be worth reusing it, you can ''extract'' a new \gdcase{} from it.   

This means that you can create ''building blocks'' from other \gdcases{}, making it unimportant if you realize at a later point that parts of a test would be better if they were specified bottom-up. 
