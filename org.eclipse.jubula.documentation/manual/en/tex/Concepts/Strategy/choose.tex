Obviously, these two approaches lie on a continuum. \app{} lets you ''mix-and-match'' so that you can choose the  approach that works best for you -- a combination of the two or one method for one \gdcase{} and the other for another \gdcase{}. 

For example, if you know that a certain test procedure (e.g. logging in) has to be executed frequently, you can specify a \gdcase{} to do this, and specfiy it to be easily reusable (using references as data-placeholders, and using general component names to be reassigned). For parts of the test which are not likely to be carried out more than once, you can specify them with data and definite component names. 