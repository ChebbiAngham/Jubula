\index{Design for Testability!Swing}
\index{Swing!Set Name}
\index{Set Name}
\index{Swing!Design For Testability}

\subsubsection{Naming components}
Although components in the \gdaut{} can be recognized even  when they are not named by the developers, using the \bxname{setName} method for the current Swing component class certainly makes it easier to test \gdauts{}. Even if a whole area of the \gdaut{} has changed, the component will still be found based on this unique name. 

\subsubsection{Adding support for text retrieval}
You can add support for renderers for Swing components without the getText() method in order to access text that is otherwise non-readable during test execution. 
\begin{itemize}
\item An example of the adapter mechanism can be found here:\\
\bxname{http://git.eclipse.org/c/jubula/org.eclipse.jubula.core.git/}\\
\bxname{tree/org.eclipse.jubula.examples.extension.swing.rc.adapter}
\item This does not replace the support for custom Swing renderers that can be changed directly by your developers. 
\item If you are able to change the renderers yourself, you can still implement one of the following in your renderer:
\begin{quote}
public String getTestableText();
public String getText();
\end{quote}
\end{itemize}
