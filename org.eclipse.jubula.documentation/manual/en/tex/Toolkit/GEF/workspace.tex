Although you can use the IDE of your choice to develop your accessibility plug-in, this guide will use Eclipse (specifically, Eclipse 3.4.1 with PDE plug-ins).

\begin{enumerate}
\item Before setting up your workspace, it is important to have the ECP Remote Control plug-in available, as this contains libraries and interfaces that are required in order to develop your accessibility plug-in. The archive file \bxname{rcp-support.zip}, available from the \app{} installation directory, contains the RCP Remote Control plug-in. Once you have extracted the RCP Remote Control plug-in, you can continue setting up your workspace.

\item In order to cleanly separate test code from application code, we recommend using separate workspaces for application and accessibility development. Create a new Eclipse workspace for accessibility plug-in development. 
\item Once you have created your new workspace, change that workspace's Plug-in Development Target Platform to include the RCP Remote Control plug-in, as well as all plug-ins that make up your \gdaut{}. Information on configuring the Target Platform plug-ins can be found in Eclipse's online Help under:\\
\bxname{Plug-in Development Environment Guide > Tools > Preferences > Target Plug-ins}.

\item Create a new Plug-in Project. The default values proposed by the New Plug-in Project wizard are acceptable, so simply enter a name for your new Project and complete the wizard without additional modifications.

\item You now have a workspace with a properly configured Target Platform as well as a new Plug-in Project. You can now begin developing your accessibility plug-in.
\end{enumerate}
