\index{iOS}
\index{AUT!iOS}
\index{iOS AUT's}
\index{iOS!Remote Control}

If you want to test an \bxname{iOS} application, you have to prepare the \gdaut{} in order to make it testable. 

\bxtipp{This preparation is designed to be undertaken by a developer who has access to the \gdaut{}'s source code as well as knowledge of developing for iOS using Objective-C and Xcode. These instructions assume you are using Xcode 4. For Xcode 3 please adapt the instructions accordingly.}

\subsubsection{Add the library to your project files}

The first step is to link the \bxname{librc.mobile.ios.nativ} static library and its header file directly 
into your \bxname{iOS} application. This lets your project and thereby your \bxname{iOS} application be run as a testable \gdaut{}. 

\begin{enumerate}
\item Locate the \bxname{development/iOS-support.zip} file in the installation directory in Finder.
\item Unzip it and drag all of its content into the Project Navigator.
\end{enumerate}

\subsubsection{Create a Testing Target}

We strongly recommend that you create a \textbf{separate target} which contains and uses all the necessary modifications for your \gdaut{} to be testable. 
Once you have created a second target for the testing-enabled version of your \gdaut{} to test, you can begin testing simply by running this second target. Having a separate target also ensures that no testing code will be released into the productive version of your app. 

The new target will start as a duplicate of your old target. To create the duplicated target:
\begin{enumerate}
\item Select the project file for your app in the Project Navigator.
\item From there, CTRL+click the target for your app and select the \bxcaption{Duplicate} option. Xcode may ask you if you want your copy to be for a different iOS device, which you don't, so choose \bxcaption{Duplicate Only}. 
\item The new target will be created. We suggest renaming it, e.g. to  \bxname{\gd{} or \jb{} Tests}.
\item You can  also (optionally) rename the new target from the default \bxname{MyApp copy} to something more meaningful e.g.  \bxname{MyApp (\gd{} or \jb{} Tests)} 
by selecting the \bxcaption{Build Settings} tab and searching for \bxcaption{Product Name}, then changing the value to a new name.
\end{enumerate}


\subsubsection{Configure the Testing Target}

Now that you have a target for your tests, add the tests to that target. 
\begin{enumerate}
\item  With the project settings still selected in the Project Navigator, and the new integration tests target selected in the project settings, select the \bxcaption{Build Phases} tab. 
\item Under the \bxcaption{Link Binary With Libraries} section, press the \bxcaption{+} button. 
\item In the sheet that appears, select \bxname{CFNetwork.framework} and \bxname{SenTestingKit.framework} and click \bxcaption{Add}.
\item Then click \bxcaption{Add other...} in the lower left corner and locate and select the library \bxname{librc.mobile.ios.nativ.a} and click \bxcaption{Open}.
\item Next, make sure that the  \bxname{UIRemoteControl.h} header file can be accessed. To do this, add the \bxname{UIRemoteControl.h} to the \bxcaption{Header Search Paths} 
build setting. Start by selecting the \bxcaption{Build Settings} tab of the project settings, and from there, use the filter control to find the 
\bxcaption{Header Search Paths} setting. 
\item Double click the value, and add the file \bxname{UIRemoteControl.h} to the list. If it's not there already, you should 
add the \$(inherited) entry as the first entry in this list.
\item  The iOS support takes advantage of Objective C's ability to add categories to an object, but this isn't enabled for static libraries by default. 
To enable this, add the -ObjC and -all\_load flags to the \bxcaption{Other Linker Flags} build settings.
\item Finally, add a preprocessor flag to the testing target so that you can conditionally include code. This will help to make sure that none of the 
testing code makes it into the production app. Call the flag \bxshell{RUN\_FUNCTIONAL\_TESTS} and add it under the \bxcaption{Preprocessor Macros}. Again, make 
sure the \$(inherited) entry is first in the list.
\end{enumerate}

\subsubsection{Add hook into the \gdaut{}}

Finally, the app needs a hook so that it actually allows the attachment and running of the tests when executing the Tests target. 
This is achieved by using the  \bxname{RUN\_FUNCTIONAL\_TESTS} macro that was defined in the preceding section. This \bxcaption{preprocessor macro} is only defined in 
the testing target, so the remote controlling won't be possible in the regular target. To allow your \gdaut{} to be remote controlled, add the 
following code to your application delegate:

\begin{verbatim}
...
#if RUN_FUNCTIONAL_TESTS
#import "UIRemoteControl.h"
#endif
...
\end{verbatim}

and the following code to the end of its - (void)applicationDidFinishLaunching[withOptions]: method

\begin{verbatim}
...
#if RUN_FUNCTIONAL_TESTS
    [UIRemoteControl attach];
    // alternatively you can
    // allow the UIRemoteControl 
    // to use a specific port number 
    // on the iOS device 
    // by using:
    // 
    // [UIRemoteControl attach:<portNo>];
    // 
    // this is necessary
    // e.g. when you're running 
    // different AUTs in parallel on 
    // the same iOS device
#endif
...
\end{verbatim}

You also have to make sure that your application delegate provides a UIWindow property called \bxname{window} which contains the window used to present the app's visual content on the device's main screen.

\begin{verbatim}
...
#if RUN_FUNCTIONAL_TESTS
    @property(nonatomic, retain) 
    UIWindow *window
#endif
...
\end{verbatim}

Everything should now be configured. When you run the \gdaut{} tests target it will launch your app and allow the \ite{} to remotely attach (on the port specified, or on 11022 if none is entered) and execute tests.

\bxwarn{If you do not follow the above steps, the \gdagent{} will not be able to communicate with your \gdaut{}!}

This documentation is derived from the KIF installation documentation (http://github.com/square/KIF) as we make use of KIF internally.
