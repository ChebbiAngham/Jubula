\index{Design for Testability!RCP}
\index{RCP!Set Data}
\index{Set Data}
\index{RCP!Design For Testability}

\subsubsection{Naming components}
Although  components  can be located in the \gdaut{} even when they are not named by the developers, naming components is nevertheless a good idea. In SWT and RCP there is no method like the Swing \bxname{setName} method to name components in the program code. However, you can improve the testability of your application by using the following method in your SWT or RCP code for the current component class: \bxname{setData(String key, String ComponentName)}. For the key, use \bxname{TEST\_COMP\_NAME}. 

Even if you do not name components, you can choose to have unique names generated for your components in the \gdaut{} in the \gdaut{} dialog \bxpref{Defineaut}. 

\subsubsection{Adding support for text retrieval}

You can add support for renderers for SWT components or items  without the getText() method in order to access text that is otherwise non-readable during test execution. 

Use the method \bxshell{setData(String key, Object value)} on the instance of the component or item whose text you want to access. The \bxname{key} is \bxshell{TEST\_TESTABLE\_TEXT}

For example, to access an otherwise unreachable text on a label:\\
\bxshell{myLabel.setData("TEST\_TESTABLE\_TEXT", label);}

If you are making text in e.g. a table accessible, then you will need to add a dynamic part for the column, e.g.:\\
\begin{verbatim}
myTableItemInstance.setData
("TEST\_TESTABLE\_TEXT\_" + colIdx, text);
\end{verbatim}
