% BREDEX LaTeX Template
%  \documentclass is either ``bxreport'' or ``bxarticle''
%                 option is bxpaper
%% \documentclass{bxarticle}
%% % ----------------------------------------------------------------------
%% \begin{document}
%% \title{}
%% \author{}
%% % \author*{Hauptautor}{Liste der Nebenautoren}
%% \maketitle
%% % ----------------------------------------------------------------------
%% \bxversion{0.1}
%% %\bxdocinfo{STATUS}{freigegeben durch}{freigegeben am}{Verteilerliste}
%% \bxdocinfo{DRAFT}{}{}{}
%% % ----------------------------------------------------------------------

%% \end{document}
\index{Constants!Value Separator}
\index{Constants!Path Separator}
\index{Constants!Indices}
\index{Value Separator}
\index{Path Separator}
\index{Indices}
Many of the actions use parameters for which you need to enter a ''path'' (e.g. menupaths, a path through a tree etc.) or a ''list'' (e.g. to select multiple items from a list or combo box).

The following are the characters you must use to enter lists and paths:

\begin{description}
\item [Path Separator]{Hierarchical components (e.g. menus, trees) use slash  {\tt '/'} as a value separator. An example menupath is therefore: 
\begin{quote}
{\tt File/Open}
\end{quote}
An example indexpath could be:
\begin{quote}
{\tt 1/2}
\end{quote}
}
\item[Value Separator]{For components which allow more than one item to be selected or checked at once, use a comma {\tt ','} to separate items. For example, to select the values ''blue'', ''green'' and ''orange'' from a list, the parameter would look like this:

\begin{quote}
{\tt blue,green,orange}
\end{quote}
}
\end{description}

\bxtipp{Do not enter a space between list items.}

If you are testing a component whose actions use these separators, and you need to enter a comma or slash as a part of the actual parameter, you will need to mask the comma or slash \bxpref{specialchar}.
\section{Parameters which use Indices}
All parameters which use indices (e.g. selecting from a list, or entering a menupath by the index) start at index {\tt '1'}. For example, the  first row in a table is indexed with {\tt '1'}. 


