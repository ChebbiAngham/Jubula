% $Id: Regex.tex 11851 2010-08-17 14:48:12Z alexandra $
% Local Variables:
% ispell-check-comments: nil
% Local IspellDict: american
% End:
% --------------------------------------------------------
% User documentation
% copyright by BREDEX GmbH 2005
% --------------------------------------------------------
\index{Regular Expressions}
% a helper macro to make things easier
% to describe a single element of regex syntax
\newcommand{\gdregex}[3]{%
#1 & #2 & #3 \\%
}%

\newcommand{\bxcarat}{\verb+^+}
\begin{itemize}
\item Using regular expressions lets you verify or select text even if you do not know exactly what the text is.
\item Using a system of placeholders for characters and some function symbols, you can enter a regular expression to find or check a text.
\item Actions which support regular expressions have an additional parameter, \bxcaption{Operator}. From this combo box, you can choose \bxcaption{matches}  to indicate that you want to use regular expressions.
\end{itemize}

\textbf{Simple matching}
\begin{itemize}
\item \bxshell{abc} matches \bxcaption{abc} and nothing else.
\end{itemize}

\textbf{Wildcards}
\begin{itemize}
\item \bxshell{.} represents one instance of any character.
\item \bxshell{..} represents two characters.  
\end{itemize}

\textbf{Repetition}
\begin{itemize}
\item Instead of using \bxshell{.} for each character, you can use symbols to indicate how many characters you are replacing with wildcards.
\item \bxshell{?} matches the previous character or group 0 or 1 time(s). E.g. \bxshell{a?} represents ''none or one \bxshell{a}''.
\item \bxshell{*} matches the previous character or group 0 or more time(s). E.g. \bxshell{a*} represents ''none or more \bxshell{a}''.
\item \bxshell{+} matches the previous character or group 1 or more time(s). E.g. \bxshell{a+} represents ''one or more \bxshell{a}''.
\item You can also use curly brackets \bxshell{\{\}}  to specify the minimum and maximum number of repetitions, e.g. \bxshell{\{4,7\}} looks for a minimum of four and a maximum of seven repetitions of the previous character.
\item To specify that a character must be repeated a minimum of four times, use: \bxshell{\{4,\}}
\end{itemize}

\textbf{Combining wildcards with repetition}
\begin{itemize}
\item You can specify a whole area of unknown text using \bxshell{.} and one of the repetition methods.
\item  \bxshell{.*} is an unlimited amount of any character, or none at all.
\item \bxshell{.?} is 0 or one of any character.
\item \bxshell{.+} is an unlimited amount of any character, but the character must appear at least once. 
\end{itemize}
\bxtipp{\app{} applies your regular expression to your entire string. To search for a match within a string, wildcards need to be placed on either side. See the examples below for more information.}


\textbf{Ranges}
\begin{itemize}
\item For each individual character, you can specify a range of things it is allowed to be. 
\item A range is specified using square brackets (\bxshell{[]}) and a dash \bxshell{-}. 
\item For example, you can specify that a particular character can be any capital letter: \bxshell{[A-Z]}.
\end{itemize}
\bxtipp{Note that there are no spaces between the ranges. }

\textbf{Alternatives}
\begin{itemize}
\item Use a pipe ('{\tt |}') to specify alternatives.
\item For example, \bxshell{[a|b].*} will match a string that begins with \bxshell{a} or \bxshell{b}.
\end{itemize}

\textbf{Escape character}
\begin{itemize}
\item Backslash \bxshell{$\backslash$} is used to negate the effect of the character following the backslash.
\item The characters that are used to construct a regular expression need to be escaped if they are to be matched within a string.
\item The characters are: \newline
 \verb1 [ ] \ . | ? * + ( ) { } ^ $  1\newline 
\item Because \app{} already uses a backslash as an escape symbol, you will need to use two backslashes to escape regular expression characters. 
\item For example, to check for a tree node:\\
\bxshell{x/y/z/***} \\
where the slashes are a part of the node, your regular expression in \app{}  would look like this: \verb1 x\/y\/z\/\\*\\*\\* 1\newline
The backslashes before the ordinary slashes are an escape symbol to tell \app{} that the following sign is not a path separator. The extra backslash before the stars tells \app{} that the second backslash is to be interpreted as a backslash in the regular expression, i.e. as an escape symbol.  
\item E.g. If you want to check for a star (\bxshell{*}), then you have to enter \bxshell{$\backslash$$\backslash$*}. 
\end{itemize}

\textbf{Verbatim}
\begin{itemize}
\item You can avoid having to use multiple backslashes by putting the whole regular expression in single quotes:
\item The example above for a tree node could be entered thus:\\
\bxshell{'x/y/z/***'} 
\end{itemize}

\textbf{Useful examples}
\begin{itemize}
\item An empty field is represented by: \verb+^$+%$
\item A string that starts with \bxshell{a} is represented by: \bxshell{a.*}
\item A string that ends in \bxshell{a} is represented by: \bxshell{.*a}
\item A string that starts with \bxshell{a}, ends in \bxshell{b} and has unknown values (0 or more) in the middle is represented by: \bxshell{a.*b}
\item A string which contains \bxshell{a} somewhere between other unknown characters (0 or more) is represented by: \bxshell{.*a.*}
\item A password which can only contain capital letters and which must be between six and eight letters is represented by: \bxshell{[A-Z]\{6,8\}}.
\item A password which can contain any alphanumerical values and which must be at least six characters is represented by: \bxshell{[A-Za-z0-9]\{6,\}}.
\item You can check that a string corresponds to a minimum and maximum length using: \\
\verb#'^.{'={MIN_COUNT}','={MAX_COUNT}'}$'#
\item To check for a text which begins with a \bxshell{*}, you must use the escape character: \verb+\\*.*+
\end{itemize}

 
%% %In order to allow for powerful verification of textual data, \app{}
%% %offers support for regular expressions conforming to the  
%% %\bxname{Perl 5} standard. In the actions which support regular
%% %expressions, there is an additional parameter which allows the user to
%% %activate/deactivate this feature.

%% %The following basic syntax can be applied for text verification with
%% %regular expressions:\bigskip

%% %% \begin{tabular}{p{1.5cm}p{5cm}p{2.5cm}}
%% %%   Syntax & Description & Example \\ \hline
%% %%   \app{}efregex{letter, digit, etc.} {Except for a few special characters, \newline
%% %%     {\tt [ ] $\backslash$ . | ? * + ( ) } \newline
%% %%     entering any character will match itself. The above special
%% %%     characters are escaped with a backslash, {\tt $\backslash$ }. }
%% %%     {\bxshell{abc} matches \bxshell{abc}.
%% %%       }
%% %% \end{tabular}

%% \begin{description}
%%   \item[\textbf{simple matching}]
%%   To match a string exactly to that in a text field, simply type it in:

%%   \bxshell{abc} matches \bxshell{abc}, and nothing else.

%% %%   To specify that a string should appear at the beginning or end of a
%% %%   text, use a carat ('\verb.^.') or dollar sign ('{\tt \$}'), respectively:

%% %%   ''\verb.^.abc'' matches \bxshell{abcxxx} but not
%% %%   \bxshell{xabcxx}.
  
%% %%   \bxshell{abc\$} matches \bxshell{xxxabc} but not \bxshell{xxabcx}.
%%   \item[\textbf{character matching}]
%%   Use brackets ('{\tt [}' and '{\tt ]}') to specify a range of allowed
%%   characters at the current location. Ranges may also be given with a
%%   minus sign ('{\tt -}'),
%%   or alternatives with a ''pipe'' ('{\tt |}'). A period ('{\tt .}')
%%   may be used as a wildcard (instead of brackets) to match any
%%   single character.

%%   \bxshell{[0-9A-F]} matches the numbers zero through nine and the
%%   capital letters '{\tt A}' through '{\tt F}' but nothing else.

%%   \bxshell{a.b} matches \bxshell{azb} and \bxshell{a6b} but not
%%   \bxshell{ab}.

%%   Characters are grouped together using parentheses ('{\tt (}' and
%%   '{\tt )}').

%%   \item[\textbf{wildcards/repetition:}]
%%   Use '{\tt ?}' to match a character or group 0 or 1 time(s), '{\tt
%%   *}' to match 0 or more times, and '{\tt +}' to match 1 or more
%%   times. Numbers enclosed in brackets
%%  ('{\tt \{}' and '{\tt \}}') may be used  to specify the minimum and maximum
%%   repetitions. The expression

%%   ''\verb1abc.+xyz1''

%%   matches \bxshell{abcfghixyz} but not \bxshell{abcxyz}.

%%   Likewise, the expression

%%  ''\verb/[a-zA-Z][a-zA-Z0-9_\-]{4,7}/'' 

%%   may be used to
%%   verify a password format which accepts letters, numbers, '\verb/_/',
%%   and '{\tt -}', with a length between 5 and 8, where the first character
%%   must be a letter (note that '{\tt -}' is preceded by a backslash as
%%   an escape character, as it is normally used to specify range).

%% \end{description}

For additional information about syntax and usage of regular
expressions in general and in \app{} in particular, please consult one
of the many textbooks on the subject.
%\bxcomment{MSH}{...or visit http://some.regex.site}
%\bxcomment{MSH}{Exactly what standard do these conform to?}
