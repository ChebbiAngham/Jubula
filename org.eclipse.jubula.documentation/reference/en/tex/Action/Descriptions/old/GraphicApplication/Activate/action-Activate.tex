% CAP description for Application --> Activate
\label{activate}
\begin{itemize}
\item When the \ite{} and \gdagent{} are running on the same computer, you may need to bring the \gdaut{} into focus before a test can begin.
\item Exactly how to \bxname{activate} the application can vary from one system to another.
\item This action offers various different ways of activating the application.
\item \app{} realises this action by clicking in the location indicated in the \bxname{activation method} parameter.
\item In the \gdaut{} configuration \bxextref{\gduserman}{user,activation}, the default is:
activate the application. The activation method is automatically set to \bxcaption{click in titlebar}.
\item You can set an activation method for the whole \gdaut{} in  its configuration, and use this in your \gdsteps{} (e.g. set the activation method parameter to \bxcaption{use default}) or you can use a different activation method.
\item In this way, you can centrally define a default for a given operating system, which you can easily change when you test on another operating system.
\item Using the activate action, you can also specify activation methods which differ from the default.

 
\end{itemize}



  
