% CAP description for Tree --> Move --> Direction
\begin{itemize}
\item Use this parameter to define the direction you want to move in. 
\item Combined with the node count, you can move any number of nodes in any direction.
\item The following directions can be used:
\begin{itemize}
\item ''next'' searches only through sibling nodes (nodes that share a parent with the currently selected node). This search begins with the sibling node directly below the currently selected node and continues downward. If the search proceeds past the bottom sibling, the component is not found.
\item ''up'' progressively searches parent nodes. This means that the first node found is the parent of the currently selected node. The second node found is the parent of that node, and so on. If the search proceeds beyond the root node, the component is not found.
\item ''down'' progressivly searches first children nodes. This means that the first node found is the top child of the currently selected node. The second node found is the top child of that node, and so on. If the search proceeds beyond a node that has no children, the component is not found.
\item ''previous'' searches only through sibling nodes (nodes that share a parent with the currently selected node). This search begins with the sibling node directly above the currently selected node and continues upward. If the search proceeds past the top sibling, the component is not found.
\end{itemize}
\end{itemize}
