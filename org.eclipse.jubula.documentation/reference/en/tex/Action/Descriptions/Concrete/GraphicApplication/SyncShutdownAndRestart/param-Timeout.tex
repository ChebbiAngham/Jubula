% CAP description for Application --> Synchronize termination and re-start of AUT --> Timeout in ms
\begin{itemize}
\item Specify the timeout in milliseconds that represents the maximum amount of time that should be waited for the \gdaut{} to terminate (to de-register from the \gdagent{}). You should specify a time that represents how long your \gdaut{} takes to close.
\item If, after closing, your \gdaut{} requires time to free up resources, save settings etc., then you can use the environment variable \bxname{TEST\_AUT\_POST\_DEREGISTRATION\_DELAY}  for the environment the \gdagent{} is running in \bxextref{\gduserman}{user,TasksSystemVariables} or as a JVM property in the \gdaut{} configuration \bxextref{\gduserman}{user,ExpertAUTConfig} to add a delay to ensure that these steps are completed successfully. 
\item If the termination timeout is exceeded, and the \gdaut{} is still registered, then a normal \bxname{restart} \gdaut{} will be executed and this \gdstep{} will be marked as failed.  
\end{itemize}

\bxtipp{The timestamp for this \gdstep{} will usually be longer than the timeout specified, due to internal actions that are necessary to synchronize the termination and re-start.}

\bxwarn{This action is currently supported on the Swing and RCP toolkits.}
