% BREDEX LaTeX Template
%  \documentclass is either ``bxreport'' or ``bxarticle''
%% %                 option is bxpaper
%% \documentclass{bxarticle}
%% % ----------------------------------------------------------------------
%% \begin{document}
%% \title{}
%% \author{}
%% % \author*{Hauptautor}{Liste der Nebenautoren}
%% \maketitle
%% % ----------------------------------------------------------------------
%% \bxversion{0.1}
%% %\bxdocinfo{STATUS}{freigegeben durch}{freigegeben am}{Verteilerliste}
%% \bxdocinfo{DRAFT}{}{}{}
%% % ----------------------------------------------------------------------

%% \end{document}
This manual provides important information about the more technical side of working with \gd. In the following sections, you will find:
\begin{itemize}
%\item a list of the keyboard shortcuts available in \app{}\bxpref{scut}.
\item an introduction to using regular expressions as parameters \bxpref{regex}.
\item an introduction to using simple matches in your parameters \bxpref{simplematch}
\item a reference of all supported actions and their parameters \bxpref{actparam}.
\item details about the constants used to enter paths and list items \bxpref{constants}. 
\item a guide to using the abstract components offered by \app{}\bxpref{overviewfam}.
\item information on relative paths in \app{} \bxpref{relativepath}.
\item a list of the special characters in \app{} \bxpref{specialchar}.
\item a table of language codes \bxpref{langcodes}.
\item information on keyboard layout files \bxpref{keyboardlayout}.
%\item a list of common icons used in \app{}\bxpref{icons}.
\item instructions on how to remotely debug your \gdaut{} with \app{} \bxpref{debugging}.
%\item \app{}error messages \bxpref{errormsgs}.
\end{itemize}
