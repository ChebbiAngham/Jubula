%% Once \jb{} has been installed on your system as described in previous
%% sections of this document, it can be set up to run as
%% a plugin within the Eclipse environment. The following
%% sections describe the steps required to do this.

%% \bxwarn{In order to run \jb{} as an \bxname{Eclipse} plugin,
%% \bxname{Eclipse} must be started using \bxname{Java} 1.5 or
%% higher. We support Eclipse 3.3 and 3.4. Please see the \bxname{Eclipse} documentation for details on how to specify the Java VM.}

%% \section{Required plugins to run \jb{} in Eclipse}
%% A list of required plugins to run \jb{} in your Eclipse environment is available on request. 

%% \section{Installing the plugin for Eclipse 3.3}
%% \begin{enumerate}
%% \item Locate the archive named \bxname{gd-plugins.zip} in your \jb{} installation directory. 
%% \item Unzip the contents of this file into the \bxname{Eclipse/plugins} directory. 
%% \item Configure the plugin installation as described below \bxpref{confplugin}.
%% \end{enumerate}

%% \subsection{Installing the plugin for a single user in Eclipse 3.3}

%% If you do not have administrative privileges for the computer on which
%% \jb{} is installed, or only wish to install the plugin for a single
%% user, follow the steps below.

     
%% \begin{enumerate}
%% \item Copy the \bxname{plugins} folder from within the gd-plugins.zip to a place where you have administrative privileges. Put them in the  \bxcaption{Eclipse/plugins} folder. The \bxcaption{Eclipse} folder must also contain a \bxcaption{features} folder, which can be empty, and an \bxshell{.eclipseextension} file. 
%% \item In Eclipse, go to:\\ \bxmenu{Help}{Software updates}{Manage configuration}.
%% \item In the window which opens, right-click on the Eclipse node and select:\\
%% \bxmenu{Add}{Extension location}{}.
%% \item Browse to the directory \emph{above} the Eclipse directory  select \bxcaption{OK}.
%% \item Configure the plugin installation as described below \bxpref{confplugin}.
%% \end{enumerate}

%% \section{Installing the plugin for Eclipse 3.4}
%% \begin{enumerate}
%% \item Locate the archive named \bxname{gd-plugins.zip} in your \jb{} installation directory. 
%% \item Unzip the contents of this file into the \bxname{Eclipse/dropins} directory. 
%% \item Configure the plugin installation as described below \bxpref{confplugin}.
%% \end{enumerate}


%% \section{Configuring the plugin}
%% \label{confplugin}
%% The Eclipse plugin can be configured using the configuration tool. 
%% \begin{enumerate}
%% \item Open the configuration tool from:\\ \bxmenu{\jb{}}{Admin}{GDConfiguration}.
%% \item When prompted, enter the path to the Eclipse folder where you unzipped the plugin files. 
%% \item You can configure the plugin in the same way as you would configure the standalone version \bxpref{custInst}. 
%% \item If you want to use the same license server as the standalone client, enter the same hostname and port number for the license server as you did in the configuration for the standalone version. 
%% \item If you want to use the same database, enter the same details in this configuration as you did previously. To use a different database, enter the new details. 
%% %Make sure you run the CreateDB tool for the plugin to create the new database. 
%%  \end{enumerate}


