The \ite{} is tested with Oracle and we recommend using an Oracle or Oracle Express \gddb{} for working with it. 

\subsection{Some tips for working with Oracle Express}

\textbf{Problems creating the \gddb{} scheme: DBA rights}\\

There is a known issue with Oracle Express when creating \gddb{} schemes with DBA-rights. In some cases, the creation of the \gddb{} scheme may fail.
To avoid this problem, do not use DBA-rights when creating the \gddb{} scheme.
 
\textbf{Increasing sessions and processes}\\

Oracle Express uses a relatively small amount of sessions. Insufficient sessions can lead to problems when working with the \ite{}. To combat this problem, the sessions and processes in Oracle Express should be set higher. 

We recommend 100 as a suitable amount. The sql script to do this looks like this:
\begin{verbatim}
sqlpls / nolog
connect / as sysdba
alter system set sessions=100 scope=spfile;
alter system set processes=100 scope=spfile;
quit
\end{verbatim}

You must run the script as an administrator and restart the
database once the script has run.


\textbf{UTF-16 support}\\
If you want to test \gdauts{} which run in languages such as
Japanese, you will need the universal edition of Oracle Ex-
press, which supports UTF-16 character encoding. The ISO-
8859 edition of Oracle Express does not support Japanese
(and similar) characters.

\textbf{User roles}\\

When creating users for the database, bear in mind that each
user must have the roles \bxname{connect} and \bxname{resource} to be able to
work with the database.
