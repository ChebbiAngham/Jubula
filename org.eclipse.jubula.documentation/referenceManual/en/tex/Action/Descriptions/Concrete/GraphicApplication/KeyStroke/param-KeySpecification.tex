% CAP description for Application --> Key Combination --> Base Key
\begin{itemize}
\item Use the \bxcaption{base key} parameter to specify which key to ''press''. 
\item The various keys have different codes.
\item The most important keycodes are:
\end{itemize}

\begin{footnotesize}
\begin{tabular}{l|l}
  \textbf{value(s)} & \textbf{description} \\\hline
0 ... 9 & top-row keys 0 through 9 \\
A ... Z & letters A to Z. For capital letters,\\
&use \bxshell{shift} in the \bxcaption{Modifier} parameter. \\
ENTER & Enter or Return key\\
SPACE & the Spacebar\\
TAB & the Tab key\\
ESCAPE & the Escape key\\
BACK\_SPACE & the Backspace key\\
F1 ... F12 & the function keys\\
HOME, END & the home and end keys\\
&(not the number pad keys!)\\
INSERT, DELETE & the insertion and deletion keys\\
&(not the number pad keys!)\\
PAGE\_UP, PAGE\_DOWN & the page up and page down keys\\
&(not the number pad keys!)\\
DOWN, UP & the Up and Down arrow keys \\
&(not the number pad keys!)\\
LEFT, RIGHT & the Left and Right arrow keys \\
&(not the number pad keys!)\\
NUMPAD0 ... NUMPAD9 & The number pad keys. \\
\end{tabular}
\end{footnotesize}


%% \begin{itemize}
%% \item Use the \bxcaption{base key} parameter to specify which key to ''press''. 
%% \item The various keys have different codes.
%% \item The most important keycodes are:
%% \end{itemize}

%% \begin{footnotesize}
%% \begin{tabular}{l|l}
%%   \textbf{value(s)} & \textbf{description} \\\hline
%% 0 ... 9 & top-row keys 0 through 9 \\
%% A ... Z & letters A to Z. For capital letters,\\
%% &use \bxshell{shift} in the \bxcaption{Modifier} parameter. \\
%% ENTER & Enter or Return key\\
%% SPACE & the Spacebar\\
%% TAB & the Tab key\\
%% ESCAPE & the Escape key\\
%% BACK\_SPACE & the Backspace key\\
%% F1 ... F12 & the function keys\\
%% HOME, END & the home and end keys\\
%% &(not the number pad keys!)\\
%% INSERT, DELETE & the insertion and deletion keys\\
%% &(not the number pad keys!)\\
%% PAGE\_UP, PAGE\_DOWN & the page up and page down keys\\
%% &(not the number pad keys!)\\
%% DOWN, UP & the Up and Down arrow keys \\
%% &(not the number pad keys!)\\
%% LEFT, RIGHT & the Left and Right arrow keys \\
%% &(not the number pad keys!)\\
%% NUMPAD0 ... NUMPAD9 & The number pad keys. \\
%% \end{tabular}
%% \end{footnotesize}




