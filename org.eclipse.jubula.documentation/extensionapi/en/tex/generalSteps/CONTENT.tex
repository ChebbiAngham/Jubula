\chapter{Introduction}
\label{introduction}

\app{} starts, controls, and observes \gdauts using its server
component. In order for the \gdagent to know how to control each
element of a GUI, we have outfitted the \gdagent with a pluggable interface
for graphic components. Using this interface, each component that \app{}
can test is described in a
so-called \textbf{\gdimplclass{}}. Each GUI toolkit that \app{} supports
is described in a toolkit plugin.

Because of the great flexibility that user customization allows, we
have opened up this interface to our users. You can add to the existing functionality of our officially-supported
\gdimplclasses{},
or provide support for in-house graphic components by
defining \gdimplclasses of your own, which we will refer to as
\textbf{\gdtesterclasses}.

\section{What does a \gdtesterclass look like?}

The functionally important aspect of a \gdtesterclass is that it
contains a public method for each \app{} test action which will appear in
the client. These methods are linked to testable actions within a user-defined
\app{} Plugin, which is described later in this handbook. Each plugin
provides a XML configuration file,
which tells \app{} which method to call, what parameters it needs to
send, as well as string externalization information.

\chapter{General Steps to take}
\label{generalSteps}

The following chapter describes the general steps to take when extending \app{} with
custom defined components and actions. Detailed information for each step can be found in the
corresponding example extension files in 
''InstallationDirectory/examples/development/extension/src''

In general you always have to extend two parts of \app{}
\begin{itemize}
\item The \app{} Client by writing your own \bxshell{Toolkit-Plugin}:\\
      This will tell the client which new components and actions are available.
\item The \app{} Server Component by putting a compiled class file to a specific directory:\\
      This is the part which is acutally performing the actions on the new component.
\end{itemize}

\section{Requirements}
To create your own \app{} extension, you need:
\begin{itemize}
\item \app{} 2.0 or later
\item Eclipse 3.4 or later
\item JDK 5.0
\end{itemize}

You must also have write access in the directories:\\
InstallationDir$\backslash$\app{}$\backslash$plugins\\
InstallationDir$\backslash$server$\backslash$lib$\backslash$extImplClasses 

\section{\app{} Client Extension}

The following steps have to be done to extend the \app{} client:
\begin{enumerate}
 \item Create an eclipse plug-in project and a corresponding feature project.
 \item Set ''InstallationDir$\backslash$\app{}$\backslash$plugins'' as your target platform
 \item Define plugin dependencies to the toolkit support plugin
 \item Enter the toolkit support plugin in your toolkit plugin project
 \item Create a MyToolkitProvider class
 \item Define and configure the toolkit extension at the extension point
 \item Create a myComponentExtension.xml
 \begin{itemize}
  \item Extend existing \app{} components with new actions
  \item Derive components from existing \app{} components
  \item Define a new component
 \end{itemize}
 \item Manage the i18n keys
 \item Export the toolkit feature to \app{}
\end{enumerate}

Under ''InstallationDirectory/examples/development/extension/src''
you will find a \\ ''eclipseProjects\_ExampleSwingClientExtension.zip'' which contains
 an example \app{} Client Extension for the Swing component ''JSlider'' as well
 as a corresponding feature project. These projects are a showcase for steps 1-8.
 
\subsection{Exporting the toolkit plugin to \app{}}
 The only steps you have to do after importing the projects into your Eclipse workspace and 
 setting the target platform (step no. 2) is to export the feature to an update site
 and then use the update site to install the feature into the \app{} you want to 
 extend.
 
To export the toolkit feature to an update site:

\begin{enumerate}
\item In the package explorer, right click on the feature project 
 (ex. \bxname{org.eclipse.jubula.examples.extension.swing.feature}) and select: 
 \bxname{Export...} from the context-sensitive menu.
\item In the dialog that appears, select \bxname{Deployable features} and click \bxcaption{Next}.
\item In the next dialog, in the \bxname{Available Features} area, ensure that 
 the checkbox next to the feature you wish to export is selected.
\item In the \bxname{Destination} tab, enter the location 
 where the feature's update site should be exported to in the \bxname{Directory} 
 field. This can be any writable directory. This directory will serve as an 
 update site, which can later be used to install your feature into \app{}.
\item In the \bxname{Options} tab, ensure that the \bxname{Package as individual JAR archives} checkbox is selected and click \bxcaption{Finish}.
\end{enumerate}

To install the toolkit feature from the update site:

\begin{enumerate}
\item Start \app{} and select \bxmenu{Help}{Install new software...}{} from the main menu.
\item In the \bxname{Install} dialog that appears, click the \bxname{Add...} button.
\item In the \bxname{Add Repository} dialog that appears, click the \bxname{Local...} button.
\item In the file selection dialog that appears, navigate to the directory that contains your update site and confirm the dialog.
\item Click \bxname{OK} to exit the \bxname{Add Repository} dialog. 
 The active dialog should now be \bxname{Install}.
\item Ensure that the \bxname{Group items by category} checkbox is deselected.
 Your feature should be visible in the central table of the dialog.
\item Ensure that the checkbox next to your feature is selected and click \bxname{Next}.
\item Confirm the \bxname{Installation Details} by again clicking \bxname{Next}.
\item Accept the license agreement terms and click \bxname{Finish}.
\item A warning dialog may appear to warn of unsigned content. Click 
 \bxname{OK} in this dialog if the feature comes from a trusted source (ex. 
 if you have written the software yourself, as in this example). This will
 begin installation.
\item After installation, a dialog appears suggesting that \app{} be restarted
 in order to safely finish the update / installation. Click \bxname{Restart Now}
 to perform the restart. Once the restart completes, your extension feature has
 been installed in \app{}.  
\end{enumerate}

\section{\app{} Server Extension}

\gdauts{} are controlled by the server. There exists a \gdtesterclass for each component that
\app{} supports. This class implements the test actions that can be carried out on the component, To add your
component to \app{}, you need to write a \gdtesterclass for it.

Please follow the following guidelines for your \gdtesterclasses:
\begin{itemize}
  \item Your build path must contain the following three JAR files:
  \bxshell{org.eclipse.jubula.rc.swing.jar}, \bxshell{org.eclipse.jubula.rc.common.jar}, and
  \bxshell{org.eclipse.jubula.tools.jar}, which contain our server classes and some utility classes. They are located in your \app{} installation directory under
  \bxshell{server/lib}.
  \item The class must be compatible with \bxname{Java 1.4}.
  \item Its declared package name must begin with: \\
    ''org.eclipse.jubula.rc.swing.swing.implclasses''
  \item It must implement the following interface: \\
    ''org.eclipse.jubula.rc.\\.swing.swing.implclasses.IImplementationClass''
  \item It must provide \textbf{public} methods for each action that is implemented for the component.
  \item Each method that implements an action must throw the following exception upon error:
    org.eclipse.jubula.rc.common.exception.StepExecutionException
    This way, \app{} will be able to know that an action has failed.
\end{itemize}

Under ''InstallationDirectory/examples/development/extension/src''
you will find a \\ ''eclipseProjects\_ExampleSwingServerExtension.zip'' which contains
 an example \app{} Server Extension for the Swing component ''JSlider''.

Now that you have written your \gdtesterclass, you still need to
make \app{} aware of its presence. This is done by first putting the
compiled class in a
location where \app{} can find it, then by altering the configuration
file, so that \app{} knows which component it refers to, and how it can
be used. The following sections explain how this is done.

\section{Where to put your \gdtesterclasses}
If you extend the SwingToolkitPlugin, your compiled \gdtesterclasses (\bxshell{*.class} files) need to
be placed in your \app{} installation directory under the following path:

\bxshell{server/lib/extImplClasses} ,

using a directory structure that corresponds to the fully-qualified
package name. Therefore, your \gdtesterclasses should be in some
sub-directory of the following path within your \app{} installation directory:

\texttt{server/lib/extImplClasses/org/eclipse\\
    /jubula/swing/swing/implclasses}

As \bxname{Eclipse} already stores compiled classes according to this
structure, you need only (recursively) copy the directory structure,
starting with \bxshell{org}, into the extImplClasses directory.

Alternatively, you may place a JAR containing the above structure
(also starting with \bxshell{org}) into the \bxshell{extImplClasses}
directory.

\bxtipp{Please note that if the \textit{extImplClasses} directory does
not already exist, you must create it in the above-mentioned location.}

\section{\app{} Example Extension}
\app{} comes with a complete example extension implementation in source and binaries. 
This example extension extends \app{} for the Swing component ''JSlider''. After deploying the \app{} Client plug-in and the \app{} Server 
extension you should be able to test the Swing component ''JSlider'' at ''Graphics Component''-level with \app{}.
The example extension code and binaries can be found in the ''InstallationDirectory/examples/development/extension'':

\begin{itemize}
 \item AUT \\
 This directory contains a trivial example AUT which uses the originally unsupported component "JSlider". 
 After installing the extensions, your \app{} will be able to test this new component.
 \item src \\
 This directory contains several archive files which are all importable Eclipse projects:
 \begin{itemize}
 	\item \bxshell{eclipseProjects\_ExampleSwingAUT.zip} \\
 	This is the source code project for the example Swing AUT.
 	\item \bxshell{eclipseProjects\_ExampleSwingClientExtension.zip} \\
 	This is the source code project of the extension plug-in for the \app{} Client.
 	\item \bxshell{eclipseProjects\_ExampleSwingServerExtension.zip} \\
 	This is the source code project of the extension for the \app{} Server.
 \end{itemize}
 \item bin \\
 This directory contains the compiled sources as directly deployable units.
\end{itemize}
