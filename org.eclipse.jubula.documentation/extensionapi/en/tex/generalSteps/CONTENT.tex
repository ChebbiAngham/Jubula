\chapter{Introduction}
\label{introduction}

\jb{} starts, controls, and observes \gdauts using its server
component. In order for the \gdserver to know how to control each
element of a GUI, we have outfitted the \gdserver with a pluggable interface
for graphic components. Using this interface, each component that \jb{}
can test is described in a
so-called \textbf{\gdimplclass{}}. Each GUI toolkit that \jb{} supports
is described in a toolkit plugin.

Because of the great flexibility that user customization allows, we
have opened up this interface to our users. You are now able to add
and to the existing functionality of our officially-supported
\gdimplclasses{},
or provide support for in-house graphic components, by
defining \gdimplclasses of your own, which we will refer to as
\textbf{\gdtesterclasses}.

\section{What does a \gdtesterclass look like?}

The functionally important aspect of a \gdtesterclass is that it
contains a public method for each \jb{}test action which will appear in
the client. These methods are linked to testable actions within a user-defined
\jb{} Plugin, which is described later in this handbook. Each plugin
provides a XML configuration file,
which tells \jb{} which method to call, what parameters it needs to
send, as well as string externalization information.

\chapter{General Steps to take}
\label{generalSteps}

The following chapter describes the general steps to take when extending \jb{} with
custom defined components and actions. Detailed information for each step can be found in the
corresponding example extension files in 
''GUIdancerInstallationDirectory/examples/development/extension/src''

In general you have to always extends two parts of \jb{}
\begin{itemize}
\item The \jb{} Client by writing your own \bxshell{Toolkit-Plugin}:\\
      This will tell the clients which new components and actions are available.
\item The \jb{} Server Component by putting a compiled class file to a specific directory:\\
      This is the part which is acutally performing the actions on the new component.
\end{itemize}

\section{Requirements}
To create your own \jb{} extension, you need:
\begin{itemize}
\item \jb{} 2.0 or later
\item a license for the toolkit you want to extend
\item Eclipse 3.4 or later
\item JDK 5.0
\end{itemize}

You must also have write access in the directories:\\
guidancerInstallationDir$\backslash$guidancer$\backslash$plugins\\
guidancerInstallationDir$\backslash$server$\backslash$lib$\backslash$extImplClasses 

\section{\jb{} Client Extension}

The following steps have to be done to extend the \jb{} Client:
\begin{enumerate}
 \item Create an eclipse plug-in project
 \item Set ''guidancerInstallationDir$\backslash$guidancer$\backslash$plugins'' as your target platform
 \item Define plugin dependencies to the toolkit support plugin
 \item Enter the toolkit support plugin in your toolkit plugin project
 \item Create a MyToolkitProvider class
 \item Define and configure the toolkit extension at the extension point
 \item Create a myComponentExtension.xml
 \begin{itemize}
  \item Extend existing \jb{} components with new actions
  \item Derive components from existing \jb{} components
  \item Define a new component
 \end{itemize}
 \item Manage the i18n keys
 \item Export the toolkit plugin to \jb{}
\end{enumerate}

Under ''GUIdancerInstallationDirectory/examples/development/extension/src''
you will find a \\ ''eclipseProjects\_ExampleSwingClientExtension.zip'' which contains
 an example \jb{} Client Extension for the Swing component ''JSlider''. This project 
 is a showcase for steps 1-8.
 
\subsection{Exporting the toolkit plugin to \jb{}}
 The only step you have to do after importing this project into your eclipse workspace and 
 setting the target platform (step no. 2) is to export it to the \jb{} you want to 
 extend. To export the toolkit plugin to \jb{}:

\begin{enumerate}
\item In the package explorer, right click on the \bxname{plugin.xml} file and select:\\
\bxmenu{Open with}{plugin manifest editor}{}\\
from the context-sensitive menu.
\item Select the \bxname{build} tab from the editor.
\item In the binary build area, select \bxname{meta-inf, bin, build properties, plugin.xml} and \bxname{resources}.
\item Save the changes in the editor and close the editor. 
\item In the package explorer, right click on the project and select \bxname{export} from the context-sensitive menu.
\item In the dialog that appears, select \bxname{deployable plugins and fragments} and click \bxcaption{next}.
\item In the next dialog, in the \bxname{destination} tab, enter the location where the plugin should be exported to in the \bxname{directory} field:\\
\bxname{GUIdancerInstallationPath/guidancer}
\item In the \bxname{options} tab, deselect the checkbox which will export the plugin as individual jar archives and click \bxcaption{finish}.
\item You should be able to find the folder \\ \bxcaption{com.bredexsw.guidancer.examples.client.toolkitExtension\_1.0.0} in the \bxname{GUIdancerInstallationPath/guidancer/plugins/} folder. The exported plugin is in this folder. 
\item If you need to make changes and export the plugin again, delete the \\ \bxname{com.bredexsw.guidancer.examples.client.toolkitExtension\_1.0.0} from the \bxname{GUIdancerInstallationPath/guidancer/plugins/} folder first
\end{enumerate}

\section{\jb{} Server Extension}

\gdauts{} are controlled by the server. There exists a \gdtesterclass for each component that
\jb{}{} supports. This class implements the test actions that can be carried out on the component, To add your
component to \jb{}, you need to write a \gdtesterclass for it.

Please follow the following guidelines for your \gdtesterclasses:
\begin{itemize}
  \item Your build path must contain the following three JAR files:
  \bxshell{AUTServer.jar}, \bxshell{AUTServerBase.jar}, and
  \bxshell{org.eclipse.jubula.tools.jar}, which contain our server classes and some utility classes. They are located in your \jb{}{} installation directory under
  \bxshell{server/lib}.
  \item The class must be compatible with \bxname{Java 1.4}.
  \item Its declared package name must begin with: \\
    ''com.bredexsw.guidancer.autserver\\.swing.implclasses''
  \item It must implement the following interface: \\
    ''com.bredexsw.guidancer.autserver\\.swing.implclasses.IImplementationClass''
  \item It must provide \textbf{public} methods for each action that is implemented for the component.
  \item Each method that implements an action must throw the following exception upon error:
    com.bredexsw.guidancer\\.autserverbase.exception.StepExecutionException
    This way, \jb{}{} will be able to know that an action has failed.
\end{itemize}

Under ''GUIdancerInstallationDirectory/examples/development/extension/src''
you will find a \\ ''eclipseProjects\_ExampleSwingServerExtension.zip'' which contains
 an example \jb{} Server Extension for the Swing component ''JSlider''.

Now that you have written your \gdtesterclass, you still need to
make \jb{} aware of its presence. This is done by first putting the
compiled class in a
location where \jb{} can find it, then by altering the configuration
file, so that \jb{} knows which component it refers to, and how it can
be used. The following sections explain how this is done.

\section{Where to put your \gdtesterclasses}
If you extend the SwingToolkitPlugin, your compiled \gdtesterclasses (\bxshell{*.class} files) need to
be placed in your \jb{} installation directory under the following path:

\bxshell{server/lib/extImplClasses} ,

using a directory structure that corresponds to the fully-qualified
package name. Therefore, your \gdtesterclasses should be in some
sub-directory of the following path within your \jb{} installation directory:

\texttt{server/lib/extImplClasses/com/bredexsw\\
    /guidancer/autserver/swing/implclasses}

As \bxname{Eclipse} already stores compiled classes according to this
structure, you need only (recursively) copy the directory structure,
starting with \bxshell{com}, into the extImplClasses directory.

Alternatively, you may place a JAR containing the above structure
(also starting with \bxshell{com}) into the \bxshell{extImplClasses}
directory.

\bxtipp{Please note that if the \textit{extImplClasses} directory does
not already exist, you must create it in the above-mentioned location.}

\section{\jb{} Example Extension}
\jb{} comes with a complete example extension implementation in source and binaries. 
This example extension extends \jb{} for the Swing component ''JSlider''. After deploying the \jb{} Client plug-in and the \jb{} Server 
extension you should be able to test the Swing component ''JSlider'' at ''Graphics Component''-level with \jb{}.
The example extension code and binaries can be found in the ''GUIdancerInstallationDirectory/examples/development/extension'':

\begin{itemize}
 \item AUT \\
 This directory contains a trivial example AUT which uses the originally unsupported component "JSlider". 
 After installing the extensions, your GUIdancer will be able to test this new component.
 \item src \\
 This directory contains several archive files which are all importable Eclipse projects:
 \begin{itemize}
 	\item \bxshell{eclipseProjects\_ExampleSwingAUT.zip} \\
 	This is the source code project for the example Swing AUT.
 	\item \bxshell{eclipseProjects\_ExampleSwingClientExtension.zip} \\
 	This is the source code project of the extension plug-in for the \jb{} Client.
 	\item \bxshell{eclipseProjects\_ExampleSwingServerExtension.zip} \\
 	This is the source code project of the extension for the \jb{} Server.
 \end{itemize}
 \item bin \\
 This directory contains the compiled sources as directly deployable units.
\end{itemize}
