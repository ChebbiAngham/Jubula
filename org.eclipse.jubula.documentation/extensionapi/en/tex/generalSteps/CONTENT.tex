\chapter{Introduction}
\label{introduction}

\app{} starts, controls, and observes \gdauts using its remote control component. In
order for the \gdagent to know how to control each element of a GUI, we have
outfitted the \gdagent with a pluggable interface for graphic components. An
adapter factory containing one or more components must be implemented for and
deployed with the \gdaut. On the client site each GUI toolkit that \app{}
supports is described in a toolkit plug-in.

We have opened up an interface to our users to allow great and flexible
customizations. You can extend existing functionality, or provide support for
in-house graphic components by implementing your own so-called
\textbf{\gdtesterclasses.}

This handbook shows general steps for creating \app{} toolkit extensions, which
is described in chapter \ref{toolkitExtension}. In this chapter you will find
also an example and some hints for a migrating to the new API. Last but not
least chapter \ref{functions} describes how to create a custom \app{} function,
which can be used in \app{} tests.

\chapter{Creating \app{} toolkit extensions}
\label{toolkitExtension}

The following chapter describes the general steps to take for creating a \app{}
toolkit extensions with custom defined components and actions. This chapter
begins with listing the requirements followed by showing the necessary steps
for creating the \app{} client and remote control extension. The extension for the
\app{} client is described in section \ref{clientExtension} and the
implementation for the \app{} remote control extension in section \ref{serverExtension}.

Extension examples with source code can
be found in the sub directory\\
\bxshell{examples/development/extension/src/}\\
of a \app{} installation.

In general you always have to extend two parts of \app{}:
\begin{itemize}
\item The \app{} client by writing your own \textbf{toolkit} plug-in:\\
      This will tell the client, which new components and actions are available.
\item The \app{} remote control by writing your own \textbf{fragment}:\\
      This part actually performs the actions on the new components.
\end{itemize}

In this chapter normally all used directory names are located in the
installation directory of \app{}.

\section{Requirements}
To create your own \app{} extension, you need:
\begin{itemize}
\item \app{} 2.0 or later
\item Eclipse 3.4 or later
\item JDK 5.0
\item Write access in the following directories:\\
\bxshell{\MakeLowercase{\app{}}/plugins/}\\
\bxshell{server/plugins/}
\end{itemize}

\section{\app{} client extension}
\label{clientExtension}

This section shows how to develop the \app{} client extension. We describe the
steps to create the toolkit plug-in and how it can be exported into \app{}.

\subsection{Creating the toolkit plug-in for \app{}}

The following steps have to be done to extend the \app{} client:
\begin{enumerate}
 \item Create an Eclipse plug-in project and a corresponding feature project.
 \item Set \bxshell{\MakeLowercase{\app{}}/plugins/} as your
 target platform.
 \item Define the plug-in dependencies to the toolkit support plug-in.
 \item Enter the toolkit support plug-in in your toolkit plug-in project.
 \item Create your own toolkit provider class, e.g. named\\
 \bxshell{MyToolkitProvider}.
 \item Define and configure the toolkit extension at the extension point
 \bxname{org.eclipse.jubula.toolkit.common.toolkitsupport}.
 \item Create the definition of your own toolkit in the file\\
 \bxshell{ComponentExtension.xml} by choosing one or more of the following
 possibilities:
 \begin{itemize}
  \item Extend existing \app{} components with new actions,
  \item derive new components from existing \app{} components, or
  \item define a new component.
 \end{itemize}
 \item Manage the i18n keys in the properties file.
\end{enumerate}

You will find projects in zip-files as a showcase for steps 1-8, which are
located in the directory\\
\bxshell{examples/development/extension/src}.\\
Each zip-file contains an example for a specific toolkit, e.g. a Swing example
extends the \bxshell{JSlider} component.

These example projects can be simply imported into your Eclipse workspace. Set
the target after importing the projects as described in step 2. All other steps
have already be done in the examples.

\subsection{Deploying the toolkit plug-in in \app{}}

After creating the toolkit feature, we describe the steps to take for deploying
it in the \app{} client. At first we export the toolkit feature to an update
site and then we use the update site to install the feature into \app{}.
 
\subsubsection{Export the toolkit feature to an update site}

\begin{enumerate}
\item In the package explorer, right click on the feature project 
 (e.g. org.eclipse.jubula.examples.extension.swing.feature) and
 select \bxmenu{Export...}{}{}.
\item Select \bxname{Deployable features} and click
\bxmenu{Next}{}{} in the dialog that appears.
\item In the next dialog, ensure that the check box next to the feature you wish
to export is selected in the \bxname{Available Features} area.
\item In the \bxname{Destination} tab, select the \bxname{Directory}
 field to define the location, where the update site of the feature should be
 exported to. Ensure, that the entered location is a writable directory. This
 directory serves as an update site, which can later be used to install your
 feature into \app{}.
\item In the \bxname{Options} tab, ensure that the check box named
\bxname{Package as individual JAR archives} is selected and click
\bxmenu{Finish}{}{}.
\end{enumerate}

\subsubsection{Install the toolkit feature from the update site}

\begin{enumerate}
\item Start \app{} and select from the main menu\\
\bxmenu{Help}{Install new software...}{}.
\item Click the \bxmenu{Add...}{}{}
button in the \bxname{Install} dialog that appears.
\item Click the \bxmenu{Local...}{}{} button in the \bxname{Add Repository} dialog
that appears
\item Navigate to the directory, that contains your update site and confirm your
selection.
\item Click \bxmenu{OK}{}{} to exit the \bxname{Add Repository} dialog.
 The active dialog should now be \bxname{Install}.
\item Ensure that the check box named \bxname{Group items by category} is
deselected. Your feature should be visible in the central table of the dialog.
\item Ensure that the check box next to your feature is selected and click
 \bxmenu{Next}{}{}.
\item Confirm the \bxname{Installation Details} by clicking the \bxmenu{Next}{}{}
 button again.
\item Accept the license agreement terms and click \bxmenu{Finish}{}{}.
\item A warning dialog may appear to warn you installing unsigned content.
 Click \bxmenu{OK}{}{} in this dialog, if the feature comes from a trusted
 source. This should be the case as long as you install your own feature.
 Then the installation process begins.
\item Click the \bxmenu{Restart Now}{}{} button to perform a restart of \app{},
 when a dialog appears, which suggests restarting \app{} in order to safely
 finish the update / installation.  After the restart your \app{} extension
 feature is installed.
\end{enumerate}

\section{\app{} remote control extension}
\label{serverExtension}

\gdauts{} are controlled by the remote control called \gdagent. The \app{}
client is also executing the \app{} remote control, if the embedded agent is
used. Apart from that, the normal AUT agent runs as a remote control in a
separate process independently from the \app{} client.
For each supported component known to \app{} a \gdtesterclass must exist.
This class implements the test actions, that can be carried out on the component.
You need to write a fragment for your component, which contains the
\gdtesterclass and a corresponding adapter factory.

\subsection{Creating the fragment}

The following steps have to be done to extend the \app{} remote control:

\begin{enumerate}
  \item Extract the feature contained in the zip-file\\
    \bxshell{development/extension-toolkit-rc.zip}
  \item Set the extracted feature as your target platform.
  \item Create a fragment project with one of the following bundles as host:
    \begin{itemize}
      \item \bxshell{org.eclipse.jubula.rc.swing}\\
            for extending Swing support.
      \item \bxshell{org.eclipse.jubula.rc.swt}\\
            for extending SWT support without RCP environment.
      \item \bxshell{org.eclipse.jubula.rc.rcp.swt}\\
            for extending SWT support in Eclipse 3.x and 4.x (including compat
            layer).
      \item \bxshell{org.eclipse.jubula.rc.rcp.e3.swt}\\
            for extending GEF support in Eclipse 3.x (without compat
            layer).
   \end{itemize}
  \item Write your own adapter and \gdtesterclasses.
\end{enumerate}

Adapters are used throughout the whole Swing, SWT and GEF implementation. They
wrap and specify graphic components in a form we need for our \gdtesterclasses. You
can write your own adapter or reuse our existing adaptors. We recommend
reusing our adapters to take advantage of the already existing code.

Before we have a look at the \app{} remote control examples in section
\ref{remoteControlExamples}, we describe in general, how to create an
adapter and \gdtesterclasses.

\subsubsection{Creating an adapter}

For using an adapter you have to implement the interface
\bxshell{IAdapterFactory}, which is defined in the package named\\
\bxname{org.eclipse.jubula.rc.common.adaptable}.\\
It is necessary, that the implementation of this interface resists in a package
named\\
\bxname{org.eclipse.jubula.rc.common.adapter}.\\
The package naming is important for the \app{} remote control, because adapters
can only be found, if they are located in a package with the mentioned name.

In order to support your own components, the targeted type must be an instance
of \bxshell{IComponent}, which is defined in the package named\\
\bxname{org.eclipse.jubula.rc.common.tester.adapter.interfaces}.\\
This targeted type can be used for text components. If you want to support your
own graphical component, you should at least use the interface
\bxshell{IWidgetComponent} as the targeted type, which is a child of
\bxshell{IComponent}. Both interfaces are defined in the same package.

Now we know, how to create an own adapter. The next step is to implement one
or more \gdtesterclass.

\subsubsection{Creating a \gdtesterclass}

The functionally important aspect of a \gdtesterclass is, that it
contains public methods for the \app{} test actions, which will appear in
the client. These methods are linked to testable actions within a user-defined
\app{} client plug-in. Each \app{} client extension provides an XML
configuration file, which defines the available methods with parameters, as well as
information for string externalization.

You can write your own \gdtesterclasses, but if you only want to support a new
component with existing actions, you could use one of our existing
\gdtesterclasses, which you can find in the package named\\
\bxname{org.eclipse.jubula.rc.common.tester}.

Please take the following guidelines into consideration for creating
\gdtesterclasses:
\begin{itemize}
  \item The class must be compatible with \bxname{Java 1.4}.
  \item It must either implement
  \begin{itemize}
    \item the interface \bxshell{ITester} defined in the package named\\
        \bxname{org.eclipse.jubula.rc.common.tester.interfaces},\\
	 \item or the abstract class \bxshell{AbsstractUITester}
	     defined in the package named\\
	     \bxname{org.eclipse.jubula.rc.common.tester}.
  \end{itemize}
  \item The component must provide \textbf{public} methods for all actions
        declared by the XML configuration file of the corresponding \app{}
        client extension.
  \item Each method, that implements an action, must throw a\\
        \bxshell{StepExecutionException}\\
        defined in the package named\\
        \bxname{org.eclipse.jubula.rc.common.exception}
        to notify \app{}, if executing of the action has been failed.
\end{itemize}

\subsection{Deploying the fragment}

Once you have written your fragment, you still need to make \app{}
aware of its presence. This is done by exporting the fragment and
deploying it into the plug-ins folder from \app{} or the \gdagent.
After you have done this you must change some configurations.

If you only want to deploy it to the \gdagent you have to modify
the file \bxshell{config.ini} located in the directory\\
\bxshell{server/configuration/}\\
by appending \bxshell{,Bundle\_name@start} at the end of the line starting with
\bxshell{osgi.bundles=}. Make sure, that the bundles are divided by comma.

If you have installed the Feature and done the steps for the fragment, you
should be able to test your component.

\section{\app{} example extension}
\label{remoteControlExamples}

\app{} comes with examples for extending supported toolkits including the
source code. The examples can be found in the directory\\
\bxshell{examples/development/extension},
which contains the following sub directories:

\begin{description}
 \item[aut] This directory contains a simple example AUT, which uses the
 originally unsupported component \bxshell{JSlider}. After installing the
 extensions, \app{} will be able to test this new component.
 \item[src] In this directory you will find some zip-files containing
 projects, which can be imported by Eclipse:
 \begin{itemize}
   \item \bxshell{eclipseProjects\_ExampleSwingAUT.zip} \\
   This is the source code project for the example Swing AUT.
   \item \bxshell{eclipseProjects\_ExampleSwingClientExtension.zip} \\
   This is the source code project of the extension plug-in for the \app{}
   client.
   \item \bxshell{eclipseProjects\_ExampleSwingServerExtension.zip} \\
   This is the source code project of the extension for the \app{} server.
 \end{itemize}
\end{description}

\subsection{Example for extending Swing toolkit support}

In the examples you will find an extension to support the Swing component
\bxshell{JSlider} in \app{}. After deploying the \app{} client extension and the
\app{} remote control extension you should be able to test the Swing component
\bxshell{JSlider} within \app{}.

\subsection{Example for extending SWT toolkit support}

\section{Migration steps}
If you have an existing implementation, which inherits from our old classes,
you have to change them. We have modified our testerclasses in a way, that they
do not use the concrete components anymore. Now the testerclasses take
advantage of the adapter class instead. If you want to migrate your old
classes, you must implement the new adapter described in the previous section.
